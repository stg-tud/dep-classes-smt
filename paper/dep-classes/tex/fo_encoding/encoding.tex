\documentclass[a4paper]{article}

%% packages
%%%%%%%%%%%
\usepackage[utf8x]{inputenc}
\usepackage[USenglish]{babel}
\usepackage{amsbsy,amscd,amsfonts,amssymb,amstext,amsmath,amsthm,latexsym}
\usepackage{mathpartir}
\usepackage{stmaryrd}
\usepackage{dot2texi}
\usepackage{url}
\usepackage{hyperref}
\usepackage[nottoc]{tocbibind}
\usepackage{pdfpages}

\usepackage{syntax} % for bnf grammar
\usepackage{bussproofs} % for type rules
\usepackage{todonotes}

\usepackage{float} % for figures
\floatstyle{boxed}
\restylefloat{figure}

\usepackage{listings}
\usepackage{xcolor}
%\usepackage{tikz}
\usetikzlibrary{positioning,chains,shapes.arrows,shapes.geometric,fit,calc,arrows,decorations.pathmorphing}

%\usepackage{subfig}
\usepackage{caption}
\usepackage{subcaption}
\usepackage{adjustbox}
\usepackage{cleveref}

%% options
%%%%%%%%%%
% Currently no such things seem to be needed.
% \newtheorem{definition}{Definition}
% \newtheorem{example}{Example}
% \newtheorem{lemma}{Lemma}
% \newtheorem{theorem}{Theorem}
% \newtheorem{claim}{Claim}
% \numberwithin{definition}{chapter}

%% macros
%%%%%%%%%
\newcommand{\loremipsum}{\todo[inline]{Lorem Ipsum}}
%\newcommand{\nat}[1]{\ensuremath{\lceil\text{#1}\rceil}}
\newcommand{\gprod}[1]{\ensuremath{\langle\text{#1}\rangle}}
\newcommand{\TNum}{\ensuremath{\mathcal{N}}}
\newcommand{\TBool}{\ensuremath{\mathcal{B}}}
\newcommand{\TInt}{\ensuremath{\mathcal{N}}}

\theoremstyle{definition}
\newtheorem{example}{Example}[chapter]
\newtheorem{definition}[example]{Definition}
%\newtheorem{definition}{Definition}[chapter]

% DCC
% DCC constraints
\newcommand{\pathEq}[2]{\ensuremath{#1 \equiv #2}}
\newcommand{\instanceOf}[2]{\ensuremath{#1 :: #2}}
\newcommand{\instOf}[2]{\instanceOf{#1}{#2}}
\newcommand{\instantiatedBy}[2]{\ensuremath{#1.\textbf{cls} \equiv #2}}
\newcommand{\instBy}[2]{\instantiatedBy{#1}{#2}}
\newcommand{\entails}[2]{\ensuremath{#1 \vdash #2}}
\newcommand*{\ldblbrace}{\{\mskip-6.9mu\{} % left double brace
\newcommand*{\rdblbrace}{\}\mskip-6.9mu\}} % right double brace
\newcommand{\sub}[2]{\ensuremath{\ldblbrace #1 \mapsto #2 \rdblbrace}}

% DCC declarations
\newcommand{\constructorDeclaration}[3]{\ensuremath{#1(#2.\ #3)}}
\newcommand{\constr}[3]{\constructorDeclaration{#1}{#2}{#3}}
\newcommand{\programEntailment}[3]{\ensuremath{ \forall #1.\ #2 \Rightarrow #3}}
\newcommand{\progEnt}[3]{\programEntailment{#1}{#2}{#3}}
\newcommand{\abstractMethodDeclaration}[4]{\ensuremath{#1(#2.\ #3): #4}}
\newcommand{\mDecl}[4]{\abstractMethodDeclaration{#1}{#2}{#3}{#4}}
\newcommand{\methodImplementation}[5]{\ensuremath{#1(#2.\ #3): #4 := #5}}
\newcommand{\mImpl}[5]{\methodImplementation{#1}{#2}{#3}{#4}{#5}}

% DCC expressions
\newcommand{\newInstance}[3]{\ensuremath{\textbf{new } #1(#2 \equiv #3)}}
\newcommand{\newInst}[3]{\newInstance{#1}{#2}{#3}}

% DCC misc
\newcommand{\obj}[3]{\ensuremath{\langle #1; #2 \equiv #3 \rangle}}
\newcommand*{\stdobj}{\ensuremath{\obj{C}{\overline{f}}{\overline{x}}}}
\newcommand{\heap}[2]{\ensuremath{#1 \mapsto #2}}
\newcommand*{\stdheap}{\ensuremath{\heap{\overline{x}}{\overline{o}}}}
\newcommand{\pair}[2]{\ensuremath{\langle #1;#2 \rangle}}
\newcommand{\eval}[4]{\ensuremath{\pair{#1}{#2} \rightarrow \pair{#3}{#4}}}
\newcommand{\typeass}[3]{\ensuremath{#1 \vdash #2 : #3}}
\newcommand{\wf}[1]{\ensuremath{\text{wf }#1}}
\newcommand{\FV}[1]{\ensuremath{FV(#1)}}
\newcommand{\FVeq}[2]{\ensuremath{\FV{#1} = \{#2\}}}

%% misc
%%%%%%%%%
\DeclareMathOperator{\pto}{\rightharpoonup}     % partial function arrow
\DeclareMathOperator{\powerset}{\mathcal{P}}    % powerset
\DeclareMathOperator{\proj}{\upharpoonright}    % projection
\DeclareMathOperator{\fst}{\pi_1}               % A x B → A
\DeclareMathOperator{\snd}{\pi_2}               % A x B → B
\newcommand{\ovl}[1]{\ensuremath{\overline{#1}}}

%%% Local Variables: 
%%% mode: latex
%%% TeX-master: "../thesis"
%%% End: 

%
\usetikzlibrary{positioning,chains,fit,calc,arrows,decorations.pathmorphing}
\usetikzlibrary{shapes.arrows,shapes.geometric,shapes.symbols}


%%%%%%%
%% tikz stuff
%%%%%%%


\tikzstyle{invisible} = []

\tikzstyle{model}
  = [ shape=rectangle
    , draw
    ]
\tikzstyle{transformation}
  = [ model
    , shape=single arrow
    , draw
    ]

\tikzstyle{meta} = [ double ]
\tikzstyle{generated} = [ dashed ]

\tikzstyle{mopdependency}
  = [ -stealth'
    , semithick
    ]
\tikzstyle{instance}
  = [ mopdependency
    , -open triangle 60
    ]

% Draws the outline of a bend arrow around a rectangle.
% 1. Parameter: name of the rectangle that contains the text
% 2. Parameter: head extend (use 0.25cm to fit with default tikz style)
% 3. Parameter: arc radius (again, 0.25cm seems to work well)
%
% This produces a path, so use it as follows:
%
% \draw[options here] \bendArrow{name of rectangle}{.25cm}{.25cm};
\newcommand{\bendArrow}[3]{
  let \p{content size} = ($(#1.north east) - (#1.south west)$) in
  let \n{tip length 1} = {veclen(0.5 * \y{content size}, 0.5 * \y{content size})} in
  let \n{tip length 2} = {veclen(#2,#2)} in

  (#1.north east)
  -- ++ (315 : \n{tip length 1})
  -- ++ (225 : \n{tip length 1} + \n{tip length 2})
  -- ++ (0, #2)
  -- (#1.south west)
  arc (270 : 180 : \y{content size} + #3)
  -- ++ (\y{content size}, 0)
  arc (180 : 270 : #3)
  (#1.north east)
  -- ++ (135 : \n{tip length 2})
  -- ++ (0, -#2)
  -- (#1.north west)}


%%%%%%%%%%%%%%%%%
%% diagram styles
%%%%%%%%%%%%%%%%%

\tikzstyle{document}
  = [ shape=rectangle
    , draw
    , minimum height=1.5em
    , text width=5em
    , text centered
    ]

\tikzstyle{component}
  = [ font=\scriptsize\it
    ]

\tikzstyle{code}
  = [ shape=rectangle
    , draw
    ]

\tikzstyle{process}
  = [ shape=single arrow
    , single arrow head extend=.75em
    , single arrow head indent=.25em
    , minimum width=3em
    , draw
    ]

\tikzstyle{point}
  = [ coordinate
    , minimum width=1em
    ]

\tikzstyle{flow diagram}
  = [ start chain
    , node distance=1em
    , every node/.style={on chain}
    ]

\tikzstyle{ast}
  = [ node distance=1em and 0.38em
    , every node/.style=ast node
    ]

\tikzstyle{ast node}
  = [ shape=circle
    , minimum size=0.5em
    , inner sep=0
    , fill
    ]

\tikzstyle{dependency}
  = [ dashed
    , -stealth'
    ]

\tikzstyle{red}
  = [ shape=rectangle
    , color=red
    ]

\tikzstyle{blue}
  = [ shape=circle
    , color=blue
    ]

\tikzstyle{green}
  = [ shape=diamond
    , color=green!80!black
    , minimum size=0.7em
    ]

\tikzstyle{note}
  = [ font=\scriptsize\it
    ]
    
\tikzstyle{zoomed arrow}
  = [ solid
    , decorate
    , decoration=snake
    , -stealth'
    ]

\tikzstyle{zoom}
  = [ dashed
    ]

\tikzstyle{uml class}
  = [ shape=rectangle
    , draw
    , font=\sf
    ]

\tikzstyle{uml package}
  = [ uml class
    , inner sep=1em
    ]

\tikzstyle{uml dependency}
  = [ dependency, 
    , dashed
%    , thick
    ]


\newcommand{\bluenode}{\tikz \node [ast node, blue, color=blue, text width=] {};}
\newcommand{\rednode}{\tikz \node [ast node, red, color=red, text width=] {};}
\newcommand{\greennode}{\tikz \node [ast node, green, color=green!80!black, text width=] {};}


\tikzstyle{double arrow}
  = [ shape=double arrow
    , double arrow head extend=.75em
    , double arrow head indent=.25em
    , minimum width=3em
    , draw
    , font=\sf
    ]

%% name graphs

\newdimen\nodedistance
\nodedistance=4em

\tikzstyle{name graph}
  = [ node distance=\nodedistance
    , every node/.style={namenode}
    , every path/.style={ref}
    ]


\tikzstyle{namenode}
  = [ circle
    , thick
    , anchor=center
    , draw
    , minimum size=2em
    , inner sep=2pt
    ]

\tikzstyle{synthesized}
  = [ namenode
    , fill=gray!45
    ]

\tikzstyle{ref}
  = [ -stealth'
    , semithick
    , draw
    ]

\tikzstyle{badref}
  = [ ref
    , dashed
    ]

%% expression trees

\newdimen\nodedistance
\nodedistance=4em

\tikzstyle{exp tree}
  = [ node distance=\nodedistance
    , every node/.style={exp node}
    , every path/.style={link}
    , execute at begin node={\strut}
    ]

\tikzstyle{exp node}
  = [ anchor=center
    ]
\tikzstyle{link}
  = [ semithick
    , draw
    ]

\tikzstyle{ctx node}
  = [ node distance = -.3em,
    ]
\tikzstyle{ctx edge}
  = [ color = red
    , -stealth'
    ]

\tikzstyle{type node}
  = [ node distance = -.5em,
    ]
\tikzstyle{type edge}
  = [ color = blue
    , -stealth'
    ]

%%%%%%%%%%%%%%%%%%%%%%%%%%%
% Pretty printing with tikz
%
% configuration

%%%%%%
%% TikZ and lstlistings
%%%%%%


% remembers a position on the page as a tikz coordinate
\newcommand{\coord}[1]{\tikz[remember picture] \coordinate (#1);}

% translation from baseline to upper border of box
\newcommand{\distanceTop}{7.15pt}

% translation from baseline to lower border of box
\newcommand{\distanceBottom}{-2.55pt}

% translation from baseline to left border of box
\newcommand{\distanceLeft}{-0.5pt}

% translation from baseline to right border of box
\newcommand{\distanceRight}{0.5pt}


% draws a rectangle around 2 points on the page
%
% optional parameter: TikZ style for the line
% 1. mandatory parameter: upper left corner, at text baseline height
% 2. mandatory parameter: lower right corner, at text baseline height
% 3. mandatory parameter: extra margin in all directions
\newcommand{\drawrect}[4][]{\begin{tikzpicture}[remember picture, overlay]
\draw[layout box, #1]
  ($(#2) +(\distanceLeft, \distanceTop) + (-#4, #4)$) rectangle
  ($(#3) + (\distanceRight, \distanceBottom) + (#4, -#4)$);
\end{tikzpicture}}

% draws a rectangle around 2 points on the page
%
% optional parameter: TikZ style for the line
% 1. mandatory parameter: left-hand-side of the first line, at text baseline height
% 2. mandatory parameter: a point on the left side of the rectangle
% 3. mandatory parameter: lower right corner, at text baseline height
% 4. mandatory parameter: extra margin in all directions
\newcommand{\drawbox}[5][]{\begin{tikzpicture}[remember picture, overlay]
\draw[layout box, #1]
  ($(#2) + (\distanceLeft, \distanceTop) + (-#5, #5)$) --
  ($(#4 |- #2) + (\distanceRight, \distanceTop) + (#5, #5)$) --
  ($(#4) + (\distanceRight, \distanceBottom) + (#5, -#5)$) --
  ($(#3 |- #4) + (\distanceLeft, \distanceBottom) + (-#5, -#5)$) --
  ($(#3 |- #2) + (\distanceLeft, \distanceBottom) + (-#5, -#5)$) --
  ($(#2) + (\distanceLeft, \distanceBottom) + (-#5, -#5)$) --
  cycle;
\end{tikzpicture}}

% draws a small \ppBox with three rows around the current point
% #1 = first.col
% #2 = left.col
% #3 = right.col
% #4 = last.col
\newcommand{\drawMiniBox}[4]{
  +(#2 * .2em, 0) --
  +(#2 * .2em, 1ex) --
  +(#1 * .2em, 1ex) --
  +(#1 * .2em, 1.5ex) --
  +(#3 * .2em, 1.5ex) --
  +(#3 * .2em, .5ex) --
  +(#4 * .2em, .5ex) --
  +(#4 * .2em, 0) --
  cycle
}

% A small box (for use inside text)
% optional argument = path options
% #1 = first.col
% #2 = left.col
% #3 = right.col
% #4 = last.col
\newcommand{\minibox}[5][]{%
  \begin{tikzpicture}[baseline=0pt]
  \path [mini layout box, #1]
    (0, 0) \drawMiniBox{#2}{#3}{#4}{#5};
  \end{tikzpicture}}

\newcommand{\miniboxA}{\minibox{-1}{0}{3}{3}}
\newcommand{\miniboxB}{\minibox{0}{0}{3}{3}}
\newcommand{\miniboxC}{\minibox{1}{0}{3}{3}}

% A small rectangle (for use inside text)
\newcommand{\minirect}{\hbox to 9pt{\drawrect[thin,scale=0.3,black]{0pt, 15pt}{26pt, -5pt}{0pt}}}


\tikzstyle{layout box}
  = [ blue
    % , semitransparent
    , semithick
    , draw
    ]

\tikzstyle{mini layout box}
  = [ black
    , very thin
    , draw
    ]

\tikzstyle{annotation}
  = [ font=\small\it
    ]

\tikzstyle{code annotation}
  = [ font=\small\it
    ]


% the font used for pretty printing
\newcommand{\selectCodeFont}{\sf\small}

% the font used for highlighting keywords
\newcommand{\selectKeywordFont}{\bfseries}

% the font used for highlighting string literals
\newcommand{\selectStringLitFont}{\tt}

% the font used for highlighting identifiers
\newcommand{\selectIdentifierFont}{\relax}

% the font used for highlighting operators
\newcommand{\selectOperatorFont}{\tt}

% the text height of token nodes
\newcommand{\tokenHeight}{2ex}

% the text depth of token nodes
\newcommand{\tokenDepth}{.5ex}

%%%%%%%%%%%%%%%%%%%%%%%%%%%
% Pretty printing with tikz
%
% tikz layer

% style for paths or scopes that do pretty printing
\tikzstyle{pretty print}
  = [ start chain=going base right
    , text height=\tokenHeight
    , text depth=\tokenDepth
    , inner sep=0
    , node distance=0em
    ]

% helper style for nodes that start a new line
% (automatically applied by style 'new line' below)
\tikzstyle{new line/helper}
  = [ on chain
    , anchor=north west,
      at={($(#1.west |- \tikzchainprevious.base) + (0, -1.1\baselineskip)$)}
    ]

\tikzstyle{new line tight/helper}
  = [ on chain
    , anchor=north west,
      at={($(#1.west |- \tikzchainprevious.base) + (0, -0.9\baselineskip)$)}
    ]


% style for nodes that start a new line
\tikzstyle{new line}[\tikzchainprevious]
  = [ on chain=placed {new line/helper=#1}
    ]

% style for nodes that start a new line
\tikzstyle{new line tight}[\tikzchainprevious]
  = [ on chain=placed {new line tight/helper=#1}
    ]

% helper style for nodes that tab forward to a labeled position
% (automatically applied by style 'tab forward' below)
\tikzstyle{tab forward/helper}
  = [ on chain
    , anchor=north west,
      at={(#1 |- \tikzchainprevious.base)}
    ]

% style for nodes that tab forward to a labeled position
\tikzstyle{tab forward}
  = [ on chain=placed {tab forward/helper=#1}
    ]

% style for nodes that indent
\tikzstyle{indent}[1em]
  = [ on chain=placed {base right=#1 of \tikzchainprevious}
    ]

% style for nodes that contain an identifier token
\tikzstyle{identifier}
  = [ on chain
    , font=\selectIdentifierFont
    ]

% style for nodes that contain an operator token
\tikzstyle{operator}
  = [ on chain
    , font=\selectOperatorFont
    ]

% style for nodes that contain a keyword token
\tikzstyle{keyword}
  = [ on chain
    , font=\selectKeywordFont
    , text=keyword
    ]

% style for nodes that contain a string literal
\tikzstyle{stringlit}
  = [ on chain
    , font=\selectStringLitFont
    , text=blue
    ]

%%%%%%%%%%%%%%%%%%%%%%%%%%%
% Pretty printing with tikz
%
% \ppBox layer

\newcommand{\Empty}{}
\newcommand{\ignore}[1]{\relax}

\makeatletter
\newcommand{\ppBox}[2][]{{
  % insert keyword token
  \newcommand{\KW}[1]{
    \node[keyword] {##1};
    \HandleToken{\tikzchaincurrent}}

  % insert identifier token
  \newcommand{\ID}[1]{
    \node[identifier] {##1};
    \HandleToken{\tikzchaincurrent}}

  % insert operator token
  \newcommand{\OP}[1]{
    \node[operator] {##1};
    \HandleToken{\tikzchaincurrent}}

  % insert whitespace token
  \newcommand{\SP}{
    \node[on chain] { };
    \HandleInsensitiveToken{\tikzchaincurrent}}

  % label a position (with a tikz node name)
  \newcommand{\LB}[1]{
    \coordinate[on chain] (##1);}

  % tab forward to a labeled position
  \newcommand{\TB}[1]{
    \coordinate[tab forward=##1];
  }

  % insert a string literal
  \newcommand{\STR}[1]{
    \node[stringlit] {##1};
    \HandleToken{\tikzchaincurrent}}

  % insert a line break
  \renewcommand{\\}{
    \coordinate[new line=\NewlineNode];
    \let\HandleToken=\HandleTokenSubsequentLine}

  \newcommand{\newlineTight}{
    \coordinate[new line tight=\NewlineNode];
    \let\HandleToken=\HandleTokenSubsequentLine}

  % the name of the first token
  \let\FirstToken=\Empty
  \newcommand{\AdjustFirst}[1]{
    \ifx\FirstToken\Empty
    \edef\FirstToken{##1}
    \fi
  }

  % the name of the last token
  \let\LastToken=\Empty
  \newcommand{\AdjustLast}[1]{
    \edef\LastToken{##1}
  }

  % the name of the left-most token
  % (except for tokens in the first line)
  \let\LeftToken=\Empty
  \newcommand{\AdjustLeft}[1]{
    \ifx\LeftToken\Empty
      \edef\LeftToken{##1}
    \else
      \pgf@process{\pgfpointanchor{##1}{west}}
      \setlength{\pgf@xa}{\pgf@x}
      \pgf@process{\pgfpointanchor{\LeftToken}{west}}
      \setlength{\pgf@xb}{\pgf@x}
      \ifdim\pgf@xa<\pgf@xb
        \edef\LeftToken{##1}
      \fi
    \fi
  }

  % the name of the rightmost token
  \let\RightToken=\Empty
  \newcommand{\AdjustRight}[1]{
    \ifx\RightToken\Empty
      \edef\RightToken{##1}
    \else
      \pgf@process{\pgfpointanchor{##1}{east}}
      \setlength{\pgf@xa}{\pgf@x}
      \pgf@process{\pgfpointanchor{\RightToken}{east}}
      \setlength{\pgf@xb}{\pgf@x}
      \ifdim\pgf@xa>\pgf@xb
        \edef\RightToken{##1}
      \fi
    \fi
  }

  % this is called for all layout-sensitive tokens in the first
  % line of a pretty printing group
  \newcommand{\HandleTokenFirstLine}[1]{
    \AdjustFirst{##1}
    \AdjustRight{##1}
    \AdjustLast{##1}
  }

  % this is called for all layout-sensitive tokens in subsequent
  % lines of a pretty printing group
  \newcommand{\HandleTokenSubsequentLine}[1]{
    \AdjustLeft{##1}
    \AdjustRight{##1}
    \AdjustLast{##1}}

  % this is called for layout-insensitive tokens in a pretty
  % printing group (such as whitespace, comments, and code that
  % uses explicit layout)
  \newcommand{\HandleInsensitiveToken}[1]{
  }

  % current token handler. will be set to one of:
  %  \HandleTokenFirstLine
  %  \HandleTokenSubsequentLine
  %  \HandleInsensitiveToken
  \newcommand{\HandleToken}[1]{}

  \newcommand{\ppSubBox}[2][]{
    \coordinate[on chain];
    {
      \let\NewlineNode\tikzchaincurrent

      % reset the token registers
      \let\FirstToken=\Empty
      \let\LastToken=\Empty
      \let\LeftToken=\Empty
      \let\RightToken=\Empty

      % start in first line
      \let\HandleToken=\HandleTokenFirstLine

      % typeset content
      ##2

      % draw boundary
      \ifx\LeftToken\Empty
        \path [##1]
          (\FirstToken.north west) rectangle
          (\LastToken.south east);
      \else
        \path [##1]
          (\RightToken.east |- \FirstToken.north) --
          (\FirstToken.north west) --
          (\FirstToken.south west) --
          (\LeftToken.west |- \FirstToken.south) --
          (\LeftToken.west |- \LastToken.south) --
          (\LastToken.south east) --
          (\LastToken.north east) --
          (\RightToken.east |- \LastToken.north) --
          cycle;
      \fi

      % remember the token registers outside the group
      \global\let\InnerFirstToken=\FirstToken
      \global\let\InnerLastToken=\LastToken
      \global\let\InnerLeftToken=\LeftToken
      \global\let\InnerRightToken=\RightToken

      % remember the state outside the group
      \global\let\InnerHandleToken=\HandleToken
    }
    % handle the significant inner tokens
    % and significant inner linebreaks
    \HandleToken{\InnerFirstToken}
    \ifx\HandleToken\HandleTokenFirstLine
    \let\HandleToken=\InnerHandleToken
    \fi
    \ifx\InnerLeftToken\Empty
    \else
    \HandleToken{\InnerLeftToken}
    \fi
    \HandleToken{\InnerRightToken}
    \HandleToken{\InnerLastToken}
  }

  % indent a ppBox
  \newcommand{\IN}[1]{\coordinate[indent]; ##1}
  \newcommand{\DE}[1]{\coordinate[indent=-1em]; ##1}

  \let\ppBox=\ppSubBox
  \ppBox[#1]{#2}}}
\makeatother




%%% Local Variables: 
%%% mode: latex
%%% TeX-master: "../thesis"
%%% End: 


\begin{document}
% Table of contents
%%%%%%%%%%%%%%%%%%%
%\tableofcontents

% Content
%%%%%%%%%
\section{How do we check entailments?}
\begin{enumerate}
  \item Create set of ``axioms'' based on the program and the entailment to check
  \begin{enumerate}
    \item Datatype Declarations
    \begin{itemize}
      \item Enumeration types: Variables, Classes, Fields
      \item ADT: Path
    \end{itemize}
    \item Function Declarations
    \begin{itemize}
      \item Declared: path-equivalence, instance-of, instantiated-by
      \item Defined recursively: substitute
    \end{itemize}
    \item Calculus rules as all quantified formulas
    \begin{itemize}
      \item Static Rules: C-Refl, C-Class, C-Subst
      \item Template Rules: C-Prog
    \end{itemize}
    \item Assert entailment to be checked: $c_1,...,c_n \vdash c \Rightarrow \neg (c_1 \land ... \land c_n → c)$
  \end{enumerate}
  \item Obtain Solution: Does the entailment contradict the rules?
  \begin{itemize}
    \item If unsat: valid/correct entailment.
    \item If sat: invalid entailment.
  \end{itemize}
\end{enumerate}

\subsection{General First-Order Encoding Template}
\begin{align}
  %&\texttt{\Variable~(...)}~~~~~\texttt{\Field~(...)}~~~~~\texttt{\Class~(...)}\\
  &\texttt{\Variable~(...)}\\
  &\texttt{\Field~(...)}\\
  &\texttt{\Class~(...)}\\
  &\Path~((\Variable)~(\Path.\Field))\\
  &\pathEq{}{}\!\!: \Path \times \Path \rightarrow \TBool \\
  &\instOf{}{}\!\!: \Path \times \Class \rightarrow \TBool\\
  &\instBy{}{}\!\!: \Path \times \Class \rightarrow \TBool\\
  &{\_}_{\sub{}{}}\!\!: \Path \times \Variable \times \Path \rightarrow \Path\\
  % C-Refl
  &\forall \sortedVar{p}{\Path}.~\pathEq{p}{p}\\
  % C-Class
  &\forall \sortedVar{a}{\TBool}, \sortedVar{p}{\Path}, \sortedVar{c}{\Class}.\\
  &\quad(a \rightarrow \instBy{p}{c}) \rightarrow (a \rightarrow \instOf{p}{c})\\
  % C-Subst: pathEq
  &\forall \sortedVar{a}{\TBool}, \sortedVar{p}{\Path}, \sortedVar{q}{\Path}, \sortedVar{r}{\Path}, \sortedVar{s}{\Path}, \sortedVar{x}{\Variable}.\\
  &\quad (a \rightarrow \pathEq{p_{\sub{x}{r}}}{q_{\sub{x}{r}}} \land (a \rightarrow \pathEq{s}{r})) \rightarrow\\
  &\qquad (a \rightarrow \pathEq{p_{\sub{x}{s}}}{q_{\sub{x}{s}}})\\
  % C-Subst: instOf
  &\forall \sortedVar{a}{\TBool}, \sortedVar{p}{\Path}, \sortedVar{c}{\Class}, \sortedVar{r}{\Path}, \sortedVar{s}{\Path}, \sortedVar{x}{\Variable}.\\
  &\quad (a \rightarrow \instOf{p_{\sub{x}{r}}}{c} \land (a \rightarrow \pathEq{s}{r})) \rightarrow\\
  &\qquad (a \rightarrow \instOf{p_{\sub{x}{s}}}{c})\\
  % C-Subst: instBy
  &\forall \sortedVar{a}{\TBool}, \sortedVar{p}{\Path}, \sortedVar{c}{\Class}, \sortedVar{r}{\Path}, \sortedVar{s}{\Path}, \sortedVar{x}{\Variable}.\\
  &\quad (a \rightarrow \instBy{p_{\sub{x}{r}}}{c} \land (a \rightarrow \pathEq{s}{r})) \rightarrow\\
  &\qquad (a \rightarrow \instBy{p_{\sub{x}{s}}}{c})\\
  % C-Prog
  &\forall \sortedVar{a}{\TBool}, \sortedVar{p}{\Path}.~(a \rightarrow \bigwedge \hole) \rightarrow (a \rightarrow \instOf{p}{\hole})
\end{align}
$(1)$, $(2)$ and $(3)$: Enumeration types, will be instantiated based on the program context.\\
$(4)$: Path datatype. A path is either a variable or a path followed by a field.\\
$(5)$, $(6)$ and $(7)$: Signatures of constraint predicates.\\
$(8)$: Signature of path substitution, $p_{\sub{x}{q}}$ can be read as substitute $x$ in $p$ with $q$.\\
$(9)$: Encoding for rule C-Refl.\\
$(10-11)$: Encoding for rule C-Class.\\
$(12-20)$: Encoding for rule C-Subst. There is one specialized rule per constraint type.\\
$(21)$: Template for rule C-Prog. Will be instantiated based on the entailment declarations defined in the program.

\subsection{Natural Numbers Program}
\begin{align}
&\constr{\texttt{Zero}}{x}{\epsilon}\\
&\constr{\texttt{Succ}}{x}{\instOf{x.p}{\texttt{Nat}}}\\
&\progEnt{x}{\instOf{x}{\texttt{Zero}}}{\instOf{x}{\texttt{Nat}}}\\
&\progEnt{x}{\instOf{x}{\texttt{Succ}}, \instOf{x.p}{\texttt{Nat}}}{\instOf{x}{\texttt{Nat}}}\\
&\mDecl{\texttt{prev}}{x}{\instOf{x}{\texttt{Nat}}}{\type{y}{\instOf{y}{\texttt{Nat}}}}\\
&\mImpl{\texttt{prev}}{x}{\instOf{x}{\texttt{Zero}}}{\type{y}{\instOf{y}{\texttt{Nat}}}}{\newInstNoArgs{\texttt{Zero}}}\\
&\mImpl{\texttt{prev}}{x}{\instOf{x}{\texttt{Succ}}, \instOf{x.p}{\texttt{Nat}}}{\type{y}{\instOf{y}{\texttt{Nat}}}}{x.p}
\end{align}

\subsection{C-Prog Rules for Natural Numbers Program}
Create C-Prog rules based on $(24)$ and $(25)$.
\begin{align}
  &\forall \sortedVar{a}{\TBool}, \sortedVar{p}{\Path}.\\
  &\quad (a \rightarrow \instOf{p}{\Zero}) \rightarrow\\
  &\qquad (a \rightarrow \instOf{p}{\Nat})\\
  &\forall \sortedVar{a}{\TBool}, \sortedVar{p}{\Path}.\\
  &\quad (a \rightarrow \instOf{p}{\Succ} \land \instOf{x.p_{\sub{x}{p}}}{\Nat}) \rightarrow\\
  &\qquad (a \rightarrow \instOf{p}{\Nat})
\end{align}

\section{Example Entailments}
\subsection{Working valid entailment}
$\entails{\instBy{p}{\Zero}}{\instOf{p}{\Nat}}$
\begin{itemize}
  \item Checks unsatisfiable
  \item Unsat Core: C-Class, C-Prog-Zero
\end{itemize}

\subsection{Working invalid entailment}
$\entails{\cdot}{\pathEq{x}{y}}$
\begin{itemize}
  \item Checks satisfiable
  \item Model: $\pathEq{p}{q} \doteq p=q$
\end{itemize}

$\entails{\pathEq{a}{b}}{\pathEq{a}{c}}$\\
  Checks satisfiable\\
  Model:
  \begin{align*}
    &\helper(\sortedVar{q}{\Path}) :=\\&\quad
    \ite{q=a}{a}{\\&\qquad
      \ite{q=c}{c}{\\&\qquad\quad
        \ite{q=x}{x}{\\&\qquad\qquad
          \ite{q=b}{b}{\\&\qquad\qquad\quad
            \ite{q=b.p}{b.p}{\\&\qquad\qquad\qquad
              \ite{q=a.b}{a.b}{c.p}
            }
          }
        }
      }
    }\\
    &\pathEq{\sortedVar{p}{\Path}}{\sortedVar{q}{\Path}} :=\\&\quad
    \Let    a_1 := \helper(p)=a.p \land \helper(q)=b.p\\
    &\quad\phantom{{}\Let{}} a_2 := \helper(p)=a.p \land \helper(q)=a.p\\
    &\quad\phantom{{}\Let{}} a_3 := \helper(p)=c.p \land \helper(q)=c.p\\
    &\quad\phantom{{}\Let{}} a_4 := \helper(p)=b.p \land \helper(q)=b.p\\
    &\quad\phantom{{}\Let{}} a_5 := \helper(p)=b.p \land \helper(q)=a.p\\
    &\quad\In
    \bigvee_{i\in\{j | 1\leq j\leq5\}} a_i\\
    &\quad\lor \helper(p)=a \land \helper(q)=a\\
    &\quad\lor \helper(p)=b \land \helper(q)=b\\
    &\quad\lor \helper(p)=b \land \helper(q)=a\\
    &\quad\lor \helper(p)=c \land \helper(q)=c\\
    &\quad\lor \helper(p)=x \land \helper(q)=x\\
    &\quad\lor \helper(p)=a \land \helper(q)=b\\
  \end{align*}

\subsection{Non-working invalid entailment}
\label{example:field-access-timeout}
$\entails{\instOf{x}{\Succ}, \instOf{x.p}{\Zero}}{\instOf{x}{\Zero}}$
\begin{itemize}
  \item Location: /paper/dep-classes/smt/semantic_entailment/fieldAccessTimeout.smt
  \item Solver has ``infinite'' runtime.
  \item Wanted behavior: Checks satisfiable.
  \item Current Solution: Impose a timeout on the solver.
  \begin{itemize}
    \item Assume an entailment to be invalid if the solver times out (to retain soundness).
    \item The system isn't complete anymore.
    \item Balance the timeout between the amount of checkable valid entailments and tolerable waiting time.
    (Set it such that the majority of valid entailments can still be checked.)
  \end{itemize}
\end{itemize}

\section{Detailed look at infinite runtime example}
\subsection{Formulae generated for \ref{example:field-access-timeout}}
$\entails{\instOf{x}{\Succ}, \instOf{x.p}{\Zero}}{\instOf{x}{\Zero}}$
\begin{align}
  &\texttt{\Variable} := \{\mathtt{x}\}\\
  &\texttt{\Field} := \{\mathtt{p}\}\\
  &\texttt{\Class} := \{\Zero,\Succ,\Nat\}\\
  &\Path := ((\Variable)~(\Path.\Field))\\
  % C-Refl
  &\forall \sortedVar{p}{\Path}.~\pathEq{p}{p}\\
  % C-Class
  &\forall \sortedVar{a}{\TBool}, \sortedVar{p}{\Path}, \sortedVar{c}{\Class}.\\
  &\quad(a \rightarrow \instBy{p}{c}) \rightarrow (a \rightarrow \instOf{p}{c})\\
  % C-Subst: pathEq
  &\forall \sortedVar{a}{\TBool}, \sortedVar{p}{\Path}, \sortedVar{q}{\Path}, \sortedVar{r}{\Path}, \sortedVar{s}{\Path}, \sortedVar{x}{\Variable}.\\
  &\quad (a \rightarrow \pathEq{p_{\sub{x}{r}}}{q_{\sub{x}{r}}} \land (a \rightarrow \pathEq{s}{r})) \rightarrow\\
  &\qquad (a \rightarrow \pathEq{p_{\sub{x}{s}}}{q_{\sub{x}{s}}})\\
  % C-Subst: instOf
  &\forall \sortedVar{a}{\TBool}, \sortedVar{p}{\Path}, \sortedVar{c}{\Class}, \sortedVar{r}{\Path}, \sortedVar{s}{\Path}, \sortedVar{x}{\Variable}.\\
  &\quad (a \rightarrow \instOf{p_{\sub{x}{r}}}{c} \land (a \rightarrow \pathEq{s}{r})) \rightarrow\\
  &\qquad (a \rightarrow \instOf{p_{\sub{x}{s}}}{c})\\
  % C-Subst: instBy
  &\forall \sortedVar{a}{\TBool}, \sortedVar{p}{\Path}, \sortedVar{c}{\Class}, \sortedVar{r}{\Path}, \sortedVar{s}{\Path}, \sortedVar{x}{\Variable}.\\
  &\quad (a \rightarrow \instBy{p_{\sub{x}{r}}}{c} \land (a \rightarrow \pathEq{s}{r})) \rightarrow\\
  &\qquad (a \rightarrow \instBy{p_{\sub{x}{s}}}{c})\\
  % C-Prog: Zero -> Nat
  &\forall \sortedVar{a}{\TBool}, \sortedVar{p}{\Path}.~(a \rightarrow \instOf{p}{\Zero}) \rightarrow (a \rightarrow \instOf{p}{\Nat})\\
  % C-Prog: Succ -> Nat
  &\forall \sortedVar{a}{\TBool}, \sortedVar{q}{\Path}.~(a \rightarrow \instOf{q}{\Succ} \land \instOf{\mathtt{x.p}_{\sub{\mathtt{x}}{q}}}{\Nat}) \rightarrow (a \rightarrow \instOf{q}{\Nat})\\
  & \neg (\instOf{\mathtt{x}}{\Succ} \land \instOf{\mathtt{x.p}}{\Zero} \rightarrow \instOf{\mathtt{x}}{\Zero})
\end{align}

\subsection{Formulae generated for alpha renamed \ref{example:field-access-timeout}}
$\entails{\instOf{a}{\Succ}, \instOf{a.p}{\Zero}}{\instOf{a}{\Zero}}$
\begin{align}
  &\texttt{\Variable} := \{\mathtt{x}, \mathtt{a}\}\\
  &\texttt{\Field} := \{\mathtt{p}\}\\
  &\texttt{\Class} := \{\Zero,\Succ,\Nat\}\\
  &\Path := ((\Variable)~(\Path.\Field))\\
  % C-Refl
  &\forall \sortedVar{p}{\Path}.~\pathEq{p}{p}\\
  % C-Class
  &\forall \sortedVar{a}{\TBool}, \sortedVar{p}{\Path}, \sortedVar{c}{\Class}.\\
  &\quad(a \rightarrow \instBy{p}{c}) \rightarrow (a \rightarrow \instOf{p}{c})\\
  % C-Subst: pathEq
  &\forall \sortedVar{a}{\TBool}, \sortedVar{p}{\Path}, \sortedVar{q}{\Path}, \sortedVar{r}{\Path}, \sortedVar{s}{\Path}, \sortedVar{x}{\Variable}.\\
  &\quad (a \rightarrow \pathEq{p_{\sub{x}{r}}}{q_{\sub{x}{r}}} \land (a \rightarrow \pathEq{s}{r})) \rightarrow\\
  &\qquad (a \rightarrow \pathEq{p_{\sub{x}{s}}}{q_{\sub{x}{s}}})\\
  % C-Subst: instOf
  &\forall \sortedVar{a}{\TBool}, \sortedVar{p}{\Path}, \sortedVar{c}{\Class}, \sortedVar{r}{\Path}, \sortedVar{s}{\Path}, \sortedVar{x}{\Variable}.\\
  &\quad (a \rightarrow \instOf{p_{\sub{x}{r}}}{c} \land (a \rightarrow \pathEq{s}{r})) \rightarrow\\
  &\qquad (a \rightarrow \instOf{p_{\sub{x}{s}}}{c})\\
  % C-Subst: instBy
  &\forall \sortedVar{a}{\TBool}, \sortedVar{p}{\Path}, \sortedVar{c}{\Class}, \sortedVar{r}{\Path}, \sortedVar{s}{\Path}, \sortedVar{x}{\Variable}.\\
  &\quad (a \rightarrow \instBy{p_{\sub{x}{r}}}{c} \land (a \rightarrow \pathEq{s}{r})) \rightarrow\\
  &\qquad (a \rightarrow \instBy{p_{\sub{x}{s}}}{c})\\
  % C-Prog: Zero -> Nat
  &\forall \sortedVar{a}{\TBool}, \sortedVar{p}{\Path}.~(a \rightarrow \instOf{p}{\Zero}) \rightarrow (a \rightarrow \instOf{p}{\Nat})\\
  % C-Prog: Succ -> Nat
  &\forall \sortedVar{a}{\TBool}, \sortedVar{q}{\Path}.~(a \rightarrow \instOf{q}{\Succ} \land \instOf{\mathtt{x.p}_{\sub{\mathtt{x}}{q}}}{\Nat}) \rightarrow (a \rightarrow \instOf{q}{\Nat})\\
  & \neg (\instOf{\mathtt{a}}{\Succ} \land \instOf{\mathtt{a.p}}{\Zero} \rightarrow \instOf{\mathtt{a}}{\Zero})
\end{align}

\subsection{Get rid of encoding x as a enumeratable variable by directly substituting with variable path, or by adding quantified variable to C-Prog rules}
% TODO

\section{Undefinedness}
\subsection{Add expert knowledge to successfully check \ref{example:field-access-timeout}}
The solver is able to check the example satisfiable if we either
\begin{itemize}
  \item add $\neg \pathEq{x}{x.p}$ to the context, resulting in ${\instOf{x}{\Succ} \land \instOf{x.p}{\Zero} \land \neg \pathEq{x}{x.p}} \rightarrow {\instOf{x}{\Zero}}$ or
  \item additionally assert $\neg \pathEq{x}{x.p}$.
\end{itemize}
%TODO: investigate steps performed by the solver

\subsection{Asserting two mutually exclusive classes to the same path}
\label{section:mutually-exclusive-classes}
Asserting both $\instOf{x}{\Zero}$ and $\instOf{x}{\Succ}$ in the solver yields SAT.

This, at first, seems to be unintended behaviour.
In reality, this is not the kind of problem we want to solve with the encoding.
What we ask of the solver is if there exists a conflict between
the entailment to check (context implies conclusion)
and the asserted calculus rules.

In other words: If the entailment context allows this to be true,
it is not a problem of the encoding.
%the problem is a wrong annotation made by the programmer.

 \section{Faithful encoding}
 As concluded in \ref{section:mutually-exclusive-classes},
 we want the system to answer the question:
 ``does the entailment conform to the calculus rules'' or in other words:
 ``does the negation of the entailment contradict the calculus rules''.
 Therefore the fact that it is allowed for a path to be an instance of
 two mutually exclusive classes (e.g. two classes that are not in a subtype relation),
 is non-problematic as long as the entailment context doesn't explicitly forbid this.

% \section{Change entailment translation}
% Instead $\bigwedge c_i \rightarrow conclusion$, use multiple asserts.
% for $c_i$ in $context$. assert $c_i$
% assert $\neg conclusion$
%  Which logically is exactly the same.
% ¬(c_1 /\ ... /\ c_n -> c) ===
% ¬(¬(c_1 /\ ... /\ c_n) \/ c) ===
% ¬¬(c_1 /\ ... /\ c_n) /\ ¬c ===
% c_1 /\ ... /\ c_n /\ ¬c
% Logically this change is equivalent. ¬(/\_{i in 1 to n} c_i -> c) === (/\_{i in 1 to n} c_i) /\ ¬c
%
% TODO: rename variables such that there is no inference with the variables introduces
% through entailment declarations in the program
%
% TODO: look if it is possible to get rid of having the variables from entailment declarations
% in the Variable enumeration type in the logic encoding.
% Is it possible to directly substitute those?
% Maybe create multiple rules per entailment declaration?

% \section{Explicitly define constraint predicates}
% \subsection{Non-working invalid entailment}
% $\entails{\pathEq{x}{y}}{\pathEq{x}{z}}$
% \begin{itemize}
%   \item Checks unsatisfiable, but shouldn't
%   \item Unsat Core: C-Subst-PathEq
%   \item TODO: possible reason? (check instantiation of subst rule)
% \end{itemize}

\end{document}
