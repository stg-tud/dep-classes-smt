\documentclass[a4paper]{article}

%% packages
%%%%%%%%%%%
\usepackage[utf8x]{inputenc}
\usepackage[USenglish]{babel}
\usepackage{amsbsy,amscd,amsfonts,amssymb,amstext,amsmath,amsthm,latexsym}
\usepackage{mathpartir}
\usepackage{stmaryrd}
\usepackage{dot2texi}
\usepackage{url}
\usepackage{hyperref}
\usepackage[nottoc]{tocbibind}
\usepackage{pdfpages}

\usepackage{syntax} % for bnf grammar
\usepackage{bussproofs} % for type rules
\usepackage{todonotes}

\usepackage{float} % for figures
\floatstyle{boxed}
\restylefloat{figure}

\usepackage{listings}
\usepackage{xcolor}
%\usepackage{tikz}
\usetikzlibrary{positioning,chains,shapes.arrows,shapes.geometric,fit,calc,arrows,decorations.pathmorphing}

%\usepackage{subfig}
\usepackage{caption}
\usepackage{subcaption}
\usepackage{adjustbox}
\usepackage{cleveref}

%% macros
%%%%%%%%%
\newcommand{\loremipsum}{\todo[inline]{Lorem Ipsum}}
%\newcommand{\nat}[1]{\ensuremath{\lceil\text{#1}\rceil}}
\newcommand{\gprod}[1]{\ensuremath{\langle\text{#1}\rangle}}
\newcommand{\TNum}{\ensuremath{\mathcal{N}}}
\newcommand{\TBool}{\ensuremath{\mathcal{B}}}
\newcommand{\TInt}{\ensuremath{\mathcal{N}}}

\theoremstyle{definition}
\newtheorem{example}{Example}[chapter]
\newtheorem{definition}[example]{Definition}
%\newtheorem{definition}{Definition}[chapter]

% DCC
% DCC constraints
\newcommand{\pathEq}[2]{\ensuremath{#1 \equiv #2}}
\newcommand{\instanceOf}[2]{\ensuremath{#1 :: #2}}
\newcommand{\instOf}[2]{\instanceOf{#1}{#2}}
\newcommand{\instantiatedBy}[2]{\ensuremath{#1.\textbf{cls} \equiv #2}}
\newcommand{\instBy}[2]{\instantiatedBy{#1}{#2}}
\newcommand{\entails}[2]{\ensuremath{#1 \vdash #2}}
\newcommand*{\ldblbrace}{\{\mskip-6.9mu\{} % left double brace
\newcommand*{\rdblbrace}{\}\mskip-6.9mu\}} % right double brace
\newcommand{\sub}[2]{\ensuremath{\ldblbrace #1 \mapsto #2 \rdblbrace}}

% DCC declarations
\newcommand{\constructorDeclaration}[3]{\ensuremath{#1(#2.\ #3)}}
\newcommand{\constr}[3]{\constructorDeclaration{#1}{#2}{#3}}
\newcommand{\programEntailment}[3]{\ensuremath{ \forall #1.\ #2 \Rightarrow #3}}
\newcommand{\progEnt}[3]{\programEntailment{#1}{#2}{#3}}
\newcommand{\abstractMethodDeclaration}[4]{\ensuremath{#1(#2.\ #3): #4}}
\newcommand{\mDecl}[4]{\abstractMethodDeclaration{#1}{#2}{#3}{#4}}
\newcommand{\methodImplementation}[5]{\ensuremath{#1(#2.\ #3): #4 := #5}}
\newcommand{\mImpl}[5]{\methodImplementation{#1}{#2}{#3}{#4}{#5}}

% DCC expressions
\newcommand{\newInstance}[3]{\ensuremath{\textbf{new } #1(#2 \equiv #3)}}
\newcommand{\newInst}[3]{\newInstance{#1}{#2}{#3}}

% DCC misc
\newcommand{\obj}[3]{\ensuremath{\langle #1; #2 \equiv #3 \rangle}}
\newcommand*{\stdobj}{\ensuremath{\obj{C}{\overline{f}}{\overline{x}}}}
\newcommand{\heap}[2]{\ensuremath{#1 \mapsto #2}}
\newcommand*{\stdheap}{\ensuremath{\heap{\overline{x}}{\overline{o}}}}
\newcommand{\pair}[2]{\ensuremath{\langle #1;#2 \rangle}}
\newcommand{\eval}[4]{\ensuremath{\pair{#1}{#2} \rightarrow \pair{#3}{#4}}}
\newcommand{\typeass}[3]{\ensuremath{#1 \vdash #2 : #3}}
\newcommand{\wf}[1]{\ensuremath{\text{wf }#1}}
\newcommand{\FV}[1]{\ensuremath{FV(#1)}}
\newcommand{\FVeq}[2]{\ensuremath{\FV{#1} = \{#2\}}}

%% misc
%%%%%%%%%
\DeclareMathOperator{\pto}{\rightharpoonup}     % partial function arrow
\DeclareMathOperator{\powerset}{\mathcal{P}}    % powerset
\DeclareMathOperator{\proj}{\upharpoonright}    % projection
\DeclareMathOperator{\fst}{\pi_1}               % A x B → A
\DeclareMathOperator{\snd}{\pi_2}               % A x B → B
\newcommand{\ovl}[1]{\ensuremath{\overline{#1}}}

%%% Local Variables: 
%%% mode: latex
%%% TeX-master: "../thesis"
%%% End: 

%
\usetikzlibrary{positioning,chains,fit,calc,arrows,decorations.pathmorphing}
\usetikzlibrary{shapes.arrows,shapes.geometric,shapes.symbols}


%%%%%%%
%% tikz stuff
%%%%%%%


\tikzstyle{invisible} = []

\tikzstyle{model}
  = [ shape=rectangle
    , draw
    ]
\tikzstyle{transformation}
  = [ model
    , shape=single arrow
    , draw
    ]

\tikzstyle{meta} = [ double ]
\tikzstyle{generated} = [ dashed ]

\tikzstyle{mopdependency}
  = [ -stealth'
    , semithick
    ]
\tikzstyle{instance}
  = [ mopdependency
    , -open triangle 60
    ]

% Draws the outline of a bend arrow around a rectangle.
% 1. Parameter: name of the rectangle that contains the text
% 2. Parameter: head extend (use 0.25cm to fit with default tikz style)
% 3. Parameter: arc radius (again, 0.25cm seems to work well)
%
% This produces a path, so use it as follows:
%
% \draw[options here] \bendArrow{name of rectangle}{.25cm}{.25cm};
\newcommand{\bendArrow}[3]{
  let \p{content size} = ($(#1.north east) - (#1.south west)$) in
  let \n{tip length 1} = {veclen(0.5 * \y{content size}, 0.5 * \y{content size})} in
  let \n{tip length 2} = {veclen(#2,#2)} in

  (#1.north east)
  -- ++ (315 : \n{tip length 1})
  -- ++ (225 : \n{tip length 1} + \n{tip length 2})
  -- ++ (0, #2)
  -- (#1.south west)
  arc (270 : 180 : \y{content size} + #3)
  -- ++ (\y{content size}, 0)
  arc (180 : 270 : #3)
  (#1.north east)
  -- ++ (135 : \n{tip length 2})
  -- ++ (0, -#2)
  -- (#1.north west)}


%%%%%%%%%%%%%%%%%
%% diagram styles
%%%%%%%%%%%%%%%%%

\tikzstyle{document}
  = [ shape=rectangle
    , draw
    , minimum height=1.5em
    , text width=5em
    , text centered
    ]

\tikzstyle{component}
  = [ font=\scriptsize\it
    ]

\tikzstyle{code}
  = [ shape=rectangle
    , draw
    ]

\tikzstyle{process}
  = [ shape=single arrow
    , single arrow head extend=.75em
    , single arrow head indent=.25em
    , minimum width=3em
    , draw
    ]

\tikzstyle{point}
  = [ coordinate
    , minimum width=1em
    ]

\tikzstyle{flow diagram}
  = [ start chain
    , node distance=1em
    , every node/.style={on chain}
    ]

\tikzstyle{ast}
  = [ node distance=1em and 0.38em
    , every node/.style=ast node
    ]

\tikzstyle{ast node}
  = [ shape=circle
    , minimum size=0.5em
    , inner sep=0
    , fill
    ]

\tikzstyle{dependency}
  = [ dashed
    , -stealth'
    ]

\tikzstyle{red}
  = [ shape=rectangle
    , color=red
    ]

\tikzstyle{blue}
  = [ shape=circle
    , color=blue
    ]

\tikzstyle{green}
  = [ shape=diamond
    , color=green!80!black
    , minimum size=0.7em
    ]

\tikzstyle{note}
  = [ font=\scriptsize\it
    ]
    
\tikzstyle{zoomed arrow}
  = [ solid
    , decorate
    , decoration=snake
    , -stealth'
    ]

\tikzstyle{zoom}
  = [ dashed
    ]

\tikzstyle{uml class}
  = [ shape=rectangle
    , draw
    , font=\sf
    ]

\tikzstyle{uml package}
  = [ uml class
    , inner sep=1em
    ]

\tikzstyle{uml dependency}
  = [ dependency, 
    , dashed
%    , thick
    ]


\newcommand{\bluenode}{\tikz \node [ast node, blue, color=blue, text width=] {};}
\newcommand{\rednode}{\tikz \node [ast node, red, color=red, text width=] {};}
\newcommand{\greennode}{\tikz \node [ast node, green, color=green!80!black, text width=] {};}


\tikzstyle{double arrow}
  = [ shape=double arrow
    , double arrow head extend=.75em
    , double arrow head indent=.25em
    , minimum width=3em
    , draw
    , font=\sf
    ]

%% name graphs

\newdimen\nodedistance
\nodedistance=4em

\tikzstyle{name graph}
  = [ node distance=\nodedistance
    , every node/.style={namenode}
    , every path/.style={ref}
    ]


\tikzstyle{namenode}
  = [ circle
    , thick
    , anchor=center
    , draw
    , minimum size=2em
    , inner sep=2pt
    ]

\tikzstyle{synthesized}
  = [ namenode
    , fill=gray!45
    ]

\tikzstyle{ref}
  = [ -stealth'
    , semithick
    , draw
    ]

\tikzstyle{badref}
  = [ ref
    , dashed
    ]

%% expression trees

\newdimen\nodedistance
\nodedistance=4em

\tikzstyle{exp tree}
  = [ node distance=\nodedistance
    , every node/.style={exp node}
    , every path/.style={link}
    , execute at begin node={\strut}
    ]

\tikzstyle{exp node}
  = [ anchor=center
    ]
\tikzstyle{link}
  = [ semithick
    , draw
    ]

\tikzstyle{ctx node}
  = [ node distance = -.3em,
    ]
\tikzstyle{ctx edge}
  = [ color = red
    , -stealth'
    ]

\tikzstyle{type node}
  = [ node distance = -.5em,
    ]
\tikzstyle{type edge}
  = [ color = blue
    , -stealth'
    ]

%%%%%%%%%%%%%%%%%%%%%%%%%%%
% Pretty printing with tikz
%
% configuration

%%%%%%
%% TikZ and lstlistings
%%%%%%


% remembers a position on the page as a tikz coordinate
\newcommand{\coord}[1]{\tikz[remember picture] \coordinate (#1);}

% translation from baseline to upper border of box
\newcommand{\distanceTop}{7.15pt}

% translation from baseline to lower border of box
\newcommand{\distanceBottom}{-2.55pt}

% translation from baseline to left border of box
\newcommand{\distanceLeft}{-0.5pt}

% translation from baseline to right border of box
\newcommand{\distanceRight}{0.5pt}


% draws a rectangle around 2 points on the page
%
% optional parameter: TikZ style for the line
% 1. mandatory parameter: upper left corner, at text baseline height
% 2. mandatory parameter: lower right corner, at text baseline height
% 3. mandatory parameter: extra margin in all directions
\newcommand{\drawrect}[4][]{\begin{tikzpicture}[remember picture, overlay]
\draw[layout box, #1]
  ($(#2) +(\distanceLeft, \distanceTop) + (-#4, #4)$) rectangle
  ($(#3) + (\distanceRight, \distanceBottom) + (#4, -#4)$);
\end{tikzpicture}}

% draws a rectangle around 2 points on the page
%
% optional parameter: TikZ style for the line
% 1. mandatory parameter: left-hand-side of the first line, at text baseline height
% 2. mandatory parameter: a point on the left side of the rectangle
% 3. mandatory parameter: lower right corner, at text baseline height
% 4. mandatory parameter: extra margin in all directions
\newcommand{\drawbox}[5][]{\begin{tikzpicture}[remember picture, overlay]
\draw[layout box, #1]
  ($(#2) + (\distanceLeft, \distanceTop) + (-#5, #5)$) --
  ($(#4 |- #2) + (\distanceRight, \distanceTop) + (#5, #5)$) --
  ($(#4) + (\distanceRight, \distanceBottom) + (#5, -#5)$) --
  ($(#3 |- #4) + (\distanceLeft, \distanceBottom) + (-#5, -#5)$) --
  ($(#3 |- #2) + (\distanceLeft, \distanceBottom) + (-#5, -#5)$) --
  ($(#2) + (\distanceLeft, \distanceBottom) + (-#5, -#5)$) --
  cycle;
\end{tikzpicture}}

% draws a small \ppBox with three rows around the current point
% #1 = first.col
% #2 = left.col
% #3 = right.col
% #4 = last.col
\newcommand{\drawMiniBox}[4]{
  +(#2 * .2em, 0) --
  +(#2 * .2em, 1ex) --
  +(#1 * .2em, 1ex) --
  +(#1 * .2em, 1.5ex) --
  +(#3 * .2em, 1.5ex) --
  +(#3 * .2em, .5ex) --
  +(#4 * .2em, .5ex) --
  +(#4 * .2em, 0) --
  cycle
}

% A small box (for use inside text)
% optional argument = path options
% #1 = first.col
% #2 = left.col
% #3 = right.col
% #4 = last.col
\newcommand{\minibox}[5][]{%
  \begin{tikzpicture}[baseline=0pt]
  \path [mini layout box, #1]
    (0, 0) \drawMiniBox{#2}{#3}{#4}{#5};
  \end{tikzpicture}}

\newcommand{\miniboxA}{\minibox{-1}{0}{3}{3}}
\newcommand{\miniboxB}{\minibox{0}{0}{3}{3}}
\newcommand{\miniboxC}{\minibox{1}{0}{3}{3}}

% A small rectangle (for use inside text)
\newcommand{\minirect}{\hbox to 9pt{\drawrect[thin,scale=0.3,black]{0pt, 15pt}{26pt, -5pt}{0pt}}}


\tikzstyle{layout box}
  = [ blue
    % , semitransparent
    , semithick
    , draw
    ]

\tikzstyle{mini layout box}
  = [ black
    , very thin
    , draw
    ]

\tikzstyle{annotation}
  = [ font=\small\it
    ]

\tikzstyle{code annotation}
  = [ font=\small\it
    ]


% the font used for pretty printing
\newcommand{\selectCodeFont}{\sf\small}

% the font used for highlighting keywords
\newcommand{\selectKeywordFont}{\bfseries}

% the font used for highlighting string literals
\newcommand{\selectStringLitFont}{\tt}

% the font used for highlighting identifiers
\newcommand{\selectIdentifierFont}{\relax}

% the font used for highlighting operators
\newcommand{\selectOperatorFont}{\tt}

% the text height of token nodes
\newcommand{\tokenHeight}{2ex}

% the text depth of token nodes
\newcommand{\tokenDepth}{.5ex}

%%%%%%%%%%%%%%%%%%%%%%%%%%%
% Pretty printing with tikz
%
% tikz layer

% style for paths or scopes that do pretty printing
\tikzstyle{pretty print}
  = [ start chain=going base right
    , text height=\tokenHeight
    , text depth=\tokenDepth
    , inner sep=0
    , node distance=0em
    ]

% helper style for nodes that start a new line
% (automatically applied by style 'new line' below)
\tikzstyle{new line/helper}
  = [ on chain
    , anchor=north west,
      at={($(#1.west |- \tikzchainprevious.base) + (0, -1.1\baselineskip)$)}
    ]

\tikzstyle{new line tight/helper}
  = [ on chain
    , anchor=north west,
      at={($(#1.west |- \tikzchainprevious.base) + (0, -0.9\baselineskip)$)}
    ]


% style for nodes that start a new line
\tikzstyle{new line}[\tikzchainprevious]
  = [ on chain=placed {new line/helper=#1}
    ]

% style for nodes that start a new line
\tikzstyle{new line tight}[\tikzchainprevious]
  = [ on chain=placed {new line tight/helper=#1}
    ]

% helper style for nodes that tab forward to a labeled position
% (automatically applied by style 'tab forward' below)
\tikzstyle{tab forward/helper}
  = [ on chain
    , anchor=north west,
      at={(#1 |- \tikzchainprevious.base)}
    ]

% style for nodes that tab forward to a labeled position
\tikzstyle{tab forward}
  = [ on chain=placed {tab forward/helper=#1}
    ]

% style for nodes that indent
\tikzstyle{indent}[1em]
  = [ on chain=placed {base right=#1 of \tikzchainprevious}
    ]

% style for nodes that contain an identifier token
\tikzstyle{identifier}
  = [ on chain
    , font=\selectIdentifierFont
    ]

% style for nodes that contain an operator token
\tikzstyle{operator}
  = [ on chain
    , font=\selectOperatorFont
    ]

% style for nodes that contain a keyword token
\tikzstyle{keyword}
  = [ on chain
    , font=\selectKeywordFont
    , text=keyword
    ]

% style for nodes that contain a string literal
\tikzstyle{stringlit}
  = [ on chain
    , font=\selectStringLitFont
    , text=blue
    ]

%%%%%%%%%%%%%%%%%%%%%%%%%%%
% Pretty printing with tikz
%
% \ppBox layer

\newcommand{\Empty}{}
\newcommand{\ignore}[1]{\relax}

\makeatletter
\newcommand{\ppBox}[2][]{{
  % insert keyword token
  \newcommand{\KW}[1]{
    \node[keyword] {##1};
    \HandleToken{\tikzchaincurrent}}

  % insert identifier token
  \newcommand{\ID}[1]{
    \node[identifier] {##1};
    \HandleToken{\tikzchaincurrent}}

  % insert operator token
  \newcommand{\OP}[1]{
    \node[operator] {##1};
    \HandleToken{\tikzchaincurrent}}

  % insert whitespace token
  \newcommand{\SP}{
    \node[on chain] { };
    \HandleInsensitiveToken{\tikzchaincurrent}}

  % label a position (with a tikz node name)
  \newcommand{\LB}[1]{
    \coordinate[on chain] (##1);}

  % tab forward to a labeled position
  \newcommand{\TB}[1]{
    \coordinate[tab forward=##1];
  }

  % insert a string literal
  \newcommand{\STR}[1]{
    \node[stringlit] {##1};
    \HandleToken{\tikzchaincurrent}}

  % insert a line break
  \renewcommand{\\}{
    \coordinate[new line=\NewlineNode];
    \let\HandleToken=\HandleTokenSubsequentLine}

  \newcommand{\newlineTight}{
    \coordinate[new line tight=\NewlineNode];
    \let\HandleToken=\HandleTokenSubsequentLine}

  % the name of the first token
  \let\FirstToken=\Empty
  \newcommand{\AdjustFirst}[1]{
    \ifx\FirstToken\Empty
    \edef\FirstToken{##1}
    \fi
  }

  % the name of the last token
  \let\LastToken=\Empty
  \newcommand{\AdjustLast}[1]{
    \edef\LastToken{##1}
  }

  % the name of the left-most token
  % (except for tokens in the first line)
  \let\LeftToken=\Empty
  \newcommand{\AdjustLeft}[1]{
    \ifx\LeftToken\Empty
      \edef\LeftToken{##1}
    \else
      \pgf@process{\pgfpointanchor{##1}{west}}
      \setlength{\pgf@xa}{\pgf@x}
      \pgf@process{\pgfpointanchor{\LeftToken}{west}}
      \setlength{\pgf@xb}{\pgf@x}
      \ifdim\pgf@xa<\pgf@xb
        \edef\LeftToken{##1}
      \fi
    \fi
  }

  % the name of the rightmost token
  \let\RightToken=\Empty
  \newcommand{\AdjustRight}[1]{
    \ifx\RightToken\Empty
      \edef\RightToken{##1}
    \else
      \pgf@process{\pgfpointanchor{##1}{east}}
      \setlength{\pgf@xa}{\pgf@x}
      \pgf@process{\pgfpointanchor{\RightToken}{east}}
      \setlength{\pgf@xb}{\pgf@x}
      \ifdim\pgf@xa>\pgf@xb
        \edef\RightToken{##1}
      \fi
    \fi
  }

  % this is called for all layout-sensitive tokens in the first
  % line of a pretty printing group
  \newcommand{\HandleTokenFirstLine}[1]{
    \AdjustFirst{##1}
    \AdjustRight{##1}
    \AdjustLast{##1}
  }

  % this is called for all layout-sensitive tokens in subsequent
  % lines of a pretty printing group
  \newcommand{\HandleTokenSubsequentLine}[1]{
    \AdjustLeft{##1}
    \AdjustRight{##1}
    \AdjustLast{##1}}

  % this is called for layout-insensitive tokens in a pretty
  % printing group (such as whitespace, comments, and code that
  % uses explicit layout)
  \newcommand{\HandleInsensitiveToken}[1]{
  }

  % current token handler. will be set to one of:
  %  \HandleTokenFirstLine
  %  \HandleTokenSubsequentLine
  %  \HandleInsensitiveToken
  \newcommand{\HandleToken}[1]{}

  \newcommand{\ppSubBox}[2][]{
    \coordinate[on chain];
    {
      \let\NewlineNode\tikzchaincurrent

      % reset the token registers
      \let\FirstToken=\Empty
      \let\LastToken=\Empty
      \let\LeftToken=\Empty
      \let\RightToken=\Empty

      % start in first line
      \let\HandleToken=\HandleTokenFirstLine

      % typeset content
      ##2

      % draw boundary
      \ifx\LeftToken\Empty
        \path [##1]
          (\FirstToken.north west) rectangle
          (\LastToken.south east);
      \else
        \path [##1]
          (\RightToken.east |- \FirstToken.north) --
          (\FirstToken.north west) --
          (\FirstToken.south west) --
          (\LeftToken.west |- \FirstToken.south) --
          (\LeftToken.west |- \LastToken.south) --
          (\LastToken.south east) --
          (\LastToken.north east) --
          (\RightToken.east |- \LastToken.north) --
          cycle;
      \fi

      % remember the token registers outside the group
      \global\let\InnerFirstToken=\FirstToken
      \global\let\InnerLastToken=\LastToken
      \global\let\InnerLeftToken=\LeftToken
      \global\let\InnerRightToken=\RightToken

      % remember the state outside the group
      \global\let\InnerHandleToken=\HandleToken
    }
    % handle the significant inner tokens
    % and significant inner linebreaks
    \HandleToken{\InnerFirstToken}
    \ifx\HandleToken\HandleTokenFirstLine
    \let\HandleToken=\InnerHandleToken
    \fi
    \ifx\InnerLeftToken\Empty
    \else
    \HandleToken{\InnerLeftToken}
    \fi
    \HandleToken{\InnerRightToken}
    \HandleToken{\InnerLastToken}
  }

  % indent a ppBox
  \newcommand{\IN}[1]{\coordinate[indent]; ##1}
  \newcommand{\DE}[1]{\coordinate[indent=-1em]; ##1}

  \let\ppBox=\ppSubBox
  \ppBox[#1]{#2}}}
\makeatother




%%% Local Variables: 
%%% mode: latex
%%% TeX-master: "../thesis"
%%% End: 


\begin{document}
\section{How do we check entailments?}
\begin{enumerate}
  \item Create set of ``axioms'' based on the program and the entailment to check
  \begin{enumerate}
    \item Datatype Declarations: Model everything as an enumeration type (Variables, Classes, Fields, Paths)
    \begin{itemize}
      \item Extract field and class names from the program (constructor declarations).
      \item Extract variable names from the entailment to be checked.
      \item Enumerate all paths that do not exceed the depth limit based on the extracted information.
    \end{itemize}
    \item Function Declarations
    \begin{itemize}
      \item Declare boolean predicates for the constraints: path-equivalence, instance-of, instantiated-by
      \item For substitution we previously had \[\Path \times \Variable \times \Path \rightarrow \Path\] as the substitution function signature.
            This is no longer possible. The enumeration of paths with a depth limit would result in a partial function,
            as we would no longer be able to compute result paths for substitutions that would exceed the depth limit.
            Since the SMT solver only supports total functions, we form a relation
            \[\Path \times \Variable \times \Path \times \Path \rightarrow \TBool\]
            that only is true for each $\Path \times \Variable \times \Path$ triple where the partial function would be defined at.
    \end{itemize}
    \item Calculus rules
    \begin{itemize}
      \item Static Rules: C-Refl, C-Class, C-Subst
      \item Dynamic Rules: C-Prog\\
            The creation of C-Prog rules is handled context-sensitive to the program.
            We transforme each constraint entailment declaration
            \[ \progEnt{x}{c_1, c_2, ..., c_n}{\instOf{x}{cls}} \]
            from the program into a C-Prog rule template of the form
            \[ p:\Path \Rightarrow \subst{c_1}{x}{p} \land \subst{c_2}{x}{p} \land ... \land \subst{c_n}{x}{p} \rightarrow \instOf{p}{cls} \]
            and instantiate the template with all available paths.
            Instantiations where one of the paths post substitution exceed the depth limit are discarded.
    \end{itemize}
    \item Assert entailment to be checked: $c_1,...,c_n \vdash c \Rightarrow \neg (c_1 \land ... \land c_n → c)$
  \end{enumerate}
  \item Obtain Solution: Does the entailment contradict the rules?
  \begin{itemize}
    \item If unsat: there is a contradiction between the negated entailment and the calculus rules, thus the entailment holds.
    \item If sat: there is no contradiction between the negated entailment and the calculus rules, thus the entailment does not hold.
  \end{itemize}
\end{enumerate}

\section{Simplified example encoding}
To encode: $\entails{\pathEq{y}{x}}{\pathEq{x}{y}}$
\begin{align}
  &\texttt{\Variable} := \{\mathtt{x}, \mathtt{y}\}\\
  % def subst
  &\subst{p}{v}{q} = s :=\\
  &\quad (p=\mathtt{x} \land v=\mathtt{x} \land q=\mathtt{x} \land s=\mathtt{x})~\lor\\
  &\quad (p=\mathtt{x} \land v=\mathtt{x} \land q=\mathtt{y} \land s=\mathtt{y})~\lor\\
  &\quad (p=\mathtt{x} \land v=\mathtt{y} \land q=\mathtt{x} \land s=\mathtt{x})~\lor\\
  &\quad (p=\mathtt{x} \land v=\mathtt{y} \land q=\mathtt{y} \land s=\mathtt{x})~\lor\\
  &\quad (p=\mathtt{y} \land v=\mathtt{x} \land q=\mathtt{x} \land s=\mathtt{y})~\lor\\
  &\quad (p=\mathtt{y} \land v=\mathtt{x} \land q=\mathtt{y} \land s=\mathtt{y})~\lor\\
  &\quad (p=\mathtt{y} \land v=\mathtt{y} \land q=\mathtt{x} \land s=\mathtt{x})~\lor\\
  &\quad (p=\mathtt{y} \land v=\mathtt{y} \land q=\mathtt{y} \land s=\mathtt{y})\\
  % C-Refl
  &\forall p.~\pathEq{p}{p}\\
  % C-Subst: pathEq
  &\forall p, q, v, r, s, a, b, c, d.\\
  &\quad \pathEq{s}{r} \land \subst{p}{v}{r}=a \land \subst{q}{v}{r}=b~\land\\
  &\quad \pathEq{a}{b} \land
         \subst{p}{v}{s}=c \land \subst{q}{v}{s}=d\\
  &\qquad \rightarrow \pathEq{c}{d}\\
  % Entailment
  &\neg (\pathEq{\mathtt{y}}{\mathtt{x}} \rightarrow \pathEq{\mathtt{x}}{\mathtt{y}})
\end{align}
\newpage

\section{Complete example encoding}
To encode: $\entails{\instBy{x}{\Succ}, \instBy{x.p}{\Zero}, \pathEq{x}{y}}{\instOf{y}{\Nat}}$
with $\mathit{depth\!\!-\!\!limit}=1$
\begin{align}
  &\texttt{\Variable} := \{\mathtt{x}, \mathtt{y}\}\\
  &\texttt{\Class} := \{\Zero,\Succ,\Nat\}\\
  &\texttt{\Path} := \{\mathtt{x}, \mathtt{x}.\mathtt{p}, \mathtt{y}, \mathtt{y}.\mathtt{p}\}\\
  % def subst
  &\subst{p}{v}{q} = s :=\\
  &\quad (p=\mathtt{x} \land v=\mathtt{x} \land q=\mathtt{x} \land s=\mathtt{x})~\lor\\
  &\quad (p=\mathtt{x} \land v=\mathtt{x} \land q=\mathtt{x.p} \land s=\mathtt{x.p})~\lor\\
  &\quad (p=\mathtt{x} \land v=\mathtt{x} \land q=\mathtt{y} \land s=\mathtt{y})~\lor\\
  &\quad (p=\mathtt{x} \land v=\mathtt{x} \land q=\mathtt{y.p} \land s=\mathtt{y.p})~\lor\\
  %&\quad (p=\mathtt{x} \land v=\mathtt{y} \land q=\mathtt{x} \land s=\mathtt{x})~\lor\\
  %&\quad (p=\mathtt{x} \land v=\mathtt{y} \land q=\mathtt{x.p} \land s=\mathtt{x})~\lor\\
  %&\quad (p=\mathtt{x} \land v=\mathtt{y} \land q=\mathtt{y} \land s=\mathtt{x})~\lor\\
  %&\quad (p=\mathtt{x} \land v=\mathtt{y} \land q=\mathtt{y.p} \land s=\mathtt{x})~\lor\\
  &\quad (p=\mathtt{x.p} \land v=\mathtt{x} \land q=\mathtt{x} \land s=\mathtt{x.p})~\lor\\
  &\quad (p=\mathtt{x.p} \land v=\mathtt{x} \land q=\mathtt{y} \land s=\mathtt{y.p})~\lor\\
  &\quad~...\\
  % C-Refl
  &\forall p.~\pathEq{p}{p}\\
  % C-Class
  &\forall p, c.~\instBy{p}{c} \rightarrow \instOf{p}{c}\\
  % C-Subst: pathEq
  &\forall p, q, v, r, s, a, b, c, d.\\
  &\quad \pathEq{s}{r} \land \subst{p}{v}{r}=a \land \subst{q}{v}{r}=b~\land\\
  &\quad \pathEq{a}{b} \land
         \subst{p}{v}{s}=c \land \subst{q}{v}{s}=d\\
  &\qquad \rightarrow \pathEq{c}{d}\\
  % C-Subst: instOf
  &\forall p, c, v, r, s, a, b.\\
  &\quad \pathEq{s}{r} \land \subst{p}{v}{r}=a~\land\\
  &\quad \instOf{a}{c} \land
         \subst{p}{v}{s}=b\\
  &\qquad \rightarrow \instOf{b}{c}\\
  % C-Subst: instBy
  &\forall p, c, v, r, s, a, b.\\
  &\quad \pathEq{s}{r} \land \subst{p}{v}{r}=a~\land\\
  &\quad \instBy{a}{c} \land
         \subst{p}{v}{s}=b\\
  &\qquad \rightarrow \instBy{b}{c}\\
  % C-Prog
  &\instOf{\mathtt{x}}{\Zero} \rightarrow \instOf{\mathtt{x}}{\Nat}\\
  &\instOf{\mathtt{x.p}}{\Zero} \rightarrow \instOf{\mathtt{x.p}}{\Nat}\\
  &\instOf{\mathtt{x}}{\Succ} \land \instOf{\mathtt{x.p}}{\Nat} \rightarrow \instOf{\mathtt{x}}{\Nat}\\
  % Entailment
  &\neg (\instBy{\mathtt{x}}{\Succ} \land \instBy{\mathtt{x.p}}{\Zero} \land \pathEq{\mathtt{x}}{\mathtt{y}} \rightarrow \instOf{\mathtt{y}}{\Nat})
\end{align}
\newpage

\section{Is using a depth limit reasonable?}
\begin{tabular}{c c c}
\parbox{0.33\textwidth}{
% C-Refl
\begin{prooftree}
\AxiomC{}
\RightLabel{(C-Refl)}
\UnaryInfC{\entails{\epsilon}{\pathEq{p}{p}}}
\end{prooftree}
}
&
\parbox{0.33\textwidth}{
% C-Ident
\begin{prooftree}
\AxiomC{}
\RightLabel{(C-Ident)}
\UnaryInfC{\entails{a}{a}}
\end{prooftree}
}
&
\parbox{0.33\textwidth}{
% C-Class
\begin{prooftree}
\AxiomC{\entails{\overline{a}}{\instantiatedBy{p}{C}}}
\RightLabel{(C-Class)}
\UnaryInfC{\entails{\overline{a}}{\instanceOf{p}{C}}}
\end{prooftree}
}
\end{tabular}
\begin{tabular}{c c}
\parbox{0.5\textwidth}{
% C-Subst
\begin{prooftree}
\AxiomC{\entails{\overline{a}}{a_{\sub{x}{p}}}}
\AxiomC{\entails{\overline{a}}{\pathEq{p'}{p}}}
\RightLabel{(C-Subst)}
\BinaryInfC{\entails{\overline{a}}{a_{\sub{x}{p'}}}}
\end{prooftree}
}
&
\parbox{0.5\textwidth}{
% C-Prog
\begin{prooftree}
\AxiomC{$(\progEnt{x}{\overline{a}}{a}) \in P$}
\AxiomC{\entails{\overline{b}}{\overline{a}_{\sub{x}{p}}}}
\RightLabel{(C-Prog)}
\BinaryInfC{\entails{\overline{b}}{a_{\sub{x}{p}}}}
\end{prooftree}
}
\end{tabular}

\subsection{Where can we use substitution?}
% C-Subst
\begin{prooftree}
\AxiomC{\entails{\overline{a}}{a_{\sub{x}{p}}}}
\AxiomC{\entails{\overline{a}}{\pathEq{p'}{p}}}
\RightLabel{(C-Subst)}
\BinaryInfC{\entails{\overline{a}}{a_{\sub{x}{p'}}}}
\end{prooftree}

We can substitute an equivalent path in the conclusion of an entailment.
This substitution might exceed the depth limit,
but we might only alter the conclusion and not the context of the entailment.
Also the prooftree-lowest conclusion is already the result of a substitution.

\subsection{How can we close branches?}
\begin{tabular}{c c}
\parbox{0.5\textwidth}{
% C-Refl
\begin{prooftree}
\AxiomC{}
\RightLabel{(C-Refl)}
\UnaryInfC{\entails{\epsilon}{\pathEq{p}{p}}}
\end{prooftree}
}
&
\parbox{0.5\textwidth}{
% C-Ident
\begin{prooftree}
\AxiomC{}
\RightLabel{(C-Ident)}
\UnaryInfC{\entails{a}{a}}
\end{prooftree}
}
\end{tabular}

\begin{itemize}
\item[\textbf{C-Refl}] Reflexive paths, independent of context. %Could potentially exceed the depth limit.
\item[\textbf{C-Ident}] Constraints that are already in the context.
\end{itemize}

\subsection{Observation}
\begin{prooftree}
% left branch
\AxiomC{...}
\UnaryInfC{\entails{\instOf{x}{\Zero}, \pathEq{x}{y}}{ \instOf{x}{\Zero} }}
\UnaryInfC{\entails{\instOf{x}{\Zero}, \pathEq{x}{y}}{ \subst{\instOf{y}{\Zero}}{\color{orange} y}{\color{blue} x} }}
% right branch
\AxiomC{...}
\UnaryInfC{\entails{\instOf{x}{\Zero}, \pathEq{x}{y}}{\pathEq{\color{blue} x}{y}}}
% op
\BinaryInfC{\entails{\instOf{x}{\Zero}, \pathEq{x}{y}}{\instOf{\color{orange} y}{\Zero}}}
\end{prooftree}

To be able to close a branch in practice, we wan't to come back to a form that is already in the context.
We wan't to substitute with path equivalences that the context already provides.
The variable we want to substitute is the base of the path in the conclusion constraint.

\subsection{Whats with program entailment instantiations?}
% C-Prog
\begin{prooftree}
\AxiomC{$(\progEnt{x}{\overline{a}}{a}) \in P$}
\AxiomC{\entails{\overline{b}}{\overline{a}_{\sub{x}{p}}}}
\RightLabel{(C-Prog)}
\BinaryInfC{\entails{\overline{b}}{a_{\sub{x}{p}}}}
\end{prooftree}

Usages of rule C-Prog replace the conclusion of an entailment with new proof goals.
It does so by instantiating a declaration from the program.

\subsection{Hypothesis}
The depth of paths used in proofs in the sequent calculus only grows within a certain bound.

\end{document}
