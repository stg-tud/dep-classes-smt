%% macros
%%%%%%%%%
\newcommand{\wip}{\todo[inline]{work in progress}}
%\newcommand{\nat}[1]{\ensuremath{\lceil\text{#1}\rceil}}
\newcommand{\gprod}[1]{\ensuremath{\langle\text{#1}\rangle}}
\newcommand{\TNum}{\ensuremath{\mathcal{N}}}
\newcommand{\TBool}{\ensuremath{\mathcal{B}}}
\newcommand{\TInt}{\ensuremath{\mathcal{Z}}}

% DCC
% DCC constraints
\newcommand{\pathEq}[2]{\ensuremath{\mathit{#1} \equiv \mathit{#2}}}
\newcommand{\instanceOf}[2]{\ensuremath{\mathit{#1} :: \mathit{#2}}}
\newcommand{\instOf}[2]{\instanceOf{#1}{#2}}
\newcommand{\instantiatedBy}[2]{\ensuremath{\mathit{#1}.\textbf{cls} \equiv \mathit{#2}}}
\newcommand{\instBy}[2]{\instantiatedBy{#1}{#2}}
\newcommand{\entails}[2]{\ensuremath{#1 \vdash #2}}
\newcommand{\entailsA}[2]{\ensuremath{#1 \vdash_{\!\!\!A} #2}}
\newcommand*{\ldblbrace}{\{\mskip-6.9mu\{} % left double brace
\newcommand*{\rdblbrace}{\}\mskip-6.9mu\}} % right double brace
\newcommand{\sub}[2]{\ensuremath{\ldblbrace #1 \mapsto #2 \rdblbrace}}

% DCC declarations
\newcommand{\constructorDeclaration}[3]{\ensuremath{\mathit{#1}(\mathit{#2}.\ \mathit{#3})}}
\newcommand{\constr}[3]{\constructorDeclaration{#1}{#2}{#3}}
\newcommand{\programEntailment}[3]{\ensuremath{ \forall \mathit{#1}.\ \mathit{#2} \Rightarrow \mathit{#3}}}
\newcommand{\progEnt}[3]{\programEntailment{#1}{#2}{#3}}
\newcommand{\abstractMethodDeclaration}[4]{\ensuremath{\mathit{#1}(\mathit{#2}.\ \mathit{#3}): \mathit{#4}}}
\newcommand{\mDecl}[4]{\abstractMethodDeclaration{#1}{#2}{#3}{#4}}
\newcommand{\methodImplementation}[5]{\ensuremath{\mathit{#1}(\mathit{#2}.\ \mathit{#3}): \mathit{#4} := \mathit{#5}}}
\newcommand{\mImpl}[5]{\methodImplementation{#1}{#2}{#3}{#4}{#5}}

% DCC expressions
\newcommand*{\new}{\ensuremath{\textbf{new }}}
\newcommand{\newInstance}[3]{\ensuremath{\new \mathit{#1}(\mathit{#2} \equiv \mathit{#3})}}
\newcommand{\newInstNoArgs}[1]{\ensuremath{\new \mathit{#1}()}}
\newcommand{\newInst}[3]{\newInstance{#1}{#2}{#3}}
\newcommand{\objConstr}[2]{\ensuremath{\new \mathit{#1}(\mathit{#2})}}

% DCC type
\newcommand{\type}[2]{\ensuremath{[#1.\ \mathit{#2}]}}

% DCC misc
\newcommand{\obj}[3]{\ensuremath{\pair{#1}{#2 \equiv #3}}}
\newcommand*{\stdobj}{\ensuremath{\obj{C}{\overline{f}}{\overline{x}}}}
\newcommand{\heap}[2]{\ensuremath{#1 \mapsto #2}}
\newcommand*{\stdheap}{\ensuremath{\heap{\overline{x}}{\overline{o}}}}
\newcommand{\pair}[2]{\ensuremath{\langle #1;#2 \rangle}}
\newcommand{\eval}[4]{\ensuremath{\pair{#1}{#2} \rightarrow \pair{#3}{#4}}}
\newcommand{\typeass}[3]{\ensuremath{#1 \vdash #2 : #3}}
\newcommand{\wf}[1]{\ensuremath{\mathsf{wf }#1}}
\newcommand{\FV}[1]{\ensuremath{FV(#1)}}
\newcommand{\FVeq}[2]{\ensuremath{\FV{#1} = \{#2\}}}

% FO
\mathchardef\mhyphen="2D
% \newcommand\nuniq{\mathit{non\mhyphen unique}} example command
\newcommand*{\pleft}{\ensuremath{\mathit{p\mhyphen left}}}
\newcommand*{\pright}{\ensuremath{\mathit{p\mhyphen right}}}
\newcommand*{\inprog}{\ensuremath{\mathit{in\mhyphen program}}}
\newcommand*{\substp}{\ensuremath{\mathit{subst\mhyphen path}}}
\newcommand*{\substc}{\ensuremath{\mathit{subst\mhyphen constraint}}}
\newcommand*{\substcs}{\ensuremath{\mathit{subst\mhyphen constraints}}}
\newcommand{\substpath}[3]{\ensuremath{\substp(\mathit{#1}, \mathit{#2}, \mathit{#3})}}
\newcommand{\substconstr}[3]{\ensuremath{\substc(\mathit{#1}, \mathit{#2}, \mathit{#3})}}
\newcommand{\substconstrs}[3]{\ensuremath{\substcs(\mathit{#1}, \mathit{#2}, \mathit{#3})}}
\newcommand{\subst}[3]{\ensuremath{\mathit{#1}_{\mathit{#2} \mapsto \mathit{#3}}}}
\newcommand*{\genp}{\ensuremath{\mathit{gen\mhyphen path}}}
\newcommand*{\genc}{\ensuremath{\mathit{gen\mhyphen constraint}}}
\newcommand{\gen}[3]{\ensuremath{\subst{#1}{#2}{#3}}}
\newcommand{\is}[1]{\ensuremath{\mathit{is\mhyphen #1}}}
\newcommand{\ite}[3]{\ensuremath{#1\ ?\ #2\ :\ #3}}
\newcommand*{\If}{\textbf{if }}
\newcommand*{\Then}{\textbf{then }}
\newcommand*{\Else}{\textbf{else }}
\newcommand*{\Let}{\textbf{let }}
\newcommand*{\In}{\textbf{in }}
\newcommand*{\match}{\textbf{ match}}
\newcommand{\case}[2]{\ensuremath{\mathit{#1} \Rightarrow \mathit{#2}}}
\newcommand*{\nil}{\ensuremath{\mathit{nil}}}
\newcommand*{\Variable}{\ensuremath{\mathtt{Variable}}}
\newcommand*{\Field}{\ensuremath{\mathtt{Field}}}
\newcommand*{\Class}{\ensuremath{\mathtt{Class}}}
\newcommand*{\Path}{\ensuremath{\mathtt{Path}}}
\newcommand*{\Constr}{\ensuremath{\mathit{Constraint}}}
\newcommand*{\Constrs}{\ensuremath{\mathit{List[\Constr]}}}
\newcommand*{\Constrss}{\ensuremath{\mathit{List[\Constrs]}}}
\newcommand*{\String}{\ensuremath{\mathit{String}}}
\newcommand{\sortedVar}[2]{\ensuremath{#1\!:\!#2}}
\newcommand\conc{\ensuremath{\mathbin{+\mkern-10mu+}}}
\newcommand{\str}[1]{\texttt{"#1"}}
\newcommand*{\hole}{\ensuremath{\_}}
\newcommand*{\helper}{\ensuremath{\mathit{helper}}}

% Natural Numbers Program
\newcommand*{\Nat}{\ensuremath{\mathtt{Nat}}}
\newcommand*{\Zero}{\ensuremath{\mathtt{Zero}}}
\newcommand*{\Succ}{\ensuremath{\mathtt{Succ}}}

% SMTLib format stuff
% SMT-Lib language definition for listings
\lstdefinelanguage{smtlib}{
  alsoletter={-,!,:,=,>}, % '-' is usable in keywords
  morekeywords={assert,!,:named,check-sat},
  morekeywords={declare-datatype,declare-fun,define-fun-rec,define-fun},
  morekeywords={ite,let},
  morekeywords={true,false,and,or,forall,exists},
  sensitive=false, % keywords are not case-sensitive
  morecomment=[l]{//}, % l is for line comment
  morestring=[b]", % defines that strings are enclosed in double quotes
  basicstyle=\small % or \small, \tiny, \footnotesize, etc: sets text size
}
\newcommand{\smtlib}[1]{\lstinline[language=smtlib,basicstyle=\normalsize]{#1}}
\newcommand{\scala}[1]{\lstinline[language=scala,basicstyle=\normalsize]{#1}}

%% misc
%%%%%%%%%
\DeclareMathOperator{\pto}{\rightharpoonup}     % partial function arrow
\DeclareMathOperator{\powerset}{\mathcal{P}}    % powerset
\DeclareMathOperator{\proj}{\upharpoonright}    % projection
\DeclareMathOperator{\fst}{\pi_1}               % A x B → A
\DeclareMathOperator{\snd}{\pi_2}               % A x B → B
\newcommand{\ovl}[1]{\ensuremath{\overline{#1}}}
\newcommand{\mIt}[1]{\ensuremath{\mathit{#1}}}

%%% Local Variables:
%%% mode: latex
%%% TeX-master: "../thesis"
%%% End:
