\documentclass[a4paper]{article}

%% packages
%%%%%%%%%%%
\usepackage[utf8x]{inputenc}
\usepackage[USenglish]{babel}
\usepackage{amsbsy,amscd,amsfonts,amssymb,amstext,amsmath,amsthm,latexsym}
\usepackage{mathpartir}
\usepackage{stmaryrd}
\usepackage{dot2texi}
\usepackage{url}
\usepackage{hyperref}
\usepackage[nottoc]{tocbibind}
\usepackage{pdfpages}

\usepackage{syntax} % for bnf grammar
\usepackage{bussproofs} % for type rules
\usepackage{todonotes}

\usepackage{float} % for figures
\floatstyle{boxed}
\restylefloat{figure}

\usepackage{listings}
\usepackage{xcolor}
%\usepackage{tikz}
\usetikzlibrary{positioning,chains,shapes.arrows,shapes.geometric,fit,calc,arrows,decorations.pathmorphing}

%\usepackage{subfig}
\usepackage{caption}
\usepackage{subcaption}
\usepackage{adjustbox}
\usepackage{cleveref}

\newtheorem{theorem}{Theorem}
\newtheorem{lemma}{Lemma}

%% macros
%%%%%%%%%
\newcommand{\loremipsum}{\todo[inline]{Lorem Ipsum}}
%\newcommand{\nat}[1]{\ensuremath{\lceil\text{#1}\rceil}}
\newcommand{\gprod}[1]{\ensuremath{\langle\text{#1}\rangle}}
\newcommand{\TNum}{\ensuremath{\mathcal{N}}}
\newcommand{\TBool}{\ensuremath{\mathcal{B}}}
\newcommand{\TInt}{\ensuremath{\mathcal{N}}}

\theoremstyle{definition}
\newtheorem{example}{Example}[chapter]
\newtheorem{definition}[example]{Definition}
%\newtheorem{definition}{Definition}[chapter]

% DCC
% DCC constraints
\newcommand{\pathEq}[2]{\ensuremath{#1 \equiv #2}}
\newcommand{\instanceOf}[2]{\ensuremath{#1 :: #2}}
\newcommand{\instOf}[2]{\instanceOf{#1}{#2}}
\newcommand{\instantiatedBy}[2]{\ensuremath{#1.\textbf{cls} \equiv #2}}
\newcommand{\instBy}[2]{\instantiatedBy{#1}{#2}}
\newcommand{\entails}[2]{\ensuremath{#1 \vdash #2}}
\newcommand*{\ldblbrace}{\{\mskip-6.9mu\{} % left double brace
\newcommand*{\rdblbrace}{\}\mskip-6.9mu\}} % right double brace
\newcommand{\sub}[2]{\ensuremath{\ldblbrace #1 \mapsto #2 \rdblbrace}}

% DCC declarations
\newcommand{\constructorDeclaration}[3]{\ensuremath{#1(#2.\ #3)}}
\newcommand{\constr}[3]{\constructorDeclaration{#1}{#2}{#3}}
\newcommand{\programEntailment}[3]{\ensuremath{ \forall #1.\ #2 \Rightarrow #3}}
\newcommand{\progEnt}[3]{\programEntailment{#1}{#2}{#3}}
\newcommand{\abstractMethodDeclaration}[4]{\ensuremath{#1(#2.\ #3): #4}}
\newcommand{\mDecl}[4]{\abstractMethodDeclaration{#1}{#2}{#3}{#4}}
\newcommand{\methodImplementation}[5]{\ensuremath{#1(#2.\ #3): #4 := #5}}
\newcommand{\mImpl}[5]{\methodImplementation{#1}{#2}{#3}{#4}{#5}}

% DCC expressions
\newcommand{\newInstance}[3]{\ensuremath{\textbf{new } #1(#2 \equiv #3)}}
\newcommand{\newInst}[3]{\newInstance{#1}{#2}{#3}}

% DCC misc
\newcommand{\obj}[3]{\ensuremath{\langle #1; #2 \equiv #3 \rangle}}
\newcommand*{\stdobj}{\ensuremath{\obj{C}{\overline{f}}{\overline{x}}}}
\newcommand{\heap}[2]{\ensuremath{#1 \mapsto #2}}
\newcommand*{\stdheap}{\ensuremath{\heap{\overline{x}}{\overline{o}}}}
\newcommand{\pair}[2]{\ensuremath{\langle #1;#2 \rangle}}
\newcommand{\eval}[4]{\ensuremath{\pair{#1}{#2} \rightarrow \pair{#3}{#4}}}
\newcommand{\typeass}[3]{\ensuremath{#1 \vdash #2 : #3}}
\newcommand{\wf}[1]{\ensuremath{\text{wf }#1}}
\newcommand{\FV}[1]{\ensuremath{FV(#1)}}
\newcommand{\FVeq}[2]{\ensuremath{\FV{#1} = \{#2\}}}

%% misc
%%%%%%%%%
\DeclareMathOperator{\pto}{\rightharpoonup}     % partial function arrow
\DeclareMathOperator{\powerset}{\mathcal{P}}    % powerset
\DeclareMathOperator{\proj}{\upharpoonright}    % projection
\DeclareMathOperator{\fst}{\pi_1}               % A x B → A
\DeclareMathOperator{\snd}{\pi_2}               % A x B → B
\newcommand{\ovl}[1]{\ensuremath{\overline{#1}}}

%%% Local Variables: 
%%% mode: latex
%%% TeX-master: "../thesis"
%%% End: 

%
\usetikzlibrary{positioning,chains,fit,calc,arrows,decorations.pathmorphing}
\usetikzlibrary{shapes.arrows,shapes.geometric,shapes.symbols}


%%%%%%%
%% tikz stuff
%%%%%%%


\tikzstyle{invisible} = []

\tikzstyle{model}
  = [ shape=rectangle
    , draw
    ]
\tikzstyle{transformation}
  = [ model
    , shape=single arrow
    , draw
    ]

\tikzstyle{meta} = [ double ]
\tikzstyle{generated} = [ dashed ]

\tikzstyle{mopdependency}
  = [ -stealth'
    , semithick
    ]
\tikzstyle{instance}
  = [ mopdependency
    , -open triangle 60
    ]

% Draws the outline of a bend arrow around a rectangle.
% 1. Parameter: name of the rectangle that contains the text
% 2. Parameter: head extend (use 0.25cm to fit with default tikz style)
% 3. Parameter: arc radius (again, 0.25cm seems to work well)
%
% This produces a path, so use it as follows:
%
% \draw[options here] \bendArrow{name of rectangle}{.25cm}{.25cm};
\newcommand{\bendArrow}[3]{
  let \p{content size} = ($(#1.north east) - (#1.south west)$) in
  let \n{tip length 1} = {veclen(0.5 * \y{content size}, 0.5 * \y{content size})} in
  let \n{tip length 2} = {veclen(#2,#2)} in

  (#1.north east)
  -- ++ (315 : \n{tip length 1})
  -- ++ (225 : \n{tip length 1} + \n{tip length 2})
  -- ++ (0, #2)
  -- (#1.south west)
  arc (270 : 180 : \y{content size} + #3)
  -- ++ (\y{content size}, 0)
  arc (180 : 270 : #3)
  (#1.north east)
  -- ++ (135 : \n{tip length 2})
  -- ++ (0, -#2)
  -- (#1.north west)}


%%%%%%%%%%%%%%%%%
%% diagram styles
%%%%%%%%%%%%%%%%%

\tikzstyle{document}
  = [ shape=rectangle
    , draw
    , minimum height=1.5em
    , text width=5em
    , text centered
    ]

\tikzstyle{component}
  = [ font=\scriptsize\it
    ]

\tikzstyle{code}
  = [ shape=rectangle
    , draw
    ]

\tikzstyle{process}
  = [ shape=single arrow
    , single arrow head extend=.75em
    , single arrow head indent=.25em
    , minimum width=3em
    , draw
    ]

\tikzstyle{point}
  = [ coordinate
    , minimum width=1em
    ]

\tikzstyle{flow diagram}
  = [ start chain
    , node distance=1em
    , every node/.style={on chain}
    ]

\tikzstyle{ast}
  = [ node distance=1em and 0.38em
    , every node/.style=ast node
    ]

\tikzstyle{ast node}
  = [ shape=circle
    , minimum size=0.5em
    , inner sep=0
    , fill
    ]

\tikzstyle{dependency}
  = [ dashed
    , -stealth'
    ]

\tikzstyle{red}
  = [ shape=rectangle
    , color=red
    ]

\tikzstyle{blue}
  = [ shape=circle
    , color=blue
    ]

\tikzstyle{green}
  = [ shape=diamond
    , color=green!80!black
    , minimum size=0.7em
    ]

\tikzstyle{note}
  = [ font=\scriptsize\it
    ]
    
\tikzstyle{zoomed arrow}
  = [ solid
    , decorate
    , decoration=snake
    , -stealth'
    ]

\tikzstyle{zoom}
  = [ dashed
    ]

\tikzstyle{uml class}
  = [ shape=rectangle
    , draw
    , font=\sf
    ]

\tikzstyle{uml package}
  = [ uml class
    , inner sep=1em
    ]

\tikzstyle{uml dependency}
  = [ dependency, 
    , dashed
%    , thick
    ]


\newcommand{\bluenode}{\tikz \node [ast node, blue, color=blue, text width=] {};}
\newcommand{\rednode}{\tikz \node [ast node, red, color=red, text width=] {};}
\newcommand{\greennode}{\tikz \node [ast node, green, color=green!80!black, text width=] {};}


\tikzstyle{double arrow}
  = [ shape=double arrow
    , double arrow head extend=.75em
    , double arrow head indent=.25em
    , minimum width=3em
    , draw
    , font=\sf
    ]

%% name graphs

\newdimen\nodedistance
\nodedistance=4em

\tikzstyle{name graph}
  = [ node distance=\nodedistance
    , every node/.style={namenode}
    , every path/.style={ref}
    ]


\tikzstyle{namenode}
  = [ circle
    , thick
    , anchor=center
    , draw
    , minimum size=2em
    , inner sep=2pt
    ]

\tikzstyle{synthesized}
  = [ namenode
    , fill=gray!45
    ]

\tikzstyle{ref}
  = [ -stealth'
    , semithick
    , draw
    ]

\tikzstyle{badref}
  = [ ref
    , dashed
    ]

%% expression trees

\newdimen\nodedistance
\nodedistance=4em

\tikzstyle{exp tree}
  = [ node distance=\nodedistance
    , every node/.style={exp node}
    , every path/.style={link}
    , execute at begin node={\strut}
    ]

\tikzstyle{exp node}
  = [ anchor=center
    ]
\tikzstyle{link}
  = [ semithick
    , draw
    ]

\tikzstyle{ctx node}
  = [ node distance = -.3em,
    ]
\tikzstyle{ctx edge}
  = [ color = red
    , -stealth'
    ]

\tikzstyle{type node}
  = [ node distance = -.5em,
    ]
\tikzstyle{type edge}
  = [ color = blue
    , -stealth'
    ]

%%%%%%%%%%%%%%%%%%%%%%%%%%%
% Pretty printing with tikz
%
% configuration

%%%%%%
%% TikZ and lstlistings
%%%%%%


% remembers a position on the page as a tikz coordinate
\newcommand{\coord}[1]{\tikz[remember picture] \coordinate (#1);}

% translation from baseline to upper border of box
\newcommand{\distanceTop}{7.15pt}

% translation from baseline to lower border of box
\newcommand{\distanceBottom}{-2.55pt}

% translation from baseline to left border of box
\newcommand{\distanceLeft}{-0.5pt}

% translation from baseline to right border of box
\newcommand{\distanceRight}{0.5pt}


% draws a rectangle around 2 points on the page
%
% optional parameter: TikZ style for the line
% 1. mandatory parameter: upper left corner, at text baseline height
% 2. mandatory parameter: lower right corner, at text baseline height
% 3. mandatory parameter: extra margin in all directions
\newcommand{\drawrect}[4][]{\begin{tikzpicture}[remember picture, overlay]
\draw[layout box, #1]
  ($(#2) +(\distanceLeft, \distanceTop) + (-#4, #4)$) rectangle
  ($(#3) + (\distanceRight, \distanceBottom) + (#4, -#4)$);
\end{tikzpicture}}

% draws a rectangle around 2 points on the page
%
% optional parameter: TikZ style for the line
% 1. mandatory parameter: left-hand-side of the first line, at text baseline height
% 2. mandatory parameter: a point on the left side of the rectangle
% 3. mandatory parameter: lower right corner, at text baseline height
% 4. mandatory parameter: extra margin in all directions
\newcommand{\drawbox}[5][]{\begin{tikzpicture}[remember picture, overlay]
\draw[layout box, #1]
  ($(#2) + (\distanceLeft, \distanceTop) + (-#5, #5)$) --
  ($(#4 |- #2) + (\distanceRight, \distanceTop) + (#5, #5)$) --
  ($(#4) + (\distanceRight, \distanceBottom) + (#5, -#5)$) --
  ($(#3 |- #4) + (\distanceLeft, \distanceBottom) + (-#5, -#5)$) --
  ($(#3 |- #2) + (\distanceLeft, \distanceBottom) + (-#5, -#5)$) --
  ($(#2) + (\distanceLeft, \distanceBottom) + (-#5, -#5)$) --
  cycle;
\end{tikzpicture}}

% draws a small \ppBox with three rows around the current point
% #1 = first.col
% #2 = left.col
% #3 = right.col
% #4 = last.col
\newcommand{\drawMiniBox}[4]{
  +(#2 * .2em, 0) --
  +(#2 * .2em, 1ex) --
  +(#1 * .2em, 1ex) --
  +(#1 * .2em, 1.5ex) --
  +(#3 * .2em, 1.5ex) --
  +(#3 * .2em, .5ex) --
  +(#4 * .2em, .5ex) --
  +(#4 * .2em, 0) --
  cycle
}

% A small box (for use inside text)
% optional argument = path options
% #1 = first.col
% #2 = left.col
% #3 = right.col
% #4 = last.col
\newcommand{\minibox}[5][]{%
  \begin{tikzpicture}[baseline=0pt]
  \path [mini layout box, #1]
    (0, 0) \drawMiniBox{#2}{#3}{#4}{#5};
  \end{tikzpicture}}

\newcommand{\miniboxA}{\minibox{-1}{0}{3}{3}}
\newcommand{\miniboxB}{\minibox{0}{0}{3}{3}}
\newcommand{\miniboxC}{\minibox{1}{0}{3}{3}}

% A small rectangle (for use inside text)
\newcommand{\minirect}{\hbox to 9pt{\drawrect[thin,scale=0.3,black]{0pt, 15pt}{26pt, -5pt}{0pt}}}


\tikzstyle{layout box}
  = [ blue
    % , semitransparent
    , semithick
    , draw
    ]

\tikzstyle{mini layout box}
  = [ black
    , very thin
    , draw
    ]

\tikzstyle{annotation}
  = [ font=\small\it
    ]

\tikzstyle{code annotation}
  = [ font=\small\it
    ]


% the font used for pretty printing
\newcommand{\selectCodeFont}{\sf\small}

% the font used for highlighting keywords
\newcommand{\selectKeywordFont}{\bfseries}

% the font used for highlighting string literals
\newcommand{\selectStringLitFont}{\tt}

% the font used for highlighting identifiers
\newcommand{\selectIdentifierFont}{\relax}

% the font used for highlighting operators
\newcommand{\selectOperatorFont}{\tt}

% the text height of token nodes
\newcommand{\tokenHeight}{2ex}

% the text depth of token nodes
\newcommand{\tokenDepth}{.5ex}

%%%%%%%%%%%%%%%%%%%%%%%%%%%
% Pretty printing with tikz
%
% tikz layer

% style for paths or scopes that do pretty printing
\tikzstyle{pretty print}
  = [ start chain=going base right
    , text height=\tokenHeight
    , text depth=\tokenDepth
    , inner sep=0
    , node distance=0em
    ]

% helper style for nodes that start a new line
% (automatically applied by style 'new line' below)
\tikzstyle{new line/helper}
  = [ on chain
    , anchor=north west,
      at={($(#1.west |- \tikzchainprevious.base) + (0, -1.1\baselineskip)$)}
    ]

\tikzstyle{new line tight/helper}
  = [ on chain
    , anchor=north west,
      at={($(#1.west |- \tikzchainprevious.base) + (0, -0.9\baselineskip)$)}
    ]


% style for nodes that start a new line
\tikzstyle{new line}[\tikzchainprevious]
  = [ on chain=placed {new line/helper=#1}
    ]

% style for nodes that start a new line
\tikzstyle{new line tight}[\tikzchainprevious]
  = [ on chain=placed {new line tight/helper=#1}
    ]

% helper style for nodes that tab forward to a labeled position
% (automatically applied by style 'tab forward' below)
\tikzstyle{tab forward/helper}
  = [ on chain
    , anchor=north west,
      at={(#1 |- \tikzchainprevious.base)}
    ]

% style for nodes that tab forward to a labeled position
\tikzstyle{tab forward}
  = [ on chain=placed {tab forward/helper=#1}
    ]

% style for nodes that indent
\tikzstyle{indent}[1em]
  = [ on chain=placed {base right=#1 of \tikzchainprevious}
    ]

% style for nodes that contain an identifier token
\tikzstyle{identifier}
  = [ on chain
    , font=\selectIdentifierFont
    ]

% style for nodes that contain an operator token
\tikzstyle{operator}
  = [ on chain
    , font=\selectOperatorFont
    ]

% style for nodes that contain a keyword token
\tikzstyle{keyword}
  = [ on chain
    , font=\selectKeywordFont
    , text=keyword
    ]

% style for nodes that contain a string literal
\tikzstyle{stringlit}
  = [ on chain
    , font=\selectStringLitFont
    , text=blue
    ]

%%%%%%%%%%%%%%%%%%%%%%%%%%%
% Pretty printing with tikz
%
% \ppBox layer

\newcommand{\Empty}{}
\newcommand{\ignore}[1]{\relax}

\makeatletter
\newcommand{\ppBox}[2][]{{
  % insert keyword token
  \newcommand{\KW}[1]{
    \node[keyword] {##1};
    \HandleToken{\tikzchaincurrent}}

  % insert identifier token
  \newcommand{\ID}[1]{
    \node[identifier] {##1};
    \HandleToken{\tikzchaincurrent}}

  % insert operator token
  \newcommand{\OP}[1]{
    \node[operator] {##1};
    \HandleToken{\tikzchaincurrent}}

  % insert whitespace token
  \newcommand{\SP}{
    \node[on chain] { };
    \HandleInsensitiveToken{\tikzchaincurrent}}

  % label a position (with a tikz node name)
  \newcommand{\LB}[1]{
    \coordinate[on chain] (##1);}

  % tab forward to a labeled position
  \newcommand{\TB}[1]{
    \coordinate[tab forward=##1];
  }

  % insert a string literal
  \newcommand{\STR}[1]{
    \node[stringlit] {##1};
    \HandleToken{\tikzchaincurrent}}

  % insert a line break
  \renewcommand{\\}{
    \coordinate[new line=\NewlineNode];
    \let\HandleToken=\HandleTokenSubsequentLine}

  \newcommand{\newlineTight}{
    \coordinate[new line tight=\NewlineNode];
    \let\HandleToken=\HandleTokenSubsequentLine}

  % the name of the first token
  \let\FirstToken=\Empty
  \newcommand{\AdjustFirst}[1]{
    \ifx\FirstToken\Empty
    \edef\FirstToken{##1}
    \fi
  }

  % the name of the last token
  \let\LastToken=\Empty
  \newcommand{\AdjustLast}[1]{
    \edef\LastToken{##1}
  }

  % the name of the left-most token
  % (except for tokens in the first line)
  \let\LeftToken=\Empty
  \newcommand{\AdjustLeft}[1]{
    \ifx\LeftToken\Empty
      \edef\LeftToken{##1}
    \else
      \pgf@process{\pgfpointanchor{##1}{west}}
      \setlength{\pgf@xa}{\pgf@x}
      \pgf@process{\pgfpointanchor{\LeftToken}{west}}
      \setlength{\pgf@xb}{\pgf@x}
      \ifdim\pgf@xa<\pgf@xb
        \edef\LeftToken{##1}
      \fi
    \fi
  }

  % the name of the rightmost token
  \let\RightToken=\Empty
  \newcommand{\AdjustRight}[1]{
    \ifx\RightToken\Empty
      \edef\RightToken{##1}
    \else
      \pgf@process{\pgfpointanchor{##1}{east}}
      \setlength{\pgf@xa}{\pgf@x}
      \pgf@process{\pgfpointanchor{\RightToken}{east}}
      \setlength{\pgf@xb}{\pgf@x}
      \ifdim\pgf@xa>\pgf@xb
        \edef\RightToken{##1}
      \fi
    \fi
  }

  % this is called for all layout-sensitive tokens in the first
  % line of a pretty printing group
  \newcommand{\HandleTokenFirstLine}[1]{
    \AdjustFirst{##1}
    \AdjustRight{##1}
    \AdjustLast{##1}
  }

  % this is called for all layout-sensitive tokens in subsequent
  % lines of a pretty printing group
  \newcommand{\HandleTokenSubsequentLine}[1]{
    \AdjustLeft{##1}
    \AdjustRight{##1}
    \AdjustLast{##1}}

  % this is called for layout-insensitive tokens in a pretty
  % printing group (such as whitespace, comments, and code that
  % uses explicit layout)
  \newcommand{\HandleInsensitiveToken}[1]{
  }

  % current token handler. will be set to one of:
  %  \HandleTokenFirstLine
  %  \HandleTokenSubsequentLine
  %  \HandleInsensitiveToken
  \newcommand{\HandleToken}[1]{}

  \newcommand{\ppSubBox}[2][]{
    \coordinate[on chain];
    {
      \let\NewlineNode\tikzchaincurrent

      % reset the token registers
      \let\FirstToken=\Empty
      \let\LastToken=\Empty
      \let\LeftToken=\Empty
      \let\RightToken=\Empty

      % start in first line
      \let\HandleToken=\HandleTokenFirstLine

      % typeset content
      ##2

      % draw boundary
      \ifx\LeftToken\Empty
        \path [##1]
          (\FirstToken.north west) rectangle
          (\LastToken.south east);
      \else
        \path [##1]
          (\RightToken.east |- \FirstToken.north) --
          (\FirstToken.north west) --
          (\FirstToken.south west) --
          (\LeftToken.west |- \FirstToken.south) --
          (\LeftToken.west |- \LastToken.south) --
          (\LastToken.south east) --
          (\LastToken.north east) --
          (\RightToken.east |- \LastToken.north) --
          cycle;
      \fi

      % remember the token registers outside the group
      \global\let\InnerFirstToken=\FirstToken
      \global\let\InnerLastToken=\LastToken
      \global\let\InnerLeftToken=\LeftToken
      \global\let\InnerRightToken=\RightToken

      % remember the state outside the group
      \global\let\InnerHandleToken=\HandleToken
    }
    % handle the significant inner tokens
    % and significant inner linebreaks
    \HandleToken{\InnerFirstToken}
    \ifx\HandleToken\HandleTokenFirstLine
    \let\HandleToken=\InnerHandleToken
    \fi
    \ifx\InnerLeftToken\Empty
    \else
    \HandleToken{\InnerLeftToken}
    \fi
    \HandleToken{\InnerRightToken}
    \HandleToken{\InnerLastToken}
  }

  % indent a ppBox
  \newcommand{\IN}[1]{\coordinate[indent]; ##1}
  \newcommand{\DE}[1]{\coordinate[indent=-1em]; ##1}

  \let\ppBox=\ppSubBox
  \ppBox[#1]{#2}}}
\makeatother




%%% Local Variables: 
%%% mode: latex
%%% TeX-master: "../thesis"
%%% End: 


\begin{document}

\section{Overview}
\section{DCC recap}
%\subsection{Syntax}
\begin{figure}[h]
\begin{align*}
  P &::= \ovl{D} && \text{(Program)}\\
  D &::= \constructorDeclaration{C}{x}{\ovl{a}}
    \mid \programEntailment{x}{\ovl{a}}{a}
    \mid \abstractMethodDeclaration{m}{x}{\ovl{a}}{t}
    \mid \methodImplementation{m}{x}{\ovl{a}}{t}{e} && \text{(Decl)}\\
  t &::= \type{x}{\ovl{a}} && \text{(Type)}\\
  a &::= \pathEq{p}{p}
    \mid \instOf{p}{C}
    \mid \instBy{p}{C} && \text{(Constr)}\\
  p &::= x \mid p.f && \text{(Path)}\\
  e &::= x
    \mid e.f
    \mid \objConstr{C}{\pathEq{\ovl{f}}{\ovl{e}}} && \text{(Expr)}
\end{align*}
\caption{DCC Syntax}
\label{fig:syntax}
\end{figure}

%\subsection{Constraint Entailment}
\begin{figure}[h]
  \begin{mathpar}
    \inferrule[C-Ident]{}{
      \entails{a}{a}
    }
    \and
    \inferrule[C-Refl]{}{
      \entails{\epsilon}{\pathEq{p}{p}}
    }
    \and
    \inferrule[C-Class]{
      \entails{\ovl{a}}{\instBy{p}{C}}
    }{
      \entails{\ovl{a}}{\instOf{p}{C}}
    }
    \and
    \inferrule[C-Cut]{
      \entails{\ovl{a}}{c}\\
      \entails{\ovl{a'},c}{b}
    }{
      \entails{\ovl{a},\ovl{a'}}{b}
    }
    \and
    \inferrule[C-Subst]{
      \entails{\ovl{a}}{a_{\sub{x}{p}}}\\
      \entails{\ovl{a}}{\pathEq{p'}{p}}
    }{
      \entails{\ovl{a}}{a_{\sub{x}{p'}}}
    }
    \and
    \inferrule[C-Prog]{
      (\programEntailment{x}{\ovl{a}}{a}) \in P\\
      \entails{\ovl{b}}{\ovl{a}_{\sub{x}{p}}}
    }{
      \entails{\ovl{a}}{a_{\sub{x}{p}}}
    }
  \end{mathpar}
  \caption{Constraint Entailment}
  \label{fig:constraint-entailment}
\end{figure}

\section{An Implementation}
\todo[inline]{%
  - interpreter for operational semantics\\
  - type inference for type assignments for expressions\\
  - type checking through inferring type and check subtype with annotated/expected type\\
  - constraint entailment via SMT solving
}

\section{Towards an SMT encoding}
\todo[inline]{%
  - idea (of master thesis): use smt solver\\
  - first approach: resemble syntax based reasoning to stay close to \Cref{fig:constraint-entailment}\\
    - unefficient (massive amount of ADTs and recursion needed, not what a SMT solver is good at)\\
  - second approach: use semantic reasoning, encode into FO with constraints being simple boolean predicates\\
    - infinite datatypes are problematic, we will never be able to reject an entailment\\
  - third approach: get rid of the infinite structures\\
    - limit path depth\\
    - decidable encoding, we could just finitely enumerate all quantifiers\\
  - 4th approach: ground encoding to see if it's more efficient\\
}

\section{SMT solver limitations}
\todo[inline]{%
  - recursion + quantifiers + infinite (abstractt) datatypes bad\\
  - need to limit the use of them
}

\section{Path Depth Limit Encoding}
To encode: $\entails{\instBy{x}{\Succ}, \instBy{x.p}{\Zero}, \pathEq{x}{y}}{\instOf{y}{\Nat}}$
with $\mathit{depth\!\!-\!\!limit}=1$
\begin{align}
  &\texttt{\Variable} := \{\mathtt{x}, \mathtt{y}\}\\
  &\texttt{\Class} := \{\Zero,\Succ,\Nat\}\\
  &\texttt{\Path} := \{\mathtt{x}, \mathtt{x}.\mathtt{p}, \mathtt{y}, \mathtt{y}.\mathtt{p}\}\\
  % def subst
  &\subst{p}{v}{q} = s :=\\
  &\quad (p=\mathtt{x} \land v=\mathtt{x} \land q=\mathtt{x} \land s=\mathtt{x})~\lor\\
  &\quad (p=\mathtt{x} \land v=\mathtt{x} \land q=\mathtt{x.p} \land s=\mathtt{x.p})~\lor\\
  &\quad (p=\mathtt{x} \land v=\mathtt{x} \land q=\mathtt{y} \land s=\mathtt{y})~\lor\\
  &\quad (p=\mathtt{x} \land v=\mathtt{x} \land q=\mathtt{y.p} \land s=\mathtt{y.p})~\lor\\
  %&\quad (p=\mathtt{x} \land v=\mathtt{y} \land q=\mathtt{x} \land s=\mathtt{x})~\lor\\
  %&\quad (p=\mathtt{x} \land v=\mathtt{y} \land q=\mathtt{x.p} \land s=\mathtt{x})~\lor\\
  %&\quad (p=\mathtt{x} \land v=\mathtt{y} \land q=\mathtt{y} \land s=\mathtt{x})~\lor\\
  %&\quad (p=\mathtt{x} \land v=\mathtt{y} \land q=\mathtt{y.p} \land s=\mathtt{x})~\lor\\
  &\quad (p=\mathtt{x.p} \land v=\mathtt{x} \land q=\mathtt{x} \land s=\mathtt{x.p})~\lor\\
  &\quad (p=\mathtt{x.p} \land v=\mathtt{x} \land q=\mathtt{y} \land s=\mathtt{y.p})~\lor\\
  &\quad~...\\
  % C-Refl
  &\forall p.~\pathEq{p}{p} && \text{(C-Refl)}\\
  % C-Class
  &\forall p, c.~\instBy{p}{c} \rightarrow \instOf{p}{c} && \text{(C-Class)}\\
  % C-Subst: pathEq
  &\forall p, q, v, r, s, a, b, c, d. && \text{(C-Subst)}\\
  &\quad \pathEq{s}{r} \land \subst{p}{v}{r}=a \land \subst{q}{v}{r}=b~\land\\
  &\quad \pathEq{a}{b} \land
         \subst{p}{v}{s}=c \land \subst{q}{v}{s}=d\\
  &\qquad \rightarrow \pathEq{c}{d}\\
  % C-Subst: instOf
  &\forall p, c, v, r, s, a, b. && \text{(C-Subst)}\\
  &\quad \pathEq{s}{r} \land \subst{p}{v}{r}=a~\land\\
  &\quad \instOf{a}{c} \land
         \subst{p}{v}{s}=b\\
  &\qquad \rightarrow \instOf{b}{c}\\
  % C-Subst: instBy
  &\forall p, c, v, r, s, a, b. && \text{(C-Subst)}\\
  &\quad \pathEq{s}{r} \land \subst{p}{v}{r}=a~\land\\
  &\quad \instBy{a}{c} \land
         \subst{p}{v}{s}=b\\
  &\qquad \rightarrow \instBy{b}{c}\\
  % C-Prog
  &\instOf{\mathtt{x}}{\Zero} \rightarrow \instOf{\mathtt{x}}{\Nat} && \text{(C-Prog)}\\
  &\instOf{\mathtt{x.p}}{\Zero} \rightarrow \instOf{\mathtt{x.p}}{\Nat} && \text{(C-Prog)}\\
  &\instOf{\mathtt{x}}{\Succ} \land \instOf{\mathtt{x.p}}{\Nat} \rightarrow \instOf{\mathtt{x}}{\Nat} && \text{(C-Prog)}\\
  &\instOf{\mathtt{y}}{\Zero} \rightarrow \instOf{\mathtt{y}}{\Nat} && \text{(C-Prog)}\\
  &\instOf{\mathtt{y.p}}{\Zero} \rightarrow \instOf{\mathtt{y.p}}{\Nat} && \text{(C-Prog)}\\
  &\instOf{\mathtt{y}}{\Succ} \land \instOf{\mathtt{y.p}}{\Nat} \rightarrow \instOf{\mathtt{y}}{\Nat} && \text{(C-Prog)}\\
  % Entailment
  &\neg (\instBy{\mathtt{x}}{\Succ} \land \instBy{\mathtt{x.p}}{\Zero} \land \pathEq{\mathtt{x}}{\mathtt{y}} \rightarrow \instOf{\mathtt{y}}{\Nat})
\end{align}
\newpage

\section{Ground Encoding}
To encode: $\entails{\instBy{x}{\Succ}, \instBy{x.p}{\Zero}, \pathEq{x}{y}}{\instOf{y}{\Nat}}$
with $\mathit{depth\!\!-\!\!limit}=1$
\setcounter{equation}{0}
% Alternatively for each align block
% \usepackage{etoolbox}
% \AtBeginEnvironment{align}{\setcounter{equation}{0}}
\begin{align}
  &\texttt{\Variable} := \{\mathtt{x}, \mathtt{y}\}\\
  &\texttt{\Class} := \{\Zero,\Succ,\Nat\}\\
  &\texttt{\Path} := \{\mathtt{x}, \mathtt{x.p}, \mathtt{y}, \mathtt{y.p}\}\\
  % C-Refl
  &\pathEq{\mathtt{x}}{\mathtt{x}} \land
  \pathEq{\mathtt{x.p}}{\mathtt{x.p}} \land
  \pathEq{\mathtt{y}}{\mathtt{y}} \land
  \pathEq{\mathtt{y.p}}{\mathtt{y.p}} && \text{(C-Refl)}\\
  % C-Class
  &\instBy{\mathtt{x}}{\Zero} \rightarrow \instOf{\mathtt{x}}{\Zero} && \text{(C-Class)}\\
  &\instBy{\mathtt{x.p}}{\Zero} \rightarrow \instOf{\mathtt{x.p}}{\Zero} && \text{(C-Class)}\\
  & ...  && \text{(C-Class)}\\
  % C-Subst
  & \pathEq{\mathtt{x}}{\mathtt{y}} \land \pathEq{\mathtt{y}}{\mathtt{y}} \rightarrow \pathEq{\mathtt{y}}{\mathtt{x}} && \text{(C-Subst)}\\
  & \pathEq{\mathtt{x}}{\mathtt{y}} \land \instOf{\mathtt{y}}{Nat} \rightarrow \instOf{\mathtt{x}}{\Nat} && \text{(C-Subst)}\\
  & \pathEq{\mathtt{x}}{\mathtt{y}} \land \instBy{\mathtt{y}}{Nat} \rightarrow \instBy{\mathtt{x}}{\Nat} && \text{(C-Subst)}\\
  & ... && \text{(C-Subst)}\\
  % C-Prog
  &\instOf{\mathtt{x}}{\Zero} \rightarrow \instOf{\mathtt{x}}{\Nat} && \text{(C-Prog)}\\
  &\instOf{\mathtt{x.p}}{\Zero} \rightarrow \instOf{\mathtt{x.p}}{\Nat} && \text{(C-Prog)}\\
  &\instOf{\mathtt{x}}{\Succ} \land \instOf{\mathtt{x.p}}{\Nat} \rightarrow \instOf{\mathtt{x}}{\Nat} && \text{(C-Prog)}\\
  &\instOf{\mathtt{y}}{\Zero} \rightarrow \instOf{\mathtt{y}}{\Nat} && \text{(C-Prog)}\\
  &\instOf{\mathtt{y.p}}{\Zero} \rightarrow \instOf{\mathtt{y.p}}{\Nat} && \text{(C-Prog)}\\
  &\instOf{\mathtt{y}}{\Succ} \land \instOf{\mathtt{y.p}}{\Nat} \rightarrow \instOf{\mathtt{y}}{\Nat} && \text{(C-Prog)}\\
  % Many more
  &...\\
  % Entailment
  &\neg (\instBy{\mathtt{x}}{\Succ} \land
        \instBy{\mathtt{x.p}}{\Zero} \land
        \pathEq{\mathtt{x}}{\mathtt{y}} \rightarrow
          \instOf{y}{\Nat})
\end{align}

\newpage
\subsection{Substitution in the Ground Encoding}
We want to finitely enumerate the quantified rule:
\begin{align*}
  % C-Subst: instOf
  &\forall p, c, v, r, s, a, b. && \text{(C-Subst)}\\
  &\quad \pathEq{s}{r} \land \subst{p}{v}{r}=a~\land\\
  &\quad \instOf{a}{c} \land
         \subst{p}{v}{s}=b\\
  &\qquad \rightarrow \instOf{b}{c}
\end{align*}
The na\"ive approach would be to take the cross product of all quantified variables.
This would leave us with a lot of meaningless implications,
e.g. if we instantiate the rule with
$p=\mathtt{x}, v=\mathtt{y}, r=\mathtt{x}, a=\mathtt{y}, s=\mathtt{x}, b=\mathtt{x}, c=\Nat$
\begin{align*}
  % C-Subst: instOf
  &\pathEq{\mathtt{x}}{\mathtt{x}} \land {\color{red}\subst{\mathtt{x}}{\mathtt{y}}{\mathtt{x}}=\mathtt{y}}~\land
  \instOf{\mathtt{y}}{\Nat} \land \subst{\mathtt{x}}{\mathtt{y}}{\mathtt{x}}=\mathtt{x}\\
  &\quad \rightarrow \instOf{\mathtt{x}}{\Nat}
\end{align*}
Since we know the substitution to be false,
we do not have to include this instantiation into the encoding.
Since we only need to include rule instantiations where the substitution
predicate holds and we can calculate the substitution prior
since all quantified variables are known,
we can get rid of the substitution predicate in the encoding altogether.\\
\\
E.g. the inatantiation with
$p=\mathtt{x}, v=\mathtt{x}, r=\mathtt{y}, a=\mathtt{y}, s=\mathtt{x}, b=\mathtt{x}, c=\Nat$
\begin{align*}
  % C-Subst: instOf
  &\pathEq{\mathtt{x}}{\mathtt{y}} \land
   {\color{blue}\subst{\mathtt{x}}{\mathtt{x}}{\mathtt{y}}=\mathtt{y}}~\land
   \instOf{\mathtt{y}}{\Nat} \land
   {\color{blue}\subst{\mathtt{x}}{\mathtt{x}}{\mathtt{x}}=\mathtt{x}}\\
  &\quad \rightarrow \instOf{\mathtt{x}}{\Nat}
\end{align*}
turns into
\[
  \pathEq{\mathtt{x}}{\mathtt{y}} \land \instOf{\mathtt{y}}{Nat} \rightarrow \instOf{\mathtt{x}}{\Nat}
\]

\section{Algorithmic Symmetry}
\label{sec:algo-symmetry}
We rely on the equivalency between the declarative- and the algorithmic system
to set our depth limit for path enumeration
as well as on the decidability of the declarative system to even have such a limit in place.\\
\\
The entailment $\entails{\pathEq{a}{b}}{\pathEq{b}{a}}$ is a counterexample
to \Cref{lem:2} (more precisely \Cref{lem:1})
and \Cref{thm:1} as it relies on \Cref{lem:2}.

\begin{lemma}[5.5.15]
  \label{lem:1}
  If \wf{P} and \entails{\ovl{a}}{a} then \entailsA{\ovl{a}}{a}.
\end{lemma}

\begin{lemma}[5.5.16]
  \label{lem:2}
  If \wf{P} then \entails{\ovl{a}}{a} iff \entailsA{\ovl{a}}{a}.
\end{lemma}

\begin{theorem}[5.5.1]
  \label{thm:1}
  If \wf{P} then derivation of \entails{\ovl{a}}{a} is decidable.
\end{theorem}

Counterexample for \Cref{lem:1}.
Choose any well-formed program.
\begin{prooftree}
  \AxiomC{}
  \RightLabel{C-Refl}
  \UnaryInfC{\entails{\cdot}{\pathEq{b}{b}}}
  \RightLabel{C-Weak}
  \UnaryInfC{\entails{\pathEq{a}{b}}{ \pathEq{b}{a}_{\sub{a}{b}} }}
  \AxiomC{}
  \RightLabel{C-Ident}
  \UnaryInfC{\entails{\pathEq{a}{b}}{\pathEq{a}{b}}}
  \RightLabel{C-Subst}
  \BinaryInfC{\entails{\pathEq{a}{b}}{ \pathEq{b}{a}_{\sub{a}{a}} }}
\end{prooftree}

\begin{prooftree}
  \AxiomC{}
  \RightLabel{CA-Refl}
  \UnaryInfC{\entailsA{\pathEq{a}{b}}{\pathEq{a}{a}}}
  \AxiomC{$a \sqsubset \pathEq{a}{b}$}
  \AxiomC{...}
  \UnaryInfC{\entailsA{\pathEq{a}{b}}{\pathEq{b}{a}}}
  \RightLabel{CA-Subst3}
  \TrinaryInfC{\entailsA{\pathEq{a}{b}}{\pathEq{b}{a}}}
\end{prooftree}

\begin{prooftree}
  \AxiomC{...}
  \UnaryInfC{\entailsA{\pathEq{a}{b}}{\pathEq{b}{a}}}
  \AxiomC{$b \sqsubset \pathEq{a}{b}$}
  \AxiomC{}
  \RightLabel{CA-Refl}
  \UnaryInfC{\entailsA{\pathEq{a}{b}}{\pathEq{b}{b}}}
  \RightLabel{CA-Subst3}
  \TrinaryInfC{\entailsA{\pathEq{a}{b}}{\pathEq{b}{a}}}
\end{prooftree}

\subsection{Symmetry Fix}
\label{sec:symmetry-fix}
We can update rule CA-Subst3 to allow the entailment used as a counterexample
to \Cref{lem:1} to have a derivation.

There are two feasible ways to update the rule:
\begin{enumerate}
  \item
  \begin{prooftree}
    \AxiomC{\entailsA{\ovl{a}}{\pathEq{p}{p''}}}
    \AxiomC{$p \sqsubset \ovl{a}$}
    \AxiomC{\entailsA{\ovl{a}}{\pathEq{p}{p'}}}
    \RightLabel{CA-Subst3Fix1}
    \TrinaryInfC{\entailsA{\ovl{a}}{\pathEq{p'}{p''}}}
  \end{prooftree}
  \item
  \begin{prooftree}
    \AxiomC{\entailsA{\ovl{a}}{\pathEq{p''}{p}}}
    \AxiomC{$p \sqsubset \ovl{a}$}
    \AxiomC{\entailsA{\ovl{a}}{\pathEq{p'}{p}}}
    \RightLabel{CA-Subst3Fix2}
    \TrinaryInfC{\entailsA{\ovl{a}}{\pathEq{p'}{p''}}}
  \end{prooftree}
\end{enumerate}

\subsubsection{Fix 1 derivation}
\begin{prooftree}
  \AxiomC{}
  \RightLabel{CA-Refl}
  \UnaryInfC{\entailsA{\pathEq{a}{b}}{\pathEq{a}{a}}}
  \AxiomC{$a \sqsubset \pathEq{a}{b}$}
  \AxiomC{}
  \RightLabel{CA-Ident}
  \UnaryInfC{\entailsA{\pathEq{a}{b}}{\pathEq{a}{b}}}
  \RightLabel{CA-Subst3Fix1}
  \TrinaryInfC{\entailsA{\pathEq{a}{b}}{\pathEq{b}{a}}}
\end{prooftree}

\subsubsection{Fix 2 derivation}
\begin{prooftree}
  \AxiomC{}
  \RightLabel{CA-Ident}
  \UnaryInfC{\entailsA{\pathEq{a}{b}}{\pathEq{a}{b}}}
  \AxiomC{$b \sqsubset \pathEq{a}{b}$}
  \AxiomC{}
  \RightLabel{CA-Refl}
  \UnaryInfC{\entailsA{\pathEq{a}{b}}{\pathEq{b}{b}}}
  \RightLabel{CA-Subst3Fix2}
  \TrinaryInfC{\entailsA{\pathEq{a}{b}}{\pathEq{b}{a}}}
\end{prooftree}

\subsection{What to do with this?}
\begin{itemize}
  \item Update rule CA-Subst3 with one of the solutions from \Cref{sec:symmetry-fix}.
  \item Redo the proofs?
\end{itemize}

%\section{Runtime Comparison}
%add table

\end{document}
