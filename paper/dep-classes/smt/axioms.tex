\documentclass{article}          % German article document class
\usepackage[utf8]{inputenc}       % Input is utf8
\usepackage[T1]{fontenc}          % T1 font
\usepackage{amsmath}              % package for formulars and other math stuff
\usepackage{amssymb}              % package for math symbols
\usepackage{amsthm}               % library for theorems etc.
\usepackage{xcolor}
\usepackage{mathpartir, proof}
\usepackage{nicefrac}             % nice inline fractions in this style: a/b
\usepackage{ulem}                 % package for underlining
\usepackage{polynom}              % package for polynom-division etc.
\usepackage{paralist}             % you can format enumerations easy
                                  % e.g. \begin{enumerate}[1.]
\begin{document}

\section{Declarations}
Enumeration Types: \texttt{Class}, \texttt{Variable}, \texttt{Field}\\
ADT: \texttt{Path}\\
Propositions: \textsf{instOf}, \textsf{instBy}, \textsf{pathEq}\\
Recursive Function: \textsf{subst-path}

\section{Basic Rules}
\begin{align*}
&\forall p:\mathtt{Path}.~\mathsf{pathEq}(p,p)\\
(&\forall a:\mathtt{Bool}.~a \rightarrow a) \equiv \mathtt{true}\\
&\forall a:\mathtt{Bool}, p:\mathtt{Path}, c:\mathtt{Class}.~(a \rightarrow \mathsf{instBy}(p, c)) \rightarrow (a \rightarrow \mathsf{instOf}(p, c))
\end{align*}

\section{Substitution Rule}
\begin{align*}
&\forall as:\mathtt{Bool}, a:\mathtt{Bool}, p_1:\mathtt{Path}, p_2:\mathtt{Path}, x:\mathtt{Variable}.\\
&\hspace{2em} (as \rightarrow {\color{red} a_{\{x \mapsto p_1\}}} \land as \rightarrow \mathsf{pathEq}(p_2, p_1))\\
&\hspace{2em} \rightarrow (as \rightarrow {\color{red} a_{\{x \mapsto p_2\}}})
\end{align*}
Problem: Since constraints are modeled as booleans, substitution cannot be applied.\\
%
Instantiate $a$ with every possible constraint type proposition, to allow for path substitution.
\begin{align*}
&\forall as:\mathtt{Bool}, q_1:\mathtt{Path}, q_2:\mathtt{Path}, p_1:\mathtt{Path}, p_2:\mathtt{Path}, x:\mathtt{Variable}.\\
&\hspace{2em} ((as \rightarrow \mathsf{pathEq}({q_1}_{\{x \mapsto p_1\}},{q_2}_{\{x \mapsto p_1\}})) \land (as \rightarrow \mathsf{pathEq}(p_2, p_1)))\\
&\hspace{2em} \rightarrow (as \rightarrow \mathsf{pathEq}({q_1}_{\{x \mapsto p_2\}},{q_2}_{\{x \mapsto p_2\}}))\\
%%%%%%%%%%%%%%%%%%%%%%%%%%%%
&\forall as:\mathtt{Bool}, q:\mathtt{Path}, c:\mathtt{Class}, p_1:\mathtt{Path}, p_2:\mathtt{Path}, x:\mathtt{Variable}.\\
&\hspace{2em} ((as \rightarrow \mathsf{instOf}({q}_{\{x \mapsto p_1\}},c)) \land (as \rightarrow \mathsf{pathEq}(p_2, p_1)))\\
&\hspace{2em} \rightarrow (as \rightarrow \mathsf{instOf}({q}_{\{x \mapsto p_2\}},c))\\
%%%%%%%%%%%%%%%%%%%%%%%%%%%%
&\forall as:\mathtt{Bool}, q:\mathtt{Path}, c:\mathtt{Class}, p_1:\mathtt{Path}, p_2:\mathtt{Path}, x:\mathtt{Variable}.\\
&\hspace{2em} ((as \rightarrow \mathsf{instBy}({q}_{\{x \mapsto p_1\}},c)) \land (as \rightarrow \mathsf{pathEq}(p_2, p_1)))\\
&\hspace{2em} \rightarrow (as \rightarrow \mathsf{instBy}({q}_{\{x \mapsto p_2\}},c))\\
\end{align*}

\newpage
\section{Program Entailment Rule}
\begin{mathpar}
\inferrule[C-Prog]{
  (\forall x.~\overline{a} \Rightarrow a) \in P \\
  \overline{b} \vdash \overline{a}_{\{x \mapsto p\}}
}{
  \overline{b} \vdash a_{\{x \mapsto p\}}
}
\end{mathpar}
Naive version
\begin{align*}
&\forall bs:\mathtt{Bool}, a:\mathtt{Bool}, as:\mathtt{Bool}, x:\mathtt{Variable}, p:\mathtt{Path}.\\
&\hspace{2em} ((as \rightarrow a) \land (bs \rightarrow {\color{red} as_{\{x \mapsto p\}}}))\\
&\hspace{2em} \rightarrow (bs \rightarrow {\color{orange} a_{\{x \mapsto p\}}})
\end{align*}
%
From well-formedness we know that $a$ must be a constraint of the form $\mathsf{instOf}(x,c)$.
%and gives us the guarantee that a constraint of the form $\mathsf{instOf}(x,c')$ is contained in $as$ ($as$ is a conjunction of at least $\mathsf{instOf}(x,c')$ for some class $c'$).
We also statically know all program entailments, and that there is only a finite amount of these declarations per program.
\begin{align*}
&\forall bs:\mathtt{Bool}, c:\mathtt{Class}, as:\mathtt{Bool}, x:\mathtt{Variable}, p:\mathtt{Path}.\\
&\hspace{2em} ((as \rightarrow \mathsf{instOf}(x,c)) \land (bs \rightarrow {\color{red} as_{\{x \mapsto p\}}}))\\
&\hspace{2em} \rightarrow (bs \rightarrow \mathsf{instOf}(p,c))
\end{align*}
%
We can enumerate over all program entailment declarations and create a custom C-Prog rule for each declaration.
As the conjunction $as$ is known, we can explicitly apply the path substitution for each constraint in the conjunction.
\end{document}
