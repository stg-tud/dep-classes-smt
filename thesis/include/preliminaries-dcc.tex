\subsection{$DC_C$ Calculus}
The $vc^n$ calculus does not support abstract classes and methods.
The $DC_C$ calculus\cite{vaidas:thesis} extends $vc^n$ with support for abstract classes and methods
and symmetric method dispatch.
$DC_C$ encodes dependent classes with a constraint system,
which is used in both the static and dynamic semantics.
The main relation of the $DC_C$ calculus is constraint entailment,
replacing $vc^n$'s relations for type equivalence, subtyping, static and dynamic dispatch.

The runtime structure is heap based,
with explicit object identities
and relationships between objects based on these identities.
Expressions evaluate to an identifier pointing to an object in the heap.
Heaps preserve object identity and enable shared references to objects.
The heap provides a direct interpretation for equivalent paths,
two paths are equivalent if they point to the same object at runtime.
Heaps can be easily translated to a set of constraints describing its objects and their relations,
which enables the usage of the constraint system for dynamic dispatch and expression typing.
%- DCc extends $vc^n$ with abstract methods and classes and symmetric method dispatch
%- gave up unification of methods and classes
%- runtime structure: heap
%- heap structure preserves object identity, enables shared references
%- heap provides direct interpretation for equivalent paths
%    - paths are equiv if they point to the same object at runtime
%- heaps can be easily translated to a set of constraints describing its objects and their relations
%    - enables using constraint system for dynamic dispatch and expression typing
\subsubsection{Syntax}
The syntax of $DC_C$ is given in figure \ref{fig:dcc-syntax}.
Types are lists of constraints to be satisfied by their instances.
Types have the form $[x. \overline(a)]$, where $x$ is a bound variable
and $\overline(a)$ is a list of constraints on $x$.
An object belongs to a type if it fulfills its constraints.

Constraints of the form $p \equiv q$ express that two paths $p$ and $q$ are equivalent
and paths are considered to be equivalent if they refer to the same object at runtime.
$p :: C$ specifies that path $p$ refers to an instance of class $C$.
The stronger form $p.\textbf{cls} \equiv C$ denotes that
path $p$ refers to an object instantiated by a constructor of class $C$,
excluding indirect instances of $C$ inferred through inheritance rules.

A Program $P$ consists of a list of declarations $D$.
Possible declarations are constructor declarations,
abstract method declarations, method implementations and constraint entailment rules.

A Path expression can be a variable $x$ or navigation over fields starting from a variable e.g. $x.f$.

Expressions can be variables, field access, object construction and method invocation.
Field assignments are not supported since the calculus is functional.

% begin syntax figure
\begin{figure}
\setlength{\grammarindent}{5em} % increase separation between LHS/RHS
% TODO: use <> on construction at rhs?
\begin{grammar}
<Program>  ::= $\overline{Decl}$

<Decl> ::= \constr{C}{x}{\overline{Constr}} % C(x. $\overline{Constr}$)
       | \progEnt{x}{\overline{Constr}}{Constr}
       \alt \mDecl{m}{x}{\overline{Constr}}{Type} % m(x. $\overline{Constr}$): Type
       | \mImpl{m}{x}{\overline{Constr}}{Type}{Expr} % m(x. $\overline{Constr}$): Type := Expr

<Type> ::= [x. $\overline{Constr}$]

<Constr> ::= \pathEq{Path}{Path} % Path $\equiv$ Path
           | \instOf{Path}{C} % Path :: C
           | \instBy{Path}{C} % Path.\textbf{cls} $\equiv$ C

<Path> ::= x
       | Path.f

<Expr> ::= x
       | Expr.f
       | \newInst{C}{\overline{f}}{\overline{Expr}} % \textbf{new} C($\overline{f}$ $\equiv$ $\overline{Expr}$)
       | m(Expr)
\end{grammar}

\begin{align*}
MType(m, x, y) &= \{ \langle\overline{a}, \overline{b}\rangle | (m(x. \overline{a}): [y. \overline{b}]...) \in P \} \\
MImpl(m, x) &= \{\langle\overline{a}, e\rangle | (m(x. \overline{a}): [y. \overline{b}] := e) \in P \}
\end{align*}

x, y $\in$ variable names\\
f $\in$ field names\\
C $\in$ class names\\
m $\in$ method names
\caption{Syntax}
\label{fig:dcc-syntax}
\end{figure}
% end syntax figure

\subsubsection{Constraint System}
The constraint system is given in the style of the sequent calculus
and the rules for the constraint system are specified in figure \ref{fig:dcc-constraint-entailment}.
The sequent \entails{\ovl{a}}{a} is interpreted as constraint entailment:
constraints \ovl{a} entail constraint $a$.
The constraints on the left-hand side are refered to as the context
and the constraint on the right-hand side as the constraint entailed by the context.
The notion \entails{\ovl{a}}{\ovl{b}} is used differently than in the sequent calculus.
For $DC_C$ it is used as a shortcut for a list of judgements
\entails{\ovl{a}}{b_i} for each $b_i \in \ovl{b}$,
meaning that all $b_i$ are entailed by \ovl{a}.

Rules C-Ident and C-Cut are standard rules of the sequent calculus,
the remainder of the rules are specific to the programming language.
The standard structural rules of the sequent calculus allowing
permutation, weakening and constraction of the context
are implicitly assumed to be specified.

The properties of path equivalence are specified with rules C-Refl and C-Subst.
Rule C-Refl establishes reflexifity of path equivalence.
Rule C-Subst specifies that paths can be substituted with equivalent paths
at any position of any other constraint.
Other typical rules of equivalence as symmetry and transitivity can be
derived from these rules.

Rule C-Class specifies that a direct instance of a class
is an instance of that class, describing that
\instOf{p}{C} is a weaker relationship than \instBy{p}{C}.

With Rule C-Prog it is possible to specify new axioms for the constraint system in programs.
This is used to express inheritance declarations between dependent classes:
the constraint at the right-hand side of the implication must be \instOf{x}{p},
where $x$ is the bound variable of the rule.
The rule is restricted to avoid an undecidable constraint system
and the restrictions are specified by the well-formedness rule WF-RD in figure \ref{fig:dcc-wf}.

% begin constraint system figure
\begin{figure}
% C-Ident
\begin{prooftree}
\AxiomC{}
\RightLabel{(C-Ident)}
\UnaryInfC{\entails{a}{a}}
\end{prooftree}
% C-Refl
\begin{prooftree}
\AxiomC{}
\RightLabel{(C-Refl)}
\UnaryInfC{\entails{\epsilon}{\pathEq{p}{p}}}
\end{prooftree}
% C-Class
\begin{prooftree}
\AxiomC{\entails{\overline{a}}{\instantiatedBy{p}{C}}}
\RightLabel{(C-Class)}
\UnaryInfC{\entails{\overline{a}}{\instanceOf{p}{C}}}
\end{prooftree}
% C-Cut
\begin{prooftree}
\AxiomC{\entails{\overline{a}}{c}}
\AxiomC{\entails{\overline{a'}, c}{b}}
\RightLabel{(C-Cut)}
\BinaryInfC{\entails{\overline{a},\overline{a'}}{b}}
\end{prooftree}
% C-Subst
\begin{prooftree}
\AxiomC{\entails{\overline{a}}{a_{\sub{x}{p}}}}
\AxiomC{\entails{\overline{a}}{\pathEq{p'}{p}}}
\RightLabel{(C-Subst)}
\BinaryInfC{\entails{\overline{a}}{a_{\sub{x}{p'}}}}
\end{prooftree}
% C-Prog
\begin{prooftree}
\AxiomC{$(\progEnt{x}{\overline{a}}{a}) \in P$}
\AxiomC{\entails{\overline{b}}{\overline{a}_{\sub{x}{p}}}}
\RightLabel{(C-Prog)}
\BinaryInfC{\entails{\overline{b}}{a_{\sub{x}{p}}}}
\end{prooftree}
\caption{Constraint Entailment}
\label{fig:dcc-constraint-entailment}
\end{figure}
% end constraint system figure

\subsubsection{Operational Semantics}
The operational semantics are given in figure \ref{fig:dcc-opsemantics}.

% begin Operational Semantics figure
\begin{figure}
% TODO: both alignments side-by-side
\begin{align*}
o &::= \stdobj \\ % \langle C; \overline{f} \equiv \overline{x} \rangle \\
h &::= \stdheap\ \ (x_i \text{ distinct})
\end{align*}
\begin{align*}
OC(x, o) &= (\instBy{x}{C}, \pathEq{x.\overline{f}}{\overline{x}}) &&\text{where } o = \stdobj \\ % \langle C; \overline{f}\equiv\overline{x} \rangle \\
HC(h) &= \bigcup_i OC(x_i, o_i) &&\text{where } h = \stdheap % \overline{x} \mapsto \overline{o}
\end{align*}
% vertical inference rule example
%\begin{prooftree}
%\AxiomC{$A\lor B$}
%\AxiomC{$[A]$}
%\noLine
%\UnaryInfC{$C$}
%\AxiomC{$[B]$}
%\noLine
%\UnaryInfC{$C$}
%\TrinaryInfC{$C$}
%\end{prooftree}
% R-New
\begin{prooftree}
\AxiomC{$x \not \in dom(h)$} % 1
\noLine
\UnaryInfC{$\constr{C}{x}{\overline{b}} \in P$} % 3
\AxiomC{o = \stdobj} % 2
\noLine
\UnaryInfC{\entails{HC(h), OC(x, o)}{\overline{b}}} % 4
\RightLabel{R-New}
\BinaryInfC{$\eval{h}{\newInst{C}{\overline{f}}{\overline{x}}}{h, x \mapsto o}{x}$}
\end{prooftree}
% R-Field
\begin{prooftree}
\AxiomC{$(\pathEq{x.f}{y}) \in HC(h)$}
\RightLabel{R-Field}
\UnaryInfC{\eval{h}{x.f}{h}{y}}
\end{prooftree}
% R-Call
\begin{prooftree}
\AxiomC{$S = \{ \pair{\overline{a}}{e}\ |\ \pair{\overline{a}}{e} \in MImpl(m, x) \land \entails{HC(h)}{\overline{a}} \}$}
\AxiomC{$\pair{\overline{a}}{e} \in S$}
\noLine
\BinaryInfC{$\forall \pair{\overline{a'}}{e'} \in S.\ (e' \neq e) \longrightarrow (\entails{\overline{a'}}{\overline{a}}) \land \neg(\entails{\overline{a}}{\overline{a'}})$}
\RightLabel{R-Call}
\UnaryInfC{\eval{h}{m(x)}{h}{e}}
\end{prooftree}
% RC-Field
\begin{prooftree}
\AxiomC{\eval{h}{e}{h'}{e'}}
\RightLabel{RC-Field}
\UnaryInfC{\eval{h}{e.f}{h'}{e'.f}}
\end{prooftree}
% RC-Call
\begin{prooftree}
\AxiomC{\eval{h}{e}{h'}{e'}}
\RightLabel{RC-Call}
\UnaryInfC{\eval{h}{m(e)}{j'}{m(e')}}
\end{prooftree}
% RC-New
\begin{prooftree}
\AxiomC{\eval{h}{e}{h'}{e'}}
\RightLabel{RC-New}
\UnaryInfC{ % TODO: too long for box
  \eval
    {h}
    {\newInst{C}
      {\overline{f} \equiv \overline{x}, f}
      {e, \overline{f'} \equiv \overline{e'}}}
    {h'}
    {\newInst{C}
      {\overline{f} \equiv \overline{x}, f}
      {e', \overline{f'} \equiv \overline{e'}}}}
\end{prooftree}

\caption{Operational semantics}
\label{fig:dcc-opsemantics}
\end{figure}
% end Operational Semantics figure

\subsubsection{Type Checking}
% begin Type assignment figure
\begin{figure}
% T-Field
\begin{prooftree}
\AxiomC{\typeass{\ovl{c}}{e}{[x.\ \ovl{a}]}}
\AxiomC{\entails{\ovl{c},\ovl{a}}{\instOf{x.f}{C}}}
\AxiomC{\entails{\ovl{c}, \ovl{a}, \pathEq{x.f}{y}}{\ovl{b}}}
\AxiomC{$x \not \in \FV{\ovl{b}}$}
\RightLabel{T-Field}
\QuaternaryInfC{\typeass{\overline{c}}{e.f}{[y.\ \overline{b}]}}
\end{prooftree}
% T-Var
\begin{prooftree}
\AxiomC{\entails{\ovl{c}}{\instOf{x}{C}}}
\RightLabel{T-Var}
\UnaryInfC{\typeass{\ovl{c}}{x}{[y.\ \pathEq{y}{x}]}}
\end{prooftree}
% T-Call
\begin{prooftree}
\AxiomC{\entails{\ovl{c}, \ovl{a}}{\ovl{a'}}} % 3
\AxiomC{\typeass{\ovl{c}}{e}{[x.\ \ovl{a}]}} % 1
\AxiomC{$\pair{\ovl{a'}}{\ovl{b}} \in MType(m, x, y)$} % 2
\noLine
\BinaryInfC{\entails{\ovl{c},\ovl{a},\ovl{b}}{\ovl{b'}}} % 4
\AxiomC{$x \not \in \FV{\ovl{b'}}$} % 5
\RightLabel{T-Call}
\TrinaryInfC{\typeass{\ovl{c}}{m(e)}{[y.\ \ovl{b'}]}}
\end{prooftree}
% T-New
\begin{prooftree}
\AxiomC{$\forall i.\ \typeass{\ovl{c}}{e_i}{[x_i.\ \ovl{a_i}]}$} % 1
\noLine
\UnaryInfC{$\ovl{b} = (\instBy{x}{C}), \bigcup_i \ovl{a_i}_{\sub{x_i}{x.f_i}}$} % 3
\AxiomC{$\constr{C}{x}{\ovl{b'}} \in P$} % 2
\noLine
\UnaryInfC{\entails{\ovl{c},\ovl{b}}{\ovl{b'}}} % 4
\RightLabel{T-New}
\BinaryInfC{\typeass{\ovl{c}}{\newInst{C}{\ovl{f}}{\ovl{e}}}{[x.\ \ovl{b}]}}
\end{prooftree}
% T-Sub
\begin{prooftree}
\AxiomC{\typeass{\ovl{c}}{e}{[x.\ \ovl{a'}]}}
\AxiomC{\entails{\ovl{c},\ovl{a'}}{\ovl{a}}}
\RightLabel{T-Sub}
\BinaryInfC{\typeass{\ovl{c}}{e}{[x.\ \ovl{a}]}}
\end{prooftree}
\caption{Type assignment}
\label{fig:dcc-typeass}
\end{figure}
% end Type assignment figure
% begin Type Checking figure
\begin{figure}
% WF-CD
\begin{prooftree}
\AxiomC{\FVeq{\ovl{a}}{x}}
\RightLabel{WF-CD}
\UnaryInfC{\wf{\constr{C}{x}{\ovl{a}}}}
\end{prooftree}
% WF-MS
\begin{prooftree}
\AxiomC{\FVeq{\ovl{a}}{x}}
\AxiomC{\FVeq{\ovl{b}}{x,y}}
\RightLabel{WF-MS}
\BinaryInfC{\wf{(\mDecl{m}{x}{\ovl{a}}{[y.\ \ovl{b}]})}}
\end{prooftree}
% WF-RD
\begin{prooftree}
\AxiomC{\FVeq{\ovl{a}}{x}}
\AxiomC{$\instOf{x}{C'} \in \ovl{a}$}
\RightLabel{WF-RD}
\BinaryInfC{\wf{(\progEnt{x}{\ovl{a}}{\instOf{x}{C}})}}
\end{prooftree}
% WF-MI
\begin{prooftree}
\AxiomC{\FVeq{\ovl{a}}{x}}
\AxiomC{\FVeq{\ovl{b}}{x,y}}
\AxiomC{\typeass{\ovl{a}}{e}{[y.\ \ovl{b}]}}
\RightLabel{WF-Mi}
\TrinaryInfC{\wf{(\mImpl{m}{x}{\ovl{a}}{[y.\ \ovl{b}]}{e})}}
\end{prooftree}
% WF-Prog
\begin{prooftree}
\AxiomC{$\forall D \in P.\ \wf{D}$}
\noLine
\UnaryInfC{$\forall m.\ \forall \pair{\ovl{a}}{\ovl{b}}, \pair{\ovl{a'}}{\ovl{b'}} \in MType(m, x, y).\ \ovl{b} = \ovl{b'}$}
\noLine
\UnaryInfC{$\forall m.\ unique(m)$}
\noLine
\UnaryInfC{$\forall m.\ \forall \pair{\ovl{a}}{\ovl{b}} \in MType(m, x, y).\ complete(m, [x.\ \ovl{a}])$}
\RightLabel{WF-Prog}
\UnaryInfC{\wf{P}}
\end{prooftree}
\caption{Type checking}
\label{fig:dcc-wf}
\end{figure}
% end Type Checking figure

%%% Local Variables: 
%%% mode: latex
%%% TeX-master: "../thesis"
%%% End: 
