\subsection{Scala Integration}
To integrate the model of the sequent calculus
into the Scala implementation, we need to translate the
first-order model into the SMT-Lib format.
We also need a procedure for preprocessing,
since we enumerated over variable names and paths
in the refined rules C-Subst and C-Prog.
Entailments can then be checked by calling the SMT solver
with the translated and preprocessed model
and the entailment to check.

\subsubsection{SMTLib Representation}
First we give a SMTLib representation for
the sort, predicate and functions definitions
and the non-preprocessed rules.\\
\\
We translate the sorts defined in \Cref{subfig:axioms-naive-general-predicates}.
The parametrized list datatype used is predefined in Z3.
We define sorts \mIt{CList} and \mIt{CsList},
to increase compatibility with other solvers
and to avoid ambiguities with the usage of \Constrs and \Constrss.
%We write \Constrs for \mIt{CList} TODO: gets too long and sort of repeats the above
%and \Constrss for \mIt{CsList}.
The SMTLib representation is defined in \Cref{fig:smtlib-sorts}.
All datatypes are defined with unique constructor- and selector names.
%
\begin{figure}[h]
\begin{lstlisting}[language=smtlib]
(declare-datatype Path (
  (var (id String))
  (pth (obj Path) (field String))))
(declare-datatype Constraint (
  (path-eq (p-left Path) (p-right Path))
  (instance-of (instance Path) (cls String))
  (instantiated-by (object Path) (clsname String))))
(declare-datatype CList (
  (empty)
  (construct (first Constraint) (rest CList))))
(declare-datatype CsList (
  (nan)
  (cons (hd CList) (tl CsList))))
\end{lstlisting}
\caption{SMTLib Datatype Declarations}
\label{fig:smtlib-sorts}
\end{figure}\\
%
\Cref{fig:smtlib-funs-list} shows a SMTLib translation
of the list functions defined in \Cref{fig:fo-list-funs}.\\
We can define recursive functions in the SMTLib format
with the keyword \smtlib{define-fun-rec}.
The functions in the first-order model used pattern matching
for better readability. While SMTLib supports pattern matching,
it is not fully supported by the latest version of Z3.
We translated the pattern matching using if statements,
where the guard checks for specific constructors of a datatype.\\
This can be observed in the translation of the list concatenating \mIt{conc}.
We used an if statement to check wheter the first argument $l1$
is a list construction with \mIt{\is{construct}(l1)}.
These constructor checking functions get automatically
generated for each constructor of a datatype declaration.
Since the \mIt{CList} datatype has only two constructors,
the else part of the if statement must match the empty list
and we do not need to further check for its constrcutor.

We avoided the introduction of new local variable bindings
through let where the new binding would be used only once.
We directly applied the selectors for those occurences.
%
\begin{figure}[t]
\begin{lstlisting}[language=smtlib]
(define-fun-rec conc ((l1 CList) (l2 CList)) CList
  (ite (is-construct l1)
    (construct (first l1) (conc (rest l1) l2))
     l2))
(define-fun-rec elem ((c Constraint) (cs CList)) Bool
  (ite (is-construct cs)
    (ite (= c (first cs))
       true
      (elem c (rest cs)))
     false))
\end{lstlisting}
\caption{SMTLib List Functions}
\label{fig:smtlib-funs-list}
\end{figure}\\
%
\begin{figure}[h]
\begin{lstlisting}[language=smtlib]
(define-fun-rec generalize-path
  ((p1 Path) (p2 Path) (x String)) Path
  (ite (= p1 p2)
    (var x)
    (ite (is-var p1)
       p1
      (pth (generalize-path (obj p1) p2 x)
           (field p1)))))
(define-fun generalize-constraint
  ((c Constraint) (p Path) (x String)) Constraint
  (ite (is-path-eq c)
    (path-eq (generalize-path (p-left c) p x)
             (generalize-path (p-right c) p x))
    (ite (is-instance-of c)
      (instance-of (generalize-path (instance c) p x) (cls c))
      (instantiated-by (generalize-path (object c) p x)
                       (clsname c)))))
\end{lstlisting}
\caption{SMTLib Representation of Generalization}
\label{fig:smtlib-funs-gen}
\end{figure}\\
%
\begin{figure}[t]
\begin{lstlisting}[language=smtlib]
(define-fun-rec subst-path
  ((p1 Path) (x String) (p2 Path)) Path
  (ite (is-var p1)
    (ite (= x (id p1))
      p2
      p1)
    (pth (subst-path (obj p1) x p2) (field p1))))
(define-fun subst-constraint
  ((c Constraint) (x String) (p Path)) Constraint
  (ite (is-path-eq c)
    (path-eq (subst-path (p-left c) x p)
             (subst-path (p-right c) x p))
    (ite (is-instance-of c)
      (instance-of (subst-path (instance c) x p) (cls c))
      (instantiated-by (subst-path (object c) x p) (clsname c)
     ))))
(define-fun-rec subst-constraints
  ((cs CList) (x String) (p Path)) CList
  (ite (is-construct cs)
    (construct (subst-constraint
                 (first cs) x p)
                 (subst-constraints (rest cs) x p))
     empty))
\end{lstlisting}
\caption{SMTLib Representation of Substitution}
\label{fig:smtlib-funs-subst}
\end{figure}
A SMTLib translation for substitution defined in \Cref{subfig:axioms-naive-general-funs}
is given in \Cref{fig:smtlib-funs-subst}
and the SMTLib representation for generalization
defined in \Cref{fig:axioms-general-gen} can be found in \Cref{fig:smtlib-funs-gen}.\\
The translations for substitution and generalization
follow the same pattern regarding pattern matching
as the translation of the list functions shown previously.
We define non-recursive functions using the keywords \smtlib{define-fun},
so we strictly distinguish syntactically between recursive
and non-recursive functions in the SMTLib format.
%
\begin{figure}[h]
\begin{lstlisting}[language=smtlib]
// Base Properties
(declare-fun class (String) Bool)
// DCC Properties
(declare-fun entails (CList Constraint) Bool)
(define-fun-rec Entails ((cs1 CList) (cs2 CList)) Bool
  (ite (is-construct cs2)
    (and (entails cs1 (first cs2))
         (Entails cs1 (rest cs2)))
     true))
\end{lstlisting}
\caption{SMTLib Predicate Definitions}
\label{fig:smtlib-predicates}
\end{figure}\\
A SMTLib representation of the predicates defined in \Cref{subfig:axioms-naive-general-predicates}
is given in \Cref{fig:smtlib-predicates}.
We can declare undefined functions with the keywords \smtlib{declare-fun}.
Those function declarations are specified,
like the predicates in the first-order model,
through requiring that specific instances of the function hold.
\\
We kept only the predicate \mIt{class}, since the refined rules
C-Subst and C-Prog enumerate over valid variable names and paths,
and the entailment predicates. %\mIt{entails} and \mIt{Entails}.
%The predicate \mIt{Entails} is defined as a recusive function. % TODO: readd these? (removed because too long and ugly in pdf)
\\
%
\begin{figure}[t]
\begin{lstlisting}[language=smtlib]
(assert (!
  (forall ((p Path)) (entails empty (path-eq p p)))
  :named C-Refl))
(assert (!
  (forall ((c Constraint) (cs CList))
    (=> (elem c cs) (entails cs c)))
  :named C-Ident))
(assert (!
  (forall ((cs CList) (p Path) (c String))
    (=> (and (class c)
             (entails cs (instantiated-by p c)))
        (entails cs (instance-of p c))))
  :named C-Class))
(assert (!
  (forall ((cs1 CList) (cs2 CList)
           (b Constraint) (c Constraint))
    (=> (and (entails cs1 c)
             (entails (construct c cs2) b))
        (entails (conc cs1 cs2) b)))
  :named C-Cut))
\end{lstlisting}
\caption{SMTLib $DC_C$ Rules}
\label{fig:smtlib-basic-rules}
\end{figure}
A SMTLib representation for the $DC_C$ rules C-Refl,
C-Ident, C-Class and \mbox{C-Cut}, that do not
require enumeration, is given in \Cref{fig:smtlib-basic-rules}.
The structural rules of the sequent calculus C-Weak and C-Perm
are defined in \Cref{fig:smtlib-structural-rules}.\\
The keyword \smtlib{assert} is used to add an assertion
to the solver and the keyword \smtlib{!} can be used to
annotate a formula.
Annotation is done via adding a key-value pair to a formula,
where the key specifies the name of the annotated property.\\
We \textit{named} the rules according to their names in the calculus.
This enables the solver to produce an unsatisfiable core
for an unsatisfiable entailment,
by returning the names of the violated rules.
%
\begin{figure}[h]
\begin{lstlisting}[language=smtlib]
(assert (!
  (forall ((cs CList) (a Constraint) (b Constraint))
    (=> (entails cs b)
        (entails (construct a cs) b)))
  :named C-Weak))
(assert (!
  (forall ((cs1 CList) (cs2 CList) (c Constraint))
    (=> (and (forall ((a Constraint))
               (and (=> (elem a cs1) (elem a cs2))
                    (=> (not (elem a cs1)) (not (elem a cs2)))
             ))
             (entails cs1 c))
        (entails cs2 c)))
  :named C-Perm))
\end{lstlisting}
\caption{SMTLib Structural Rules}
\label{fig:smtlib-structural-rules}
\end{figure}\\
%
\begin{figure}[t]
\begin{lstlisting}[language=smtlib]
(define-fun-rec big-or-Entails ((ccs CsList) (cs CList)) Bool
  (ite (is-cons ccs)
    (or (Entails cs (hd ccs))
        (big-or-Entails (tl ccs) cs))
     false))
(assert (!
  (forall ((cs CList) (c Constraint))
    (let ((ccs (lookup-program-entailment c)))
      (=> (and (not (= ccs nan))
               (big-or-Entails ccs cs))
          (entails cs c))))
  :named C-Prog))
\end{lstlisting}
\caption{SMTLib C-Prog and Helper Function}
\label{fig:smtlib-cprog}
\end{figure}
%
\Cref{fig:smtlib-cprog} shows a SMTLib representation
of the refined rule C-Prog defined in \Cref{fig:axioms-cprog}.\\
The function \smtlib{big-or-Entails} models
$\bigvee_{\ovl b \in ccs}{\entails{cs}{\ovl b}}$
as used in the refined model of rule C-Prog.
The function \smtlib{lookup-program-entailment} is a lookup
function of the form seen in \Cref{ex:lookup-fun}.

\subsubsection{SMTLib Format in Scala}
%
\lstset{ language=scala,basicstyle=\small }
%
We modeled the SMTLib format in Scala as an AST.
Each node of the AST knows how to format itself
and formatting will be propagated to each sub-node.\\
\Cref{fig:scala-format} shows 
\scala{trait SMTLibFormatter}, as well as
\scala{object SMTLibFormatter}.
The trait provides the function \scala{format}
and each node inherits from this trait.
The singleton \scala{object SMTLibFormatter}
provides a function for formatting a SMTLib formattable
sequence using a specifiable separator.
%
\begin{figure}[h]
\begin{subfigure}[c]{1\textwidth}
\begin{lstlisting}
trait SMTLibFormatter {
  def format(): String
}
\end{lstlisting}
\end{subfigure}
\begin{subfigure}[c]{1\textwidth}
\begin{lstlisting}
object SMTLibFormatter {
  def format(seq: Seq[SMTLibFormatter]
           , separator: String = " "): String =
    seq.foldRight(""){
    (x, xs) => s"${x.format()}$separator$xs"}.dropRight(1)
}
\end{lstlisting}
\end{subfigure}
\caption{SMTLib Formatting in Scala}
\label{fig:scala-format}
\end{figure}\\
%
\Cref{ex:scala-format} shows the implementation
of the universal quantifier.\\
The \scala{trait Term} models the category \textit{term}
of the SMTLib format.
The universal quantifier is a member of this category
and therfore \scala{case class Forall}\ inherits from \scala{trait Term}.\\
The format of an universal quantifier is defined
in \scala{case class Forall} and starts
with the keyword \smtlib{forall}.
The keyword is followed by the format of the quantified variables
enclosed in parens and by the format of the body.
Finally the formatted components are enclosed in parens
to obtain the format of the quantifier.
%
\begin{example}[SMTLib Implementation: Universal quantification]
\label{ex:scala-format}
\begin{lstlisting}
trait Term extends SMTLibFormatter

case class Forall(vars: Seq[SortedVar], body: Term) extends Term {
  override def format(): String =
    s"(forall (${SMTLibFormatter.format(vars)}) ${body.format()})"
}
\end{lstlisting}
\end{example}

\subsubsection{Preprocessing Variable Names and Paths}
The previously shown SMTLib representation is
missing the refined components of the model,
which enumerate over variable names and paths.

\paragraph{Lookup Function generation}
is the process described in \Cref{sec:rule-refinement},
with the goal to generate a function modeling
the existence check of program entailments
combined with the search for a substitution
matching the right-hand side from the entailment
to be shown, to an entailment declaration of the program.
%
\begin{lstlisting}[caption={Lookup Function Generation},label=lst:lookup,captionpos=b,frame={lines}]
def makeProgramEntailmentLookupFunction(
        p: Program, paths: List[Path]): SMTLibCommand = {
  val x = SimpleSymbol("c")
  val body = makeProgramEntailmentLookupFunctionBody(
        paths.flatMap(instantiateProgramEntailments(p, _)), x)
  DefineFun(FunctionDef(
    SimpleSymbol("lookup-program-entailment"),
    Seq(SortedVar(x, SimpleSymbol("Constraint"))),
    SimpleSymbol("CsList"),
    body
  ))
}

private def instantiateProgramEntailments(p: Program, path: Path
  , entailments: Map[Constraint, List[List[Constraint]]] = Map())
              : Map[Constraint, List[List[Constraint]]] = p match {
  case Nil => entailments
  case ConstraintEntailment(x, as, a) :: rst =>
    val cs: List[Constraint] = substitute(x, path, as)
    val c: Constraint = substitute(x, path, a)
    entailments.get(c) match {
      case None => instantiateProgramEntailments(
                     rst, path, entailments + (c -> List(cs)))
      case Some(ccs) => instantiateProgramEntailments(
                     rst, path, entailments + (c -> (cs :: ccs)))
    }
  case _ :: rst => instantiateProgramEntailments(
                     rst, path, entailments)
}

private def makeProgramEntailmentLookupFunctionBody(
   entailments: List[(Constraint, List[List[Constraint]])], x: Term)
                          : Term = entailments match {
  case Nil => SimpleSymbol("nan")
  case (c, ccs) :: rst =>
    Ite(
      Eq(x, convertConstraint(c)),
      Axioms.makeCsList(
        ccs.map(cs => Axioms.makeList(cs.map(convertConstraint)))
      ),
      makeProgramEntailmentLookupFunctionBody(rst, x)
    )
}
\end{lstlisting}
%
\Cref{lst:lookup} shows the Scala implementation
of the lookup function generation.\\
\\
The function \scala{instantiateProgramEntailments}
implements the instantiation of entailment declarations
for a program $p$ with a given path $path$.
The already instantiated entailments are accumulated
in \mIt{entailments}
and returned if no uninstantiated declaration remains.\\
%
We match entailment declarations \progEnt{x}{as}{a}
and substitutes the bound variable $x$ with $path$ in $as$ and $a$.
We declare $cs := \subst{as}{x}{path}$ and $c := \subst{a}{x}{path}$.
We check if we previously instantiated a constraint matching $c$.
If not, we create an entry mapping from $c$ to $cs$.
If we did, we add the constraints $cs$ to the already
accumulated constraints for $c$.
We repeat this until no declaration remains
and return the accumulated mappings.\\
\\
The function \scala{makeProgramEntailmentLookupFunctionBody}
is responsible for creating the body of the lookup function.
It implements the process of generating a series
of if statements out of mappings from $\Constr$ to \Constrss.\\
%
We match $c \mapsto ccs :: rst$
and create an if statement with the guard $x = c$,
where $x$ is the argument of the lookup function.
In the then brach, we put the SMTLib representation of $ccs$.
The else branch then constains a recursive call with $rst$.
We repeat this until all mappings are processed.\\
\\
The function \scala{makeProgamEntailmentLookupFunction}
generates a lookup function for entailment declarations in programs.
It takes a program and a list of paths as argument
and instantiates each path with each
entailment declaration in the program.
The function returns a SMTLib representation usable by the solver.\\
%
We set $x$ to be the argument of the function to be created.
We map function \scala{instantiateProgramEntailments}
over the provided paths and obtain a list of mappings $m$.
We pass $m$ into \scala{makeProgramEntailmentLookupFunctionBody}
to obtain a series of if statements $body$ as a SMTLib representation.
We then create the lookup function with the signature
$\Constr \rightarrow \Constrss$ and the body $body$.

\newpage
\paragraph{C-Subst preprocessing} is the process of enumeration,
to ground the quantification over variable names and paths
in rule C-Subst as shown in \Cref{sec:rule-refinement}.
An implementation of the C-Subst template from \Cref{fig:axioms-csubst}
incorporating the optimizations demonstrated with \Cref{ex:subst-optimzed}
is given in \Cref{lst:subst-template}.
The process of generating instances of this template
is implemented in \Cref{lst:subst}.
%
\begin{lstlisting}[caption={Rule Generation for C-Subst},label=lst:subst,captionpos=b,frame={lines}]
def generateSubstRules(
    vars: List[Id], paths: List[Path]): Seq[SMTLibCommand] = {
  var rules: Seq[SMTLibCommand] = Seq()
  val pathPairs = makePathPairs(paths)

  vars.foreach(x => pathPairs.foreach{
    case (p, q) if x == p && p == q => () // skip
    case (p, q) => rules = rules
                         :+ instantiateSubstRule(x, p, q)
  })
  rules
}

private def instantiateSubstRule(
    variable: Id, p: Path, q: Path): SMTLibCommand =
  Assert(
    Annotate(
      substRuleTemplate(variable, p, q),
      Seq(KeyValueAttribute(Keyword("named")
        , SimpleSymbol("C-Subst")))))
\end{lstlisting}
%
The function \scala{generateSubstRules} takes
a list of variables and a list of paths as argument
and returns a sequence of C-Subst rules.
In the function, we iterate over all variables $x$
and for each $x$ we iterate over all paths $p$ and $q$.
We call \scala{instantiateSubstRule} with $x$, $p$ and $q$
to generate a rule C-Subst and append it to the sequence
of already generated rules.
We skip cases where $x = p = q$.
Finally, we return the collected rules.\\
\\
The function \scala{instantiateSubstRule}
creates the surrounding frame of rule C-Subst
and calls \scala{substRuleTemplate} to
instantiate the template of rule C-Subst.\\
\\
The function \scala{substRuleTemplate}
takes a variable and two paths as argument
and returns a SMTLib representation
of rule C-Subst.
In the function, we convert the given variable
and paths into their counterpart $x$, $p$ and $q$
in the SMTLib format.
The function defines three subroutines.
These subroutines implement the optimizations
discussed in \Cref{sec:rule-refinement}.\\
We check in the subroutine \scala{gen}
if variable $x$ equals path $p$.
If not, we generalize constraint $a2$
and call subroutine \scala{subst}
with the generalized constraint.
Otherwise we skip the generalization and proceed
with \scala{subst} directly.\\
Subroutine \scala{subst} takes a term $a$
representing a constraint as argument.
We check if variable $x$ equals path $q$.
If not, we substitute $a$ and call
subroutine \scala{conjunction}
on the result of the substitution.
Otherwise we call \scala{conjunction},
without substituting the argument.\\
%
\begin{lstlisting}[caption={C-Subst Template},label=lst:subst-template,captionpos=b,frame={lines}]
private def substRuleTemplate(
    variable: Id, p1: Path, p2: Path): Term = {
  val x: Term = convertId(variable)
  val p: Term = convertPath(p1)
  val q: Term = convertPath(p2)

  def conjunction(a1: Term) =
    if (p1 == p2)
      Apply(SimpleSymbol("entails"),
        Seq(SimpleSymbol("cs"), a1))
    else
      And(
        Apply(SimpleSymbol("entails"), Seq(SimpleSymbol("cs"),
          Apply(SimpleSymbol("path-eq"), Seq(p, q)))),
        Apply(SimpleSymbol("entails"),
          Seq(SimpleSymbol("cs"), a1))
      )
  def subst(a: Term) =
    if(variable == p2)
      conjunction(a)
    else
      Let(
        Seq(VarBinding(SimpleSymbol("a1"),
          Apply(SimpleSymbol("subst-constraint"), Seq(a, x, q))
        )),
        conjunction(SimpleSymbol("a1"))
      )
  val gen =
    if (variable == p1)
      subst(SimpleSymbol("a2"))
    else
      Let(
        Seq(VarBinding(SimpleSymbol("a"),
          Apply(SimpleSymbol("generalize-constraint"),
            Seq(SimpleSymbol("a2"), p, x))
        )),
        subst(SimpleSymbol("a"))
      )

  Forall(
    Seq(
      SortedVar(SimpleSymbol("a2"), SimpleSymbol("Constraint")),
      SortedVar(SimpleSymbol("cs"), SimpleSymbol("CList"))
    ),
    Implies(
      gen,
      Apply(SimpleSymbol("entails"),
        Seq(SimpleSymbol("cs"), SimpleSymbol("a2")))
    )
  )
}
\end{lstlisting}
Subroutine \scala{conjunction} takes
a term $a1$ representing a constraint as argument.
We check if path $p$ equals path $q$.
If not, we build the conjunction
of the entailments \entails{cs}{\pathEq{p}{q}}
and \entails{cs}{a1}.
Otherwise we build the entailment \entails{cs}{a1}
and skip the check for equivalent paths.\\
In \scala{substRuleTemplate},
we create the SMTLib representation of
\[ \forall a_2: \Constr, cs: \Constr.\ \mathbf{gen} \rightarrow \entails{cs}{a_2} \]
where $\mathbf{gen}$ is a call to subroutine \scala{gen}
as our return value.

\subsubsection{Calling the Solver}
With the SMTLib representation and preprocessing
of the model defined, we can call the solver
to evaluate if entailments \entails{\ovl a}{a}
hold in the model.
For this we define \scala{trait SMTSolver}
in \Cref{lst:smtsolver} as an interface for SMT solvers
and function \scala{entails} to check entailments
in \Cref{lst:entails}.
%
\begin{lstlisting}[caption={SMTSolver Interface},label=lst:smtsolver,captionpos=b,frame={lines}]
trait SMTSolver {
  val axioms: SMTLibScript

  def addCommand(command: SMTLibCommand): Boolean
  def addCommands(commands: Seq[SMTLibCommand]): Boolean
  def addScript(script: SMTLibScript): Boolean
  def flush(): Unit

  /**
    * Executes the SMTSolver with the currently held commands.
    * @param timeout The timeout for each query
             to the SMTSolver in milliseconds.
    * @return The return code of the SMTSolver
                or -1 in case of a timeout
    *         and the raw output of the solver.
    */
  def execute(timeout: Int = 1000): (Int, Seq[String])

  /**
    * Executes the SMTSolver with the currently held commands
    * and checks them for satisfiability.
    * @param timeout The timeout for each query
             to the SMTSolver in milliseconds.
    * @return `Sat` if the input is satisfiable
    *        `Unsat` if the infput is unsatisfiable
    *        `Unknown` if the solver can't decide.
    */
  def checksat(timeout: Int = 1000): CheckSatResponse
}
\end{lstlisting}
%
The trait \scala{SMTSolver} defines
functionality for calling a SMT solver.
The trait defines a set of \scala{axioms}
which are true for each query to the solver,
as well as functions \scala{addCommand},
\scala{addCommands} and \scala{addScript}
for adding additional commands to be passed to the solver.
The function \scala{flush} removes all added commands
and keeps only the axioms.\\
Function \scala{execute} calls
and uses the \scala{axioms} as well as
all added commands as input for the solver.
The function returns the exit code
and the raw output of the solver.\\
Function \scala{checksat} does call
the solver like \scala{execute}, with the
additional smtlib input \smtlib{(check-sat)}.
It then parses the output of the solver
and returns either
\smtlib{sat} if the input was satisfiable,
\smtlib{unsat} if the input was unsatisfiable or
\smtlib{unknown} if the solver could not decide
whether the input was satisfiable or unsatisfiable.
The result \smtlib{unknown} does not imply
that the input formula is undecidable.
There can be other reasons resulting
in the output \smtlib{unknown}, as timeouts.
%
\begin{lstlisting}[caption={Entailment Function},label=lst:entails,captionpos=b,frame={lines}]
  def entails(
      context: List[Constraint], c: Constraint): Boolean = {
      val (vars, paths, classes) = [...]

      // SMTLib representation
      val entailment =
          SMTLibConverter.convertEntailment(ctx, c)
      val classPredicates = classes.map(
          cls => Apply(SimpleSymbol("class"),
                       Seq(SMTLibString(cls))))

      val solver: SMTSolver = new Z3Solver(Axioms.all)
      solver.addCommands(
        SMTLibConverter.makeAsserts(classPredicates))
      solver.addCommands(
        SMTLibConverter.generateSubstRules(vars, paths))
      solver.addCommand(SMTLibConverter
          .makeProgramEntailmentLookupFunction(P, paths))
      solver.addCommand(Axioms.cProg)
      solver.addCommand(Assert(Not(entailment)))

      solver.checksat() match {
        case Sat => false
        case Unsat => true
        case Unknown => false
      }
    }
  }

  def entails(
      ctx: List[Constraint], cs: List[Constraint]): Boolean =
    cs.forall(c => entails(ctx, c))
\end{lstlisting}
%
The function \scala{entails} can be used
to check constraint entailments.
It has two implementations.
The first implementation models \entails{context}{c}
for constraint lists $context$ and constraints $c$.
The second implementation models \entails{ctx}{cs}
for constraint lists $ctx$ and $cs$.
It requires that \scala{entails(ctx, c)} holds for
all $c \in cs$.\\
\\
The function is responsible for
invoking the preprocessing process.
For this, we implemented the function as follows:
We first extract the variable names, paths
and class names from the function arguments.
We then convert the arguments and create
a SMTLib representation \entails{context}{c}
as well as generating $\mIt{class}(cls)$
for each class name $cls \in \mIt{classes}$.

We create an instance of \scala{SMTSolver}
with \scala{Axioms.all}, where \scala{Axioms.all}
contains the SMTLib representations of all
declarations, definitions and rules.
The axioms exclude rule C-Subst, rule C-Prog
and a definition of a lookup function.
We add the generated class predicates to the solver.
Afterwards we call \scala{generateSubstRules}
to enumerate over variable names and paths to
generate rules C-Subst
and \scala{makeProgramEntailmentLookupFunction}
to generate a lookup function from the program and paths.
We add the generated C-Subst rules as well,
as the lookup function and rule C-Prog to the solver.
Finally, we add the negation of the generated
entailment \entails{context}{c} to the solver.

We call \scala{solver.checksat} to check
for satisfiability with the SMT solver.
We return true if the call returns \smtlib{unsat},
because the entailment \entails{context}{c}
is valid if its negation is unsatisfiable.

%%% Local Variables: 
%%% mode: latex
%%% TeX-master: "../thesis"
%%% End: 
