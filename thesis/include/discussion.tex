\chapter{Discussion}
\label{chp:discuss}
In this chapter we conclude the thesis.
We discuss the contributions of the thesis,
as well as possible future work.

\section{Constraint System}
We implemented the constraint system of the $DC_C$ calculus
in \Cref{sec:constraintsystem}.
The $DC_C$ constraint system is given as a sequent calculus.
We defined a model of the constraint system in first-order logic.
Our aim for the model was to be structurally similar to the sequent calculus,
as we want to use the same structural reasoning
as in the rules of the sequent calculus.
Since we employ a SMT solver for resolution of the constraint system,
we defined a representation of the first-order model in the SMTLib format.\\
\\
In \Cref{sec:rule-refinement} we refined the first-order model,
by improving the usability of some rules with the SMT solver.
We enumerated over variables and paths in rule C-Subst
to reduce the amount of quantified variables.
This reduction in quantified variables through enumeration came
with the disadvantage of creating additional rules,
where we need to create one rule for each variable and path combination.\\
\\
We expressed declarative elements in rules C-Subst and C-Prog in an algorithmic way.

For rule C-Subst we introduced generalization as the inverse function of substitution.
This allowed us to express the relation from the input $a_2$ to $a$ via generalization
and from $a$ to $a_1$ via substitution.

For rule C-Prog we developed a process to convert the existence check for
entailment declarations in the program to a lookup function.
This allowed us to call the lookup function in rule C-Prog with the input $a$.\\
\\
We observed that the SMT solver can loop and time out,
if we supply unnecessary rules to find a derivation
for the given input.
We introduced pruning in \Cref{sec:pruning} to
remove rules from the model.
In the generation process for rules C-Subst with enumeration,
we multiplied the amount of available rules.
We plugged the pruning process into this process
to skip the generation of some rules.
Since the removal of valid rules can lead to incompleteness
we used a two-pass call to the SMT solver,
one with enabled pruning and one with disabled pruning.\\
\\
Despite these improvements to the model,
the SMT solver still does time out
for some valid inputs.
%
%- we did (for constraint system)
%  - a model of the DCC constraint system in first-order logic
%  - the model is createt to be strucurally similar to the sequent calculus
%  - usage of SMT solver to solve the constraint system
%  - we had a bunch of problems with the SMT solver (cant solve shit, times out for everything)
%  - we already implemented/incorporated some improvements to the model
%    - grounding of quantified variables
%    - express declarative rules in an algorithmic way (at least more than previously)
%      - lookup function for prog
%      - generalization for subst
%  - a big problem of the solver is the presence of unnecessary rules
%    - e.g. remove subst if x = p = q, showed that their is no information gain in thos rules
%    - we introduced pruning to remove some of the obivous culprits
%    - removing subst if p = q

\section{Interpreter}
We implemented the operational semantics with an interpreter in \Cref{sec:interp}.
The implementation of the interpreter is straightforward.
The rules of the operational semantics are syntax oriented
and can be "read from bottom to top" to yield
an algorithm~\cite{tpl}.\\
\\
The performance of the interpreter is bound
by the performance of the SMT solver.
This includes the time needed to evaluate an expression to a value
as well as the question if we can fully evaluate an expression,
given that the expression is reducible to a value.
%
%- for operational semantics
%  - implemented interpreter
%  - implementation of semantic rules were straightforwards
%  - all rules syntax oriented
%  - all information specified in the rules
%  - could be read in a directed way and implemented from top to bottom (or is it reverse? check pierce book)
%  - performance of interpreter is bound by the performance of the implementation of the constraint system / solver

\section{Type Checker}
We implemented the type relation of the $DC_C$ calculus
in \Cref{sec:types}.\\
\\
The definition of type assignments for expressions
in the $DC_C$ calculus is declarative.
The rules are not syntax oriented,
as rule T-Sub can be applied to any expression.
Type assignment is also not unique.\\
\\
The implementation of the type assignment relation
is not as straightforward as the implementation
of the operational semantics.

For the implementation of the subsumption rule T-Sub,
we integrated subtyping checks into the remaining type rules.
the inheritance relation is expressed via the constraint system
and we can therefore use the SMT solver to check for subtyping.

Since type assignments are not unique,
our implementation returns a list of types for
an expression.\\
\\
The rules for the well-formedness of programs are
again syntax oriented
and the rules form a structure for the implementation.
The rules of well-formedness check mostly the free variables
of declarations.
The outlier being the rule WF-Mi for method implementations.
The rule requires that the return type of a method
equals the type of the method body.
For this, the rule uses type assignment on the body
and compares the result to the annotated return type.
Since the type assignment implementation
returns a list of types instead of one type,
our implementation requires that the annotated return type
is contained in the types returned by the type assignment.\\
\\
The performance of the type checker is,
as the performance of the interpreter,
bound by the performance of the SMT solver.
%
%- for type relation / type checking
%  - implemented type assignment
%    - type relation / rules are quite declarative
%    - rules are also not syntax oriented
%      - T-Sub matched on all expressions
%    - type assignment is also not unique
%  - implementation needed to deal with this
%    - for subsumption we integrated subtyping checks into the other rules
%      - subtyping check via constraint system
%    - for non unique
%      - find a list of types instead only one type
%      - changed return type
%  - implemented well formedness
%    - mostly straightforward
%    - all syntax oriented
%    - most checks easy, since requirements mostly free vars
%    - method implementations require check if annotated return type matches the type of the body
%      - since our type assignment implementation returns list of types
%        we needed to change this check
%      - original: checking for equality
%      - our version: checking if return type is one of the possible types of the body
%  - as interpreter, performance bound by constraint system / solver

\section{Information Gain From Type Checker}
The SMT solver plays a role in the time it takes
to interpret an expression.
The time it takes the SMT solver to find a
solution for a solvable input, depends on the amount
of rule instantiations needed to close the derivation.

We proposed \textit{context enrichment} in \Cref{chp:learning},
as a process to reduce the number of rule applications needed
to solve a constraint entailment.
It is possible to reduce the amount of steps it takes
to solve a constraint entailment \mIt{ce}
by adding constraints $t$ to the context of \mIt{ce}.
Adding constraints to the context
can result in the solver being able to find a solution for
a "invalid" constraint entailment.
Thus, we need to make sure that the added constraints
do not contradict the information available in the context.

In the $DC_C$ calculus, a type of an expression $e$ is a list of constraints that
hold for each possible value of that expression during runtime.
So it is safe to add $t$ to the context of \mIt{ce},
if \mIt{ce} originated from the interpretation of $e$.
%
%- learning / transferring information from compile to runtime
%  - as performance is bound by the performance of the constraint system/solver
%  - runtime overhead through solver is unwanted
%  - find a way to decrease overhead
%  - we proposed: context enrichment / augmentation
%  - type information is valid for all expressions
%  - add type constraints to the contexts of entailments
%  - leading to new possibilities for the solver to find a solution
%  - need to be careful that added constraints are not contradicting

\section{Future Work}
The used SMT solver, as already stated, is unable to successfully
show all solvable constraint entailments.
There are multiple possibilities to increase the performance of the solver.
\begin{itemize}
  \item Exploring if Cut-elimination~\cite{cutelim} can be applied to the defined sequent calculus
        of the $DC_C$ calculus.
  \item Further optimizations to the presented first-order model.
  \begin{itemize}
    \item Applying the enumeration process used for rule C-Subst
          onto more variables and rules.
    \item There are still declarative rules in the model.
          Searching for patterns that can be expressed in an algorithmic
          way in those rules can help the solver to successfully instantiate those rules.
    \item The presence of unnecessary rules is a problem for the SMT solver.
          Developing a heuristic to determine which rules will not be used,
          would prevent the solver from looping in such a rule and time out less as a result.
    \item SMT solvers in general and Z3 in particular are highly configurable.
          The amount of different available options and their possible side-effects on each other
          make it difficult to find the best configuration for the problem under consideration.
  \end{itemize}
  \item Developing a new model of the constraint system.
        It might yield better result in combination with a SMT solver,
        if a model does not mimic the structural reasoning of the sequent calculus.
  \item There is a wide range of SMT solvers available,
        all with unique strengths and weaknesses.
        Other SMT solvers than Z3 could be more suitable for our model of the constraint system.
  \item There are automated theorem provers available apart from SMT solvers.
        As with the previous point about alternative SMT solvers,
        one could explore the capabilities of different proving systems
        on the constraint system.
\end{itemize}

%  - besides all/yet the solver does still time out regulary
%  - possible ways of action
%    - further optimize the presented/current model
%      - there are still variables to be grounded
%      - further rules could be expressed in an algorithmic way
%      - find a good heuristic (or other approach) to determine unused rules to remove
%      - explore properties of the SMT solver to yield better results
%      - explore if cut-elimination can be applied to DCC
%    - build a new model
%      - not tying the model to the structure of the sequent calculus might yield better results
%        because the solver might be able to do more logic stuff directly if the model is not structural
%    - explore the usage of other SMT solvers with the current model
%      - other solvers might yield better results as Z3,
%        or worse, or for some problems better for some worse
%    - explore the usage of completely different automatic proof systems
%      - datalog style or such?

%%% Local Variables: 
%%% mode: latex
%%% TeX-master: "../thesis"
%%% End: 
