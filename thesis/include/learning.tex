\chapter{Improving Interpreter with Type Information}
In the $DC_C$ calculus we need to solve a constraint system
when interpreting an expression, as well when type checking a program.
The need to solve a constraint system during runtime produces unwanted overhead,
as solving the constraint system takes some time.

For example, we interpreted a method call in \Cref{ex:eval-call}.
During evaluation, we needed to find the most specific method
that was applicable to the provided parameter.
For this we needed to solve a constraint entailment
for each method implementation.
Another example is the type checking of the natural numbers program
as seen in \Cref{ex:wf-naturalnumbers}.
There, we again needed to call the SMT solver.

In both examples, we needed to solve constraint entailments during execution.
There are two cases to consider: The solver
\begin{enumerate}
  \item does find a derivation.
  \item does not find a derivation.
\end{enumerate}
For (1):
If solver finds a derivation,
the constraint entailment holds.
The amount of time it takes the solver to find this derivation
is between $0$ and \mIt{limit},
where \mIt{limit} is the maximum amount of time before the solver returns \smtlib{unknown}.\\
\\
For (2):
If the solver finds a contradiction of the constraint entailment to the rules,
the execution time of the solver is between $0$ and \mIt{limit}.
If the SMT solver does not a contradiction for the constraint entailment,
the time it takes for the solver to return is \mIt{limit},
as in those cases the solver needs to evaluate all possible rule instantiations
to find either a derivation or a contradiction.
In both examples, we try to solve constraint entailments that have no derivation.\\
\\
While the provided implementation of the $DC_C$ calculus
can surely be further optimized without
touching the model of the constraint system,
this is an inherent problem of the $DC_C$ calculus.

\section{Formulating An Optimization Goal} % TODO: rename?
\label{sec:goal}
We have seen that solving a constraint system,
even in the best case scenario,
adds overhead to the execution time.
Therefore a goal for optimization could be
to remove the need to solve a constraint system,
although the complete elimination of the constraint system
might be too optimistic.

An alternative goal
is to minimize the time needed for the solver to find a solution.
The time it takes the solver to find a derivation for a solvable
constraint entailment, depends on the size of the search space.
The size of the search space is affected by
\begin{itemize}
  \item the amount of variables in quantifiers.
        Each additional quantified variables leads to an exponential growth in possible instantiations.
  \item the amount of rules supplied to the solver.
        Each additional rule, leads to new possible rule applications.
  \item the amount of constraints in the context of a constraint entailment.
        Each additional constraint, adds a restriction to the constraint entailment.
\end{itemize}
In \Cref{sec:rule-refinement},
we explored the reduction of quantified variables.
For rule C-Subst, we removed quantification
for variables and paths through grounding.
Additionally, we introduced generalization
to find variables $a$ and $a_1$ algorithmically.
In the process to ground the quantified variables
we introduced new rules to the solver.
We reduced the amount of possible instantiations of the rule,
but at the same time added new rules.\\
\\
In \Cref{sec:pruning} we explored the removal of rules,
through the introduction of pruning.
Pruning removed valid rules,
in order to solve constraint entailments
where the solver would otherwise
instantiate the same rule in a loop.
We introduced pruning
as an additional call to the solver
to preserve completeness.
We reduced the amount of rules,
but at the same time added
the need to solve the same constraint entailment twice,
if the first try times out.\\
\\
Both, grounding and pruning, had positive effects
as well as negative effects on the search space.
Technically pruning did not increase the search space,
but it has nonetheless negative effects on the execution time
for a constraint entailment.
Only the introduction of generalization removed quantified variables
without introducing new rules or other means to increase execution time.\\
\\
For a constraint entailment \mIt{ce := \entails{\ovl a}{a}},
the addition of new constraints to \ovl a
does not require the introduction of
additional rules or quantified variables into the model.
It only affects the possible derivations of \mIt{ce}.
It specifies information about the constraint entailment
that was previously unknown.
\newpage

\section{Reducing Evaluation Time} % TODO: other title? maybe: Context Enrichment
In this section we want to explore,
how we can enrich the context \ovl a
of a constraint entailment \mIt{ce := \entails{\ovl a}{a}}
with new information to reduce
the time it takes the solver to find a derivation for \mIt{ce}.

To add new information to the context of \mIt{ce},
we can concatenate \ovl a with new constraints \ovl b
and solve \entails{\ovl a \conc \ovl b}{a}.
The SMT solver can, after the addition of \ovl b,
make new rule instantiations that
can lead to new derivations for \mIt{ce}.\\
\\
The addition of new constraints \ovl b to the context
can be dangerous.
A misuse is possible, where the added constraints
result in the SMT solver being able to
prove an otherwise unsolvable constraint entailment.
%\Cref{ex:enrich-context-misuse} shows a faulty
%context enrichment.

\begin{example}[Faulty derivation]
\label{ex:enrich-context-misuse}\quad\\
The constraint entailment \entails{\instOf{x}{C}}{\instOf{y}{C}}
is not solvable,
as we have no way to derive the class of $y$ from the class of $x$.

We enrich the context with \pathEq{x}{y} and obtain
\entails{\instOf{x}{C}, \pathEq{x}{y}}{\instOf{y}{C}}.
The enriched constraint entailment is solvable,
because we added the information that $x$ and $y$ are equivalent.
\end{example}

\begin{example}[Contradicting Information]
\label{ex:enrich-context-contra}\quad\\
\[ \entails{ \instOf{x}{\texttt{Succ}}, \instOf{x.p}{\texttt{Nat}} }{c} \]
Taking the natural numbers program as a base,
the context of the constraint entailment specifies $x$ to be
an instance of class \texttt{Succ}.
        
\[ \entails{ \instOf{x}{\texttt{Succ}}, \instOf{x.p}{\texttt{Nat}}, \instOf{x}{\texttt{Zero}} }{c} \]
We added the information \instOf{x}{\texttt{Zero}} to the context.
This contradicts the inheritance relation of the program,
as a class can not be an instance of \texttt{Zero} and \texttt{Succ} simultaneously.
\end{example}
\quad\\
%
In \Cref{ex:enrich-context-misuse}
we created \pathEq{x}{y} out of thin air
and added it to the context of the constraint entailment.
This enabled the solver to find a derivation and to solve
the entailment.
In \Cref{ex:enrich-context-contra}
we added contradicting information to the context.

Hence, we need to make sure that the information
we want to add is true for the constraint entailment
we want to show.


\quad\\\\\\\\\\
%- in the $DC_C$ calculus it is needed to solve a constraint system during runtime
%- this produces overhead not wanted during runtime 
%  - as solving can take a while
%    - up to multiple minutes etc
%    - eg typechecking the natural numbers program already takes ~40 sec in the provided implementation
%- would be cool if the need for solving a constraint system could be removed (benificial?)
%- removal is too optimistic (not doable)
%- alternative goal is to minimize the time needed to solve the constraint system
%- time needed for solving depends on the size of the search space
%- the search space scales with
%  - the provided axioms
%    - amount of variables in a quantifier
%    - amount of rules
%  - the context of an entailment: context |- c
%    - we can add restrictions by adding constraints into the context
%- grounding variables introduces additional rules, as seen in the C-Subst generation
%- removal of rules, makes the model uncomplete. as seen in the pruning process
%- adding additional constraints to the context doesnt alter any rules
- need to make sure that the added constraints do not contradict the context
  - a contradiction in the context means that we could show anything and invalidating soundness (perhaps?)
- how to find constraints that do not contradict?
- types are constraints with a bound variable
- if t := (x, a) is type of e
  - then all constraints a hold for each possible instance of e
- idea: add type constraints into the context when interpreting e
- need to find substitution, since t has bound variable
  - substitution must unify x with the variable e reduces to in a

% TODO: for discussion?  
- can not remove all vars or rules
- the quantified variables are used to match the entailment as an input
- we need the amount of rules, to be able to show entailments

%%% Local Variables: 
%%% mode: latex
%%% TeX-master: "../thesis"
%%% End: 
