%% macros
%%%%%%%%%
\newcommand{\loremipsum}{\todo[inline]{Lorem Ipsum}}
%\newcommand{\nat}[1]{\ensuremath{\lceil\text{#1}\rceil}}
\newcommand{\gprod}[1]{\ensuremath{\langle\text{#1}\rangle}}
\newcommand{\TNum}{\ensuremath{\mathcal{N}}}
\newcommand{\TBool}{\ensuremath{\mathcal{B}}}
\newcommand{\TInt}{\ensuremath{\mathcal{N}}}

\theoremstyle{definition}
\newtheorem{example}{Example}[chapter]
\newtheorem{definition}[example]{Definition}
%\newtheorem{definition}{Definition}[chapter]

% DCC constraints
\newcommand{\pathEq}[2]{\ensuremath{#1 \equiv #2}}
\newcommand{\instanceOf}[2]{\ensuremath{#1 :: #2}}
\newcommand{\instOf}[2]{\instanceOf{#1}{#2}}
\newcommand{\instantiatedBy}[2]{\ensuremath{#1.\textbf{cls} \equiv #2}}
\newcommand{\instBy}[2]{\instantiatedBy{#1}{#2}}
\newcommand{\entails}[2]{\ensuremath{#1 \vdash #2}}
\newcommand*{\ldblbrace}{\{\mskip-6.9mu\{} % left double brace
\newcommand*{\rdblbrace}{\}\mskip-6.9mu\}} % right double brace
\newcommand{\sub}[2]{\ensuremath{\ldblbrace #1 \mapsto #2 \rdblbrace}}

%% misc
%%%%%%%%%
\DeclareMathOperator{\pto}{\rightharpoonup}     % partial function arrow
\DeclareMathOperator{\powerset}{\mathcal{P}}    % powerset
\DeclareMathOperator{\proj}{\upharpoonright}    % projection
\DeclareMathOperator{\fst}{\pi_1}               % A x B → A
\DeclareMathOperator{\snd}{\pi_2}               % A x B → B

%%% Local Variables: 
%%% mode: latex
%%% TeX-master: "../thesis"
%%% End: 
