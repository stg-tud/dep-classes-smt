%% macros
%%%%%%%%%
\newcommand{\loremipsum}{\todo[inline]{Lorem Ipsum}}
%\newcommand{\nat}[1]{\ensuremath{\lceil\text{#1}\rceil}}
\newcommand{\gprod}[1]{\ensuremath{\langle\text{#1}\rangle}}
\newcommand{\TNum}{\ensuremath{\mathcal{N}}}
\newcommand{\TBool}{\ensuremath{\mathcal{B}}}
\newcommand{\TInt}{\ensuremath{\mathcal{N}}}

\theoremstyle{definition}
\newtheorem{example}{Example}[chapter]
\newtheorem{definition}[example]{Definition}
%\newtheorem{definition}{Definition}[chapter]

% DCC
% DCC constraints
\newcommand{\pathEq}[2]{\ensuremath{#1 \equiv #2}}
\newcommand{\instanceOf}[2]{\ensuremath{#1 :: #2}}
\newcommand{\instOf}[2]{\instanceOf{#1}{#2}}
\newcommand{\instantiatedBy}[2]{\ensuremath{#1.\textbf{cls} \equiv #2}}
\newcommand{\instBy}[2]{\instantiatedBy{#1}{#2}}
\newcommand{\entails}[2]{\ensuremath{#1 \vdash #2}}
\newcommand*{\ldblbrace}{\{\mskip-6.9mu\{} % left double brace
\newcommand*{\rdblbrace}{\}\mskip-6.9mu\}} % right double brace
\newcommand{\sub}[2]{\ensuremath{\ldblbrace #1 \mapsto #2 \rdblbrace}}

% DCC declarations
\newcommand{\constructorDeclaration}[3]{\ensuremath{#1(#2.\ #3)}}
\newcommand{\constr}[3]{\constructorDeclaration{#1}{#2}{#3}}
\newcommand{\programEntailment}[3]{\ensuremath{ \forall #1.\ #2 \Rightarrow #3}}
\newcommand{\progEnt}[3]{\programEntailment{#1}{#2}{#3}}
\newcommand{\abstractMethodDeclaration}[4]{\ensuremath{#1(#2.\ #3): #4}}
\newcommand{\mDecl}[4]{\abstractMethodDeclaration{#1}{#2}{#3}{#4}}
\newcommand{\methodImplementation}[5]{\ensuremath{#1(#2.\ #3): #4 := #5}}
\newcommand{\mImpl}[5]{\methodImplementation{#1}{#2}{#3}{#4}{#5}}

% DCC expressions
\newcommand{\newInstance}[3]{\ensuremath{\textbf{new } #1(#2 \equiv #3)}}
\newcommand{\newInst}[3]{\newInstance{#1}{#2}{#3}}

% DCC misc
\newcommand{\obj}[3]{\ensuremath{\langle #1; #2 \equiv #3 \rangle}}
\newcommand*{\stdobj}{\ensuremath{\obj{C}{\overline{f}}{\overline{x}}}}
\newcommand{\heap}[2]{\ensuremath{#1 \mapsto #2}}
\newcommand*{\stdheap}{\ensuremath{\heap{\overline{x}}{\overline{o}}}}
\newcommand{\pair}[2]{\ensuremath{\langle #1;#2 \rangle}}
\newcommand{\eval}[4]{\ensuremath{\pair{#1}{#2} \rightarrow \pair{#3}{#4}}}

%% misc
%%%%%%%%%
\DeclareMathOperator{\pto}{\rightharpoonup}     % partial function arrow
\DeclareMathOperator{\powerset}{\mathcal{P}}    % powerset
\DeclareMathOperator{\proj}{\upharpoonright}    % projection
\DeclareMathOperator{\fst}{\pi_1}               % A x B → A
\DeclareMathOperator{\snd}{\pi_2}               % A x B → B

%%% Local Variables: 
%%% mode: latex
%%% TeX-master: "../thesis"
%%% End: 
