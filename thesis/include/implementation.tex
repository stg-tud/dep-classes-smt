\chapter{$DC_C$ Implementation}
- implementation in scala
- general structure of impl
  - constraint system solved with SMT solver
  - SMT solver input is SMTLib format
  - "connection" to smt solver simple external process call

\section{Constraint System}
- constraint system in DCc given as sequent calculus
- in order to use SMT on constraint system
  → transform sequent calculus into first order axioms
  → use SMTlib representation of these axioms as input to solver
\subsection{Na\"ive approach}
- naive approach
- goal is to be as close to the calculus rules as possible
- "preserve" structure of calculus rules

- make figure of needed functions, predicates, sorts
- make figure of rules
- explain rules
- list problems of these rules
  - too complex (e.g. quantified variables, no direct way of deduction in subst rule)
  - not "structured" (permutation)

\begin{figure}[t]
% Sorts
\begin{subfigure}[c]{1\textwidth}
\centering
\begin{subfigure}[c]{0.45\textwidth}
% BNF
\begin{align*}
\mIt{Path} &::=
     \mIt{String}\\
  &\quad|\ \mIt{Path.String}
\end{align*}
\end{subfigure}
\begin{subfigure}[c]{0.45\textwidth}
% BNF
\begin{align*}
\mIt{Constraint} &::=
     \pathEq{Path}{Path}\\
  &\quad|\ \instOf{Path}{String}\\
  &\quad|\ \instBy{Path}{String}
\end{align*}
\end{subfigure}
% Set
%\begin{align*}
%\mIt{Path} := \{\\
%  &\mIt{var} &&\text{where } \mIt{var}\in \mIt{String}\\
%  &\mIt{obj.field} &&\text{where } \mIt{obj} \in \mIt{Path}, \mIt{field} \in \mIt{String}\\ \}\\
%\mIt{Constraint} := \{\\
%  &\pathEq{\pleft}{\pright} &&\text{where } \pleft,\pright \in \mIt{Path}\\
%  &\instOf{instance}{cls} &&\text{where } \mIt{instance} \in \mIt{Path}, \mIt{cls} \in \mIt{String}\\
%  &\instBy{object}{clsname} &&\text{where } \mIt{object} \in \mIt{Path}, \mIt{clsname} \in \mIt{String}\\ \}
%\end{align*}
%
%\[
%\begin{aligned}
%\mIt{Path} :=\\
%  &x &&\text{where } x\in \mIt{String}\\
%  &obj.field &&\text{where } \mIt{obj} \in \mIt{Path}, \mIt{field} \in \mIt{String}\\
%\mIt{Constraint} :=\\
%  &\pathEq{p}{q} &&\text{where } p,q \in \mIt{Path}\\
%  &\instOf{instance}{cls} &&\text{where } \mIt{instance} \in \mIt{Path}, \mIt{cls} \in \mIt{String}\\
%  &\instBy{object}{clsname} &&\text{where } \mIt{object} \in \mIt{Path}, \mIt{clsname} \in \mIt{String}
%\end{aligned}
%\]
\subcaption{Sorts}
\label{subfig:axioms-naive-general-sorts}
\end{subfigure}\\
\hrule
% Predicates
\begin{subfigure}[c]{1\textwidth}
\centering
\begin{align*}
&\mathit{class}(\mathit{String}) \\
&\mathit{variable}(\mathit{String})\\
&\inprog(\mathit{String},\mathit{List[Constraint]}, \mathit{Constraint}) \\
&\mathit{entails}(\mathit{List[Constraint]}, \mathit{Constraint})
\end{align*}
\subcaption{Predicates}
\label{subfig:axioms-naive-general-predicates}
\end{subfigure}\\
\hrule
% Functions
\begin{subfigure}[c]{1\textwidth}
\centering
\begin{align*}
\mathit{subst-path(Path, String, Path): Path} \\
\mathit{subst-constraint(Constraint, String, Path): Constraint} \\
\mathit{subst-constraints(List[Constraint], String, Path): List[Constraint]} \\
\mathit{subst(c1: Constraint, x: String, p: Path, c2: Constraint)} \\
= \mathit{subst-constraint(c1, x, p) = c2} \\
\mathit{Entails(cs1: List[Constraint], cs2: List[Constraint])} \\
= \mathit{cs2.foreach(entails(cs1, \_))}
\end{align*}
\subcaption{Functions}
\label{subfig:axioms-naive-general-funs}
\end{subfigure}
\caption{Sorts, Predicates, Functions of Na\"ive approach}
\label{fig:axioms-naive-general}
\end{figure}

\subsection{Refined version}
\subsection{Scala integration}

\section{Interpreter}
\section{Type Relation}
\subsection{Well formedness of Programs}
\subsection{Type assignments for Expressions}

%%% Local Variables: 
%%% mode: latex
%%% TeX-master: "../thesis"
%%% End: 
