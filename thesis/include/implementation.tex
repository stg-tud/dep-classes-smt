\chapter{$DC_C$ Implementation}
- implementation in scala
- general structure of impl
  - constraint system solved with SMT solver
  - SMT solver input is SMTLib format
  - "connection" to smt solver simple external process call

\section{Constraint System}
- constraint system in DCc given as sequent calculus
- in order to use SMT on constraint system
  → transform sequent calculus into first order axioms
  → use SMTlib representation of these axioms as input to solver
\subsection{Na\"ive approach}
- naive approach
- goal is to be as close to the calculus rules as possible
- "preserve" structure of calculus rules

- make figure of needed functions, predicates, sorts
- make figure of rules
- explain rules
- list problems of these rules
  - too complex (e.g. quantified variables, no direct way of deduction in subst rule)
  - not "structured" (permutation)
  
  
\Cref{test-a,test-b}
\begin{figure}
\centering
\begin{subfigure}[c]{0.5\textwidth}
\centering
\begin{align*}
  foo\\
  bar
\end{align*}
\subcaption{Subfigure Bild Nr. 1}
\label{test-a}
\end{subfigure}
\begin{subfigure}[c]{0.5\textwidth}
\centering
\begin{align*}
\mathit{Path}\\
  &\mathit{var(id: String)}\\
  &\mathit{pth(obj: Path, field: String)}\\
\mathit{Constraint}\\
  &\mathit{path-eq(p: Path, q: Path)}\\
  &\mathit{instance-of(instance: Path, cls: String)}\\
  &\mathit{instantiated-by(object: Path, clsname: String)}\\
\end{align*}
\subcaption{Subfigure Bild Nr. 2}
\label{test-b}
\end{subfigure}
\caption{Zwei Bilder mit Subfigure nebeneinander}
\end{figure}
  
\begin{figure}
% Sorts
\begin{align*}
\mathit{Path}\\
  &\mathit{var(id: String)}\\
  &\mathit{pth(obj: Path, field: String)}\\
\mathit{Constraint}\\
  &\mathit{path-eq(p: Path, q: Path)}\\
  &\mathit{instance-of(instance: Path, cls: String)}\\
  &\mathit{instantiated-by(object: Path, clsname: String)}\\
\end{align*}

% Predicates
\begin{align*}
&\mathit{class}(\mathit{String}) \\
&\mathit{variable}(\mathit{String})\\
&\mathit{in\text{-}program}(\mathit{String},\mathit{List[Constraint]}, \mathit{Constraint}) \\
&\mathit{entails}(\mathit{List[Constraint]}, \mathit{Constraint})
\end{align*}

% Functions
\begin{align*}
\mathit{subst-path(Path, String, Path): Path} \\
\mathit{subst-constraint(Constraint, String, Path): Constraint} \\
\mathit{subst-constraints(List[Constraint], String, Path): List[Constraint]} \\
\mathit{subst(c1: Constraint, x: String, p: Path, c2: Constraint)} \\
= \mathit{subst-constraint(c1, x, p) = c2} \\
\mathit{Entails(cs1: List[Constraint], cs2: List[Constraint])} \\
= \mathit{cs2.foreach(entails(cs1, \_))}
\end{align*}
\caption{Sorts, Predicates, Functions of Na\"ive approach}
\label{fig:axioms-naive-sorts}
\end{figure}
\subsection{Refined version}
\subsection{Scala integration}

\section{Interpreter}
\section{Type Relation}
\subsection{Well formedness of Programs}
\subsection{Type assignments for Expressions}

%%% Local Variables: 
%%% mode: latex
%%% TeX-master: "../thesis"
%%% End: 
