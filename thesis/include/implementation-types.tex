\section{Type Relation}
In this section we implement the type relations presented in \Cref{sec:dcc-types}.
The $DC_C$ calculus defines type assignments for expressions
shown in \Cref{fig:dcc-typeass}
and well-formedness of programs shown in \Cref{fig:dcc-wf}.
%
\begin{figure}[h]
\begin{lstlisting}
private def mTypeSubst(m: Id, x: Id, y: Id)
    : List[(List[Constraint], List[Constraint])] =
  P.foldRight(Nil: List[(List[Constraint], List[Constraint])]){
    case (AbstractMethodDeclaration(
            `m`, x1, a, Type(y1, b)), rst) =>
      (substitute(x1, x, a),
       substitute(y1, y, b)) :: rst
    case (MethodImplementation(
            `m`, x1, a, Type(y1, b), _), rst) =>
      (substitute(x1, x, a),
       substitute(y1, y, b)) :: rst
    case (_, rst) => rst
  }
\end{lstlisting}
\caption{Function mTypeSubst}
\label{fig:scala-mtype}
\end{figure}\\
We define function \scala{mTypeSubst} in \Cref{fig:scala-mtype}.
The function implements function \mIt{mType}
presented in \Cref{fig:dcc-syntax}.
The function takes a method name $m$ and
variables $x$ and $y$ as arguments.
In \scala{mTypeSubst}, we explicitly unify the formal argument
with $x$ and the bound variable of the declared type with $y$.
In the implementation we fold over the declarations of the program.
We match abstract declarations \mDecl{m}{x_1}{a}{\type{y_1}{b}},
as well as implementations \mImpl{m}{x_1}{a}{\type{y_1}{b}}{e} of method $m$.
For each match, we substitute $x_1$ with $x$ in $a$ and $y_1$ with $y$ in $b$.


\subsection{Type Assignments for Expressions}
\label{sec:typeass}

\subsection{Well-formedness of Programs}
\label{sec:wf}
We define a function \scala{typecheck},
which implements well-formedness of programs.
The function has two implementations,
one taking a program
and the other taking a declaration
as argument.
%
\begin{figure}[h]
\begin{lstlisting}
def typecheck(P: Program): Boolean = {
  val x = freshvar()
  val y = freshvar()

  methods.forall { m =>
    val mTypes = mTypeSubst(m, x, y)

    mTypes.forall {
      case (_, b) =>
        mTypes.forall {
          case (_, b1) =>
            b.size == b1.size &&
            b.forall(c => b1.contains(c))
        }
    }
  } && P.forall(typecheck)
}
\end{lstlisting}
\caption{Well-formedness Of Programs}
\label{fig:scala-wf-prog}
\end{figure}\\
%
\Cref{fig:scala-wf-prog} shows the implementation
of function \scala{typecheck} for program inputs.
We generate fresh variables $x$ and $y$
and for all methods $m$ and variables $x$ and $y$,
we call \mIt{mTypeSubst} to obtain the argument- and return types for method $m$.
We check for each combination of specified return types  $b$ and $b_1$,
if $b$ and $b_1$ have the same number of elements
and that each element from $b$ is also contained in $b_1$.
This ensures that each implementation of method $m$
have the same specified return type.
%
Additionally, we check if each declaration is well-formed,
by passing function \scala{typecheck} to each declaration
with \scala{P.forall(typecheck)}.
%
\begin{figure}[t]
\begin{lstlisting}
def typecheck(D: Declaration): Boolean = D match {
  // WF-CD
  case ConstructorDeclaration(cls, x, a) =>
    FV(a) == List(x) || FV(a).isEmpty
    
  // WF-RD
  case ConstraintEntailment(x, a, InstanceOf(y, _))
   if x == y =>
    FV(a) == List(x) && a.exists {
      case InstanceOf(`x`, _) => true
      case _ => false
    }
    
  // WF-MS
  case AbstractMethodDeclaration(_, x, a, Type(y, b)) =>
    val vars = FV(b)
    FV(a) == List(x) && vars.nonEmpty &&
      vars.forall(v => v == x || v == y)
      
  // WF-MI
  case MethodImplementation(_, x, a, Type(y, b), e) =>
    val vars = FV(b)
    FV(a) == List(x) && vars.nonEmpty &&
      vars.forall(v => v == x || v == y) &&
    typeass(a, e).exists {
      case Type(z, c) =>
        c.size == b.size &&
        substitute(z, y, c).forall(b.contains(_))
    }
}
\end{lstlisting}
\caption{Well-formedness Of Declarations}
\label{fig:scala-wf-decl}
\end{figure}\\
\\
The implementation of well-formedness for declarations
is given in \Cref{fig:scala-wf-decl}.
The rules are syntax-directed and we match for the
structure of the declarations in the implementation.\\
\\
For constructor declarations \constr{cls}{x}{a}
we require the free variables of $a$ to be the bound
variable of the constructor $x$ or to be empty.
We allow the free variables to be empty,
because the constructor \constr{\texttt{Zero}}{x}{\epsilon}
as seen in \Cref{ex:dcc-naturalnumbers} is well-formed.\\
\\
In the implementation of rule WF-RD,
we match for constraint entailment declarations \progEnt{x}{a}{\instOf{y}{C}}
with $x = y$.
This pattern ensures that the entailed constraint
expresses that $x$ is the instance of some class.
Further we require that $x$ is the only variable free in $a$
and that at least one constraint of the form \instOf{x}{C'}
exists in $a$.\\
\\
In the implementation of rule WF-MS,
we match for abstract method declarations \mDecl{m}{x}{a}{\type{y}{b}}.
We require the free variables of $a$ to be $x$
and the free variables of $b$ to be either $x$ or $y$.\\
\\
In the implementation of rule WF-MI,
we match for method implementation declarations \mImpl{m}{x}{a}{\type{y}{b}}{e}.
We require the free variables of $a$ to be $x$
and the free variables of $b$ to be either $x$ or $y$.
We further check if the type of the method body $e$
matches the annotated type \type{y}{b}.
For this, we call function \scala{typeass}
with the context $a$ and expression $e$
to obtain a list of types of $e$.
For each type \type{z}{c} of $e$, we
substitute $z$ with $y$ in $c$ to unify the type variable
and check that each element of $c$ is also contained in $b$,
as well as requiring that the size of $c$ equals the size of $b$.

%%% Local Variables: 
%%% mode: latex
%%% TeX-master: "../thesis"
%%% End: 
