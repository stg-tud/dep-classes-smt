\section{Type Relation}
In this section we implement the type relations presented in \Cref{sec:dcc-types}.
The $DC_C$ calculus defines type assignments for expressions
shown in \Cref{fig:dcc-typeass}
and well-formedness of programs shown in \Cref{fig:dcc-wf}.
%
\begin{figure}[h]
\begin{lstlisting}
private def mTypeSubst(m: Id, x: Id, y: Id)
    : List[(List[Constraint], List[Constraint])] =
  P.foldRight(Nil: List[(List[Constraint], List[Constraint])]){
    case (AbstractMethodDeclaration(
            `m`, x1, a, Type(y1, b)), rst) =>
      (substitute(x1, x, a),
       substitute(y1, y, b)) :: rst
    case (MethodImplementation(
            `m`, x1, a, Type(y1, b), _), rst) =>
      (substitute(x1, x, a),
       substitute(y1, y, b)) :: rst
    case (_, rst) => rst
  }
\end{lstlisting}
\caption{Function mTypeSubst}
\label{fig:scala-mtype}
\end{figure}\\
We define function \scala{mTypeSubst} in \Cref{fig:scala-mtype}.
The function implements function \mIt{mType}
presented in \Cref{fig:dcc-syntax}.
The function takes a method name $m$ and
variables $x$ and $y$ as arguments.
In \scala{mTypeSubst}, we explicitly unify the formal argument
with $x$ and the bound variable of the declared type with $y$.
In the implementation we fold over the declarations of the program.
We match abstract declarations \mDecl{m}{x_1}{a}{\type{y_1}{b}},
as well as implementations \mImpl{m}{x_1}{a}{\type{y_1}{b}}{e} of method $m$.
For each match, we substitute $x_1$ with $x$ in $a$ and $y_1$ with $y$ in $b$.


\subsection{Type Assignments for Expressions}
\label{sec:typeass}
%
\begin{figure}[h]
\begin{lstlisting}
def typeassignment(context: List[Constraint], expr:Expression)
    : List[Type] = expr match {
  case x@Id(_) => [...] // T-Var
  case FieldAccess(e, f) => [...] // T-Field
  case MethodCall(m, e) => [...] // T-Call
  case ObjectConstruction(cls, args) => [...] // T-New
}
\end{lstlisting}
\caption{Type Assignment}
\label{fig:scala-typeass}
\end{figure}\quad\\
%
In this section, we define a function \scala{typeassignment}
implementing type assignments for expressions presented in \Cref{fig:dcc-typeass}.
The type relation of the $DC_C$ calculus is not immediately suitable for implementation,
as the rules are not syntax directed.
The main obstacle is the rule of subsumption T-Sub,
as the rule has a bare metavariable $e$ in its conclusion.
The other rules apply only to a specific form of expressions.
A process of transforming such a declarative system into an algorithmic one
is shown in~\cite{tpl} for the simply typed lambda-calculus.
This process includes the transformation of the declarative subtyping relation
into an algoritmic subtyping.
Since inheritance in the $DC_C$ calculus is expressed through constraint entailments,
we can use the SMT solver to check for subtyping
and incorporate subtyping into the type rules.

Type assignments in the $DC_C$ calculus are not unique.
For this, we define the function \scala{typeassignment}
to return a list of types, instead of a single type.\\
\\
The implementation of function \scala{typeassignment}
is given in \Cref{fig:scala-typeass}.
The four cases of the pattern matching are given in
\Cref{lst:scala-typeass-var,lst:scala-typeass-field,lst:scala-typeass-call,lst:scala-typeass-new}.
The function takes a list of constraints as the type context
and an expression to assign a type for as arguments.
It returns as list of types.
The function matches the structure of the expression.
%
\begin{lstlisting}[caption={Case T-Var},label=lst:scala-typeass-var,captionpos=b,frame={lines}]
case x@Id(_) => // T-Var
  classes.foldRight(Nil: List[Type]) {
    case (cls, _)
    if entails(context, InstanceOf(x, cls)) =>
      val y = freshvar()
      Type(y, List(PathEquivalence(y, x))) :: Nil
    case (_, clss) => clss
  }
\end{lstlisting}
%
For rule T-Var, we match for variable access.
To assign a type to variable $x$, we need to ensure
that the variable is an instance of some class.
We check for each class \mIt{cls} if \entails{context}{\instanceOf{x}{cls}}.
If so, we generate a fresh variable $y$ and return \type{y}{\pathEq{y}{x}}.
If $x$ is not an instance of any class, we return the empty list.
For this case it is sufficient to return after the first match,
because every further type assignment
would result in the same type
after unification of the bound variable of the types.
E.g. if \entails{context}{\instanceOf{x}{C}},
we would generate a fresh variable $z$
and obtain type \type{z}{\pathEq{z}{x}}
and \subst{\type{z}{\pathEq{z}{x}}}{z}{y} = \type{y}{\pathEq{y}{x}}.
%
\begin{lstlisting}[caption={Case T-Field},label=lst:scala-typeass-field,captionpos=b,frame={lines}]
case FieldAccess(e, f) => // T-Field
  val types = typeassignment(context, e)
  var ts: List[Type] = Nil
  types.foreach {
    case Type(x, a) =>
      val y = freshvar()
      // instance of relations for type constraints
      classes.foldRight(Nil: List[Constraint]) {
        case (cls, clss)
        if entails(context ++ a,
                   InstanceOf(FieldPath(x, f), cls)) =>
          val c = InstanceOf(y, cls)
          ts = Type(y, List(c)) :: ts
          c :: clss
        case (_, clss) => clss
      }
  }
  ts
\end{lstlisting}
%
For rule T-Field, we match for field access $e.f$.
First, we call \scala{typeassignment} on subexpression $e$ to obtain
the type assignments of $e$ and we create a list $\mIt{ts}$ to collect
all the possible types for $e.f$.
We iterate over each type \type{x}{a} of $e$ and generate a fresh variable $y$.
We fold over all classes \mIt{cls} and check if \instOf{x.f}{cls}
is entailed by the context concatenated with $a$.
If so, we add \type{y}{\instOf{y}{cls}} to the collected types \mIt{ts}.
The entailment check of \instOf{x}{cls} does consider subtyping,
since we check for all classes and also add the subtype constraints $a$
to the context.\\
%
\begin{lstlisting}[caption={Case T-Call},label=lst:scala-typeass-call,captionpos=b,frame={lines}]
// T-Call
case MethodCall(m, e) =>
  val eTypes = typeassignment(context, e)
  val y = freshvar()

  var types: List[Type] = Nil
  // for all possible argument types
  for (Type(x, a) <- eTypes) {
    // for all method declarations
    for ((a1, b) <- mTypeSubst(m, x, y)) {
      val entailsArgs = entails(context ++ a, a1)
      val b1 = (a1 ++ b).foldRight(Nil: List[Constraint]) {
        case (c, cs) if !FV(c).contains(x) => c :: cs
        case (_, cs) => cs
      }
      if (entailsArgs && entails(context ++ a ++ b, b1))
        types = Type(y, b1) :: types
    }
  }
  types
\end{lstlisting}
%
We implement rule T-Call by matching for method calls $m(e)$.
We create a fresh variable $y$
and call \scala{typeassignment} on the parameter $e$.
We iterate over each parameter type \type{x}{a}
and call \scala{mTypeSubst} to obtain
the types of method $m$,
where the parameter constraints are substituted with $x$
and the return type constraints are substituted with $y$.
For each $(a_1, b)$ obtained from \scala{mTypeSubst},
we set $b_1$ to be the constraints free of $x$
from the parameter constraints $a_1$
and the type constraints $b$.
The provided parameter is applicable to the method,
if the supplied parameter entails the formal argument of the method.
We check this through the entailment \entails{context \conc a}{a1}.
Finally, if \entails{context \conc a \conc b}{b_1} holds
we set \type{y}{b_1} as one valid type of $m(e)$.\\
\\
For rule T-New, we check for object constructions \objConstr{cls}{args}
and generate a fresh variable $x$.
We also extract the field names from the arguments
and set \mIt{argsTypes} as the mapping of
\scala{typeassignment} over the values of the arguments.
We create a list $\mIt{types}$ to collect
the types of \objConstr{cls}{args}.
There are two cases to consider.
%
\begin{enumerate}
  \item The arguments are empty.
  \item The arguments are not empty.
\end{enumerate}
\newpage
%
\begin{lstlisting}[caption={Case T-New},label=lst:scala-typeass-new,captionpos=b,frame={lines}]
// T-New
case ObjectConstruction(cls, args) =>
  val fields: List[Id] = args.map(_._1)
  val argsTypes: List[List[Type]] =
    args.map(arg => typeassignment(context, arg._2))

  val x = freshvar()
  var types: List[Type] = Nil

  argsTypes match {
    case Nil =>
      val b = List(InstantiatedBy(x, cls))
      val (x1, b1) = classInProgram(cls, P)
                       .getOrElse(return Nil)

      if (entails(context ++ b, b1))
        types = Type(x, b) :: types

      classes.foreach{
        c =>
          if (entails(context ++ b, InstanceOf(x, c)))
            types = Type(x, List(InstanceOf(x, c))) :: types
      }
    case _ => combinations(argsTypes).foreach {
      argsType =>
        val argsPairs: List[(Id, Type)] = fields.zip(argsType)
        val argsConstraints: List[Constraint] =
        argsPairs.flatMap{
          case (fi, Type(xi, ai)) =>
            substitute(xi, FieldPath(x, fi), ai)
        }

        val b: List[Constraint] =
          InstantiatedBy(x, cls) :: argsConstraints

        val (x1, b1) = classInProgram(cls, P)
                         .getOrElse(return Nil)

        if (entails(context ++ b, substitute(x1, x, b1)))
          types = Type(x, b) :: types

        classes.foreach{
          c =>
            if (entails(context ++ b, InstanceOf(x, c)))
              types = Type(x, List(InstanceOf(x, c))) :: types
        }
    }
  }
  types
\end{lstlisting}
\newpage
%
\quad\\
For (1), if there are no arguments supplied to the constructor
we set $b := \instBy{x}{cls}$.
We check if the program has a constructor \constr{cls}{x_1}{b_1}.
If so, we check \entails{context \conc b}{\subst{b_1}{x_1}{x}}
and add \type{x}{b} to \mIt{types}.
Otherwise we return the empty list.
Additionally, we examine subtyping through
checking \entails{context \conc b}{\instOf{x}{C}} for all classes $C$.
If the entailment holds, we add \type{x}{\instOf{x}{C}} to \mIt{types}.\\
\\
For (2), we generate all possible combinations
of the types of the arguments \mIt{argsTypes}.
E.g. the possible combinations of lists $[[1, 2], [3, 4], [5]]$
are the lists $[1, 3, 5]$, $[1, 4, 5]$, $[2, 3, 5]$ and $[2, 4, 5]$.
For each combination of types,
we zip the extracted field names
with the types of the arguments
to obtain a mapping from field name to type.
For each of the zipped pairs $(f_i, \type{x_i}{a_i})$,
we instantiate the type of field $f_i$
with the field access $x.f_i$,
where $x$ is the variable of the new object.
To do this, we substitute $x_i$ with $x.f_i$ in $a_i$.
%After mapping the substitution \subst{a_i}{x_i}{x.f_i}
%over all pairs, we obtain \mIt{argsConstraints}.
We set \mIt{argsConstraints} to be the mapping of
substitution \subst{a_i}{x_i}{x.f_i} over all pairs.
We set $b := \instBy{x}{cls} :: \mIt{argsConstraints}$.
We check if a constructor \constr{cls}{x_1}{b_1}
exists in the program.
If not, we can not create a new object of a non-existing class
and return the empty list.
Otherwise, if \entails{context \conc b}{\subst{b_1}{x_1}{x}}
holds we add \type{x}{b} to \mIt{types}.
We examine subtyping through
checking \entails{context \conc b}{\instOf{x}{C}} for all classes $C$.
If the entailment holds, we add \type{x}{\instOf{x}{C}} to \mIt{types}.

\subsection{Well-formedness of Programs}
\label{sec:wf}
We define a function \scala{typecheck},
which implements well-formedness of programs.
The function has two implementations,
one taking a program
and the other taking a declaration
as argument.
%
\begin{figure}[h]
\begin{lstlisting}
def typecheck(P: Program): Boolean = {
  val x = freshvar()
  val y = freshvar()

  methods.forall { m =>
    val mTypes = mTypeSubst(m, x, y)

    mTypes.forall {
      case (_, b) =>
        mTypes.forall {
          case (_, b1) =>
            b.size == b1.size &&
            b.forall(c => b1.contains(c))
        }
    }
  } && P.forall(typecheck)
}
\end{lstlisting}
\caption{Well-formedness Of Programs}
\label{fig:scala-wf-prog}
\end{figure}\\
%
\Cref{fig:scala-wf-prog} shows the implementation
of function \scala{typecheck} for program inputs.
We generate fresh variables $x$ and $y$
and for all methods $m$ and variables $x$ and $y$,
we call \mIt{mTypeSubst} to obtain the argument- and return types for method $m$.
We check for each combination of specified return types  $b$ and $b_1$,
if $b$ and $b_1$ have the same number of elements
and that each element from $b$ is also contained in $b_1$.
This ensures that each implementation of method $m$
have the same specified return type.
%
Additionally, we check if each declaration is well-formed,
by passing function \scala{typecheck} to each declaration
with \scala{P.forall(typecheck)}.
%
\begin{figure}[t]
\begin{lstlisting}
def typecheck(D: Declaration): Boolean = D match {
  // WF-CD
  case ConstructorDeclaration(cls, x, a) =>
    FV(a) == List(x) || FV(a).isEmpty
    
  // WF-RD
  case ConstraintEntailment(x, a, InstanceOf(y, _))
   if x == y =>
    FV(a) == List(x) && a.exists {
      case InstanceOf(`x`, _) => true
      case _ => false
    }
    
  // WF-MS
  case AbstractMethodDeclaration(_, x, a, Type(y, b)) =>
    val vars = FV(b)
    FV(a) == List(x) && vars.nonEmpty &&
      vars.forall(v => v == x || v == y)
      
  // WF-MI
  case MethodImplementation(_, x, a, Type(y, b), e) =>
    val vars = FV(b)
    FV(a) == List(x) && vars.nonEmpty &&
      vars.forall(v => v == x || v == y) &&
    typeassignment(a, e).exists {
      case Type(z, c) =>
        c.size == b.size &&
        substitute(z, y, c).forall(b.contains(_))
    }
}
\end{lstlisting}
\caption{Well-formedness Of Declarations}
\label{fig:scala-wf-decl}
\end{figure}\\
\\
The implementation of well-formedness for declarations
is given in \Cref{fig:scala-wf-decl}.
The rules are syntax-directed and we match for the
structure of the declarations in the implementation.\\
\\
For constructor declarations \constr{cls}{x}{a}
we require the free variables of $a$ to be the bound
variable of the constructor $x$ or to be empty.
We allow the free variables to be empty,
because the constructor \constr{\texttt{Zero}}{x}{\epsilon}
as seen in \Cref{ex:dcc-naturalnumbers} is well-formed.\\
\\
In the implementation of rule WF-RD,
we match for constraint entailment declarations \progEnt{x}{a}{\instOf{y}{C}}
with $x = y$.
This pattern ensures that the entailed constraint
expresses that $x$ is the instance of some class.
Further we require that $x$ is the only variable free in $a$
and that at least one constraint of the form \instOf{x}{C'}
exists in $a$.\\
\\
In the implementation of rule WF-MS,
we match for abstract method declarations \mDecl{m}{x}{a}{\type{y}{b}}.
We require the free variables of $a$ to be $x$
and the free variables of $b$ to be either $x$ or $y$.\\
\\
In the implementation of rule WF-MI,
we match for method implementation declarations \mImpl{m}{x}{a}{\type{y}{b}}{e}.
We require the free variables of $a$ to be $x$
and the free variables of $b$ to be either $x$ or $y$.
We further check if the type of the method body $e$
matches the annotated type \type{y}{b}.
For this, we call function \scala{typeassignment}
with the context $a$ and expression $e$
to obtain a list of types of $e$.
For each type \type{z}{c} of $e$, we
substitute $z$ with $y$ in $c$ to unify the type variable
and check that each element of $c$ is also contained in $b$,
as well as requiring that the size of $c$ equals the size of $b$.

% TODO: example: wf natural numbers
\begin{example}[Typechecking the Natural Numbers Program]\quad\\
In this example, we check if the natural numbers program
given in \Cref{ex:dcc-naturalnumbers} is well-formed.
For this, we apply \scala{typecheck} to the program.

\end{example}

%%% Local Variables: 
%%% mode: latex
%%% TeX-master: "../thesis"
%%% End: 
