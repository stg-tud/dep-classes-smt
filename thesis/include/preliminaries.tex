\chapter{Preliminaries}
\label{chp:pre}
\section{Satisfiability modulo theories}
\label{sec:smt}
Many problems like formal verification of hard- and software can be reduced
to checking the satisfiability of a formula in some logic.
Some of these problems can be easily described as a satisfiability problem
in propositional logic and solved using a propositional SAT solver.
Other problems can be described more naturally in other classical logics like first-order logic.
These logics support more expressive language features including
non-boolean variables, function and predicate-symbols and quantifiers.

The defining problem of Satisfiability modulo theories (SMT)
is checking if a logical formula is satisfiable in the context of some background theory.
Checking validity can be achieved using SMT if the formula is closed under logical negation,
since a formula is valid in a theory if it's negation is not satisfiable in the theory.

There is a trade-off between the expressiveness of a logic and
the ability to automatically check satisfiability of formulae.
A practical compromise is the usage of fragments of first-order logic,
where the used fragment is restricted syntactically or semantically. % TODO: add examples for restrictions, see notes
Such restrictions achieve the decidability of the satisfiability problem and
allow the development of procedures that exploit properties of the fragment for efficiency.
A SMT solver is any software implementing a procedure for determining satisfiability modulo a given theory.
SMT solvers can be distinguished based on the underlying logic (first-order, modal, temporal, ...),
the background theory, the accepted input formulas and the interface provided by the solver.\cite{smt,smtlib}
Z3 is a SMT solver from Microsoft Research\cite{z3}, which is used in this thesis.
%- SMT problem
%    - is logical formular satisfiable given a logical theory
%- propositional SAT solver
%- some problems described more naturally in classical logics
%    - such as fo or higher order logics
%    - with more expressive features/languages
%    - including non-Boolean variables, function and predicate-symbols (positive arity), quantifiers
%- trade-off between expressivenes of logic and ability of automatically checking satisfiability of formulae
%- practical compromise: use fragments of first-order logic
%    - fragment is restricted
%        - syntactically and/or
%        - constraining the interpretation of certain function and predicate symbols
%        - makes the satisfiability problem decidable
%        - allows development of procedures that exploit properties of the fragment for efficiency

\subsection{SMT-Lib}
SMT-Lib is a international initiative with the goal to easing
research and development in SMT.
SMT-Lib's main motivation is the availability of common standards in SMT
with the focused on
\begin{itemize}
    \item providing a standard description of background theories
    \item developing a common in- and output language for SMT solvers
    \item establishing a library of benchmarks for SMT solvers
\end{itemize}
to advance the state of the art in the field.
The SMT-Lib Standard: Version 2.6\cite{smtlib} defines a language for
writing terms and formulas in sorted first-order logic,
specifying background theories,
specifying logics,
as well as a command language for interacting with SMT solvers.
The SMT-Lib format is accepted by the majority of current SMT solvers.
%- SMT-Lib international initiative aimed at facilitating research and developmentin SMT
%- focused on
%    - provide standard rigorous descriptions of background theories used in SMT
%    - develop + promote common in/output language for SMT solvers
%    - establish library of benchmarks for SMT solvers
%- main motivation for SMT-Lib: availability of common standards
%    and of library of benchmarks facilitate evulution and comparison for SMT systems
%- advance the state of the art in the field
%- SMT-Lib format is accepted by majority of current SMT solvers
%- SMT-Lib Standard 2.6 defines
%    - language for writing terms and formulas in sorted fo
%    - language for specifying background theories
%        - standard vocabulary of sort, function and predicate symbols
%    - language for specifying logics
%        - suitably restricted classes of formulas
%    - command language for interacting with SMT solvers
%        - asserting and retracting formulas
%        - querying about satisfiability
%        - examining models or unsat proofs
%        - ...
An example of the SMT-Lib format is shown in figure 2.1. %TODO: figure numbering style?
The example is showing a formula in sorted first-order logic and the SMT-Lib representation of the formula.
The formula is quantifying over two integer lists,
using a let binding to extract the head and the tail from the first list
to construct a identical list out of these bindings and checks for equality to the second list.
The example is purely for demonstrating the syntax of the SMT-Lib format.

\begin{figure}
\begin{align*}
&\forall x: List[Int]. \exists y: List[Int]. \\
&\text{ let } h := head(x), t := tail(x)\\
&\text{ in } insert(h, t) = y
\end{align*}
\begin{center}
(forall ((x (List Int))) (exists ((y (List Int))) \\
(let ((h (head x)) (t (tail x))) \\
(= (insert h t) y))))
\caption{SMT-Lib format example}
\end{center}
\end{figure}

\section{Dependent Classes}
\label{sec:depcls}
Classes tend to be a too isolated unit to achieve modularity.
Desired functionality involves groups of related classes.
Grouping mechanisms exist, like namespaces in C++ and packages in Java,
but these mechanisms do not cover inheritance and polymorphism for expressing variability.
Virtual classes\cite{virtual:classes, vaidas:thesis} provide a solution for inheritance and polymorphism.
They introduce classes as a kind of object members and treat these as virtual methods,
which allows for overriding in subclasses and are late-bound.
Virtual classes are inner classes refinable in the subclasses of the enclosing class.
%- classes not good enough for modularity
%- functionality involves a group of related classes
%- grouping mechanisms (namespaces, packages, ...)
%- does not cover inheritance and polymorphism for expressing variability
%- virtual classes \cite{virtual:classes}
%    - provide solution for inheritance and polymorphism
%    - introduce classes as "object members" and treat them like (virtual) methods
%        - allow overriding in subclasses, late-bound
%    - are inner classes that can be refined in the subclasses of the enclosing class

The downside of Virtual Classes is that they must be nested within other classes.
This requires to cluster classes depending on instances of some class together,
introducing maybe unwanted coupling between these classes.
Introducing a new class depending on the instances of an existing class
requires the modification this class and its subclasses,
limiting extensibility.
This Nesting does also limit the expression of variability:
The interface and the implementation of a virtual class can only depend
on its single enclosing object.

Dependent classes\cite{dc,vaidas:thesis} are generalization of virtual classes.
The structure of a dependent class depends on arbitrarily many objects and
dependency is expressed with class parameters.
Dependent classes can be seen as a combination of virtual classes with multi-dispatch.

The semantics of dependent classes involve non-trivial aspects,
such as method calls and class instantiation expressions,
which rely on dynamic dispatch over multiple parameters and the types of their fields.
As well as the complexity of the type system,
which integrates static dispatch with dependent typing.
%- semantics of dependent classes
%    - invole multiple non-trivial aspects
%    - method calls and class instantiation expressions rely on dynamic dispatch over multiple parameters
%        and the types of fields
%    - most of the complexity in type system
%        - integrates static dispatch with dependent typing

The $vc^n$ calculus captures the core semantics of dependent classes,
such as static and dynamic dispatch and dependent typing based on paths.
The calculus ensures that the static dispatch and static normalization of terms in types
define a proper abstraction over dynamic dispatch and evaluation of expressions.
It is defined in an algorithmic style.
This algorithmic style has the advantage of being constructive
and through this outlining a possible implementation.
The algorithmic nature of the calculus comes also with the disadvantage
of being difficult to extend semantics with new expressive power,
such as extending the calculus with support for abstract declarations.
%- vcn calculus
%    - captures core semantics of dependent classes,
%        including static and dynamic dispatch and dependent typing based on paths
%    - calculus defined to ensure static dispatch and static normalization of terms in types
%        defines proper abstraction over dynamic dispatch and evaluation of expressions
%    - defined in an algorithmic style
%        - for ensuring decidability of type system through straightforward proof TODO: add this?
%    - algorithmic style has advantage of being constructive -> outlines a possible implementation
%    - at the same time disadvantage: difficult to extend semantics with new expressive power
%        - difficult to extend calculus with support for abstract declarations

\subsection{$DC_C$ Calculus}
The $vc^n$ calculus does not support abstract classes and methods.
The $DC_C$ calculus\cite{vaidas:thesis} extends $vc^n$ with support for abstract classes and methods
and symmetric method dispatch.
$DC_C$ encodes dependent classes with a constraint system,
which is used in both the static and dynamic semantics.
The main relation of the $DC_C$ calculus is constraint entailment,
replacing $vc^n$'s relations for type equivalence, subtyping, static and dynamic dispatch.

The runtime structure is heap based,
with explicit object identities
and relationships between objects based on these identities.
Expressions evaluate to an identifier pointing to an object in the heap.
Heaps preserve object identity and enable shared references to objects.
The heap provides a direct interpretation for equivalent paths,
two paths are equivalent if they point to the same object at runtime.
Heaps can be easily translated to a set of constraints describing its objects and their relations,
which enables the usage of the constraint system for dynamic dispatch and expression typing.
%- DCc extends $vc^n$ with abstract methods and classes and symmetric method dispatch
%- gave up unification of methods and classes
%- runtime structure: heap
%- heap structure preserves object identity, enables shared references
%- heap provides direct interpretation for equivalent paths
%    - paths are equiv if they point to the same object at runtime
%- heaps can be easily translated to a set of constraints describing its objects and their relations
%    - enables using constraint system for dynamic dispatch and expression typing
\subsubsection{Syntax}
The syntax of $DC_C$ is given in figure 2.2.
Types are lists of constraints to be satisfied by their instances.
Types have the form $[x. \overline(a)]$, where $x$ is a bound variable
and $\overline(a)$ is a list of constraints on $x$.
An object belongs to a type if it fulfills its constraints.

Constraints of the form $p \equiv q$ express that two paths $p$ and $q$ are equivalent
and paths are considered to be equivalent if they refer to the same object at runtime.
$p :: C$ specifies that path $p$ refers to an instance of class $C$.
The stronger form $p.\textbf{cls} \equiv C$ denotes that
path $p$ refers to an object instantiated by a constructor of class $C$,
excluding indirect instances of $C$ inferred through inheritance rules.

A Program $P$ consists of a list of declarations $D$.
Possible declarations are constructor declarations,
abstract method declarations, method implementations and constraint entailment rules.

A Path expression can be a variable $x$ or navigation over fields starting from a variable e.g. $x.f$.

Expressions can be variables, field access, object construction and method invocation.
Field assignments are not supported since the calculus is functional.

% begin syntax figure
\begin{figure}
\setlength{\grammarindent}{5em} % increase separation between LHS/RHS
% TODO: use <> on construction at rhs?
\begin{grammar}
<Program>  ::= $\overline{Decl}$

<Decl> ::= \constr{C}{x}{\overline{Constr}} % C(x. $\overline{Constr}$)
       | \progEnt{x}{\overline{Constr}}{Constr}
       \alt \mDecl{m}{x}{\overline{Constr}}{Type} % m(x. $\overline{Constr}$): Type
       | \mImpl{m}{x}{\overline{Constr}}{Type}{Expr} % m(x. $\overline{Constr}$): Type := Expr

<Type> ::= [x. $\overline{Constr}$]

<Constr> ::= \pathEq{Path}{Path} % Path $\equiv$ Path
           | \instOf{Path}{C} % Path :: C
           | \instBy{Path}{C} % Path.\textbf{cls} $\equiv$ C

<Path> ::= x
       | Path.f

<Expr> ::= x
       | Expr.f
       | \newInst{C}{\overline{f}}{\overline{Expr}} % \textbf{new} C($\overline{f}$ $\equiv$ $\overline{Expr}$)
       | m(Expr)
\end{grammar}

\begin{align*}
MType(m, x, y) &= \{ \langle\overline{a}, \overline{b}\rangle | (m(x. \overline{a}): [y. \overline{b}]...) \in P \} \\
MImpl(m, x) &= \{\langle\overline{a}, e\rangle | (m(x. \overline{a}): [y. \overline{b}] := e) \in P \}
\end{align*}

x, y $\in$ variable names\\
f $\in$ field names\\
C $\in$ class names\\
m $\in$ method names
\caption{Syntax}
\end{figure}
% end syntax figure

\subsubsection{Constraint System}
% begin constraint system figure
\begin{figure}
% C-Ident
\begin{prooftree}
\AxiomC{}
\RightLabel{(C-Ident)}
\UnaryInfC{\entails{a}{a}}
\end{prooftree}
% C-Refl
\begin{prooftree}
\AxiomC{}
\RightLabel{(C-Refl)}
\UnaryInfC{\entails{\epsilon}{\pathEq{p}{p}}}
\end{prooftree}
% C-Class
\begin{prooftree}
\AxiomC{\entails{\overline{a}}{\instantiatedBy{p}{C}}}
\RightLabel{(C-Class)}
\UnaryInfC{\entails{\overline{a}}{\instanceOf{p}{C}}}
\end{prooftree}
% C-Cut
\begin{prooftree}
\AxiomC{\entails{\overline{a}}{c}}
\AxiomC{\entails{\overline{a'}, c}{b}}
\RightLabel{(C-Cut)}
\BinaryInfC{\entails{\overline{a},\overline{a'}}{b}}
\end{prooftree}
% C-Subst
\begin{prooftree}
\AxiomC{\entails{\overline{a}}{a_{\sub{x}{p}}}}
\AxiomC{\entails{\overline{a}}{\pathEq{p'}{p}}}
\RightLabel{(C-Subst)}
\BinaryInfC{\entails{\overline{a}}{a_{\sub{x}{p'}}}}
\end{prooftree}
% C-Prog
\begin{prooftree}
\AxiomC{$(\progEnt{x}{\overline{a}}{a}) \in P$}
\AxiomC{\entails{\overline{b}}{\overline{a}_{\sub{x}{p}}}}
\RightLabel{(C-Prog)}
\BinaryInfC{\entails{\overline{b}}{a_{\sub{x}{p}}}}
\end{prooftree}
\caption{Constraint Entailment}
\end{figure}
% end constraint system figure

\subsubsection{Operational Semantics}
% begin Operational Semantics figure
\begin{figure}
% TODO: both alignments side-by-side
\begin{align*}
o &::= \stdobj \\ % \langle C; \overline{f} \equiv \overline{x} \rangle \\
h &::= \stdheap\ \ (x_i \text{ distinct})
\end{align*}
\begin{align*}
OC(x, o) &= (\instBy{x}{C}, \pathEq{x.\overline{f}}{\overline{x}}) &&\text{where } o = \stdobj \\ % \langle C; \overline{f}\equiv\overline{x} \rangle \\
HC(h) &= \bigcup_i OC(x_i, o_i) &&\text{where } h = \stdheap % \overline{x} \mapsto \overline{o}
\end{align*}
% vertical inference rule example
%\begin{prooftree}
%\AxiomC{$A\lor B$}
%\AxiomC{$[A]$}
%\noLine
%\UnaryInfC{$C$}
%\AxiomC{$[B]$}
%\noLine
%\UnaryInfC{$C$}
%\TrinaryInfC{$C$}
%\end{prooftree}
% R-New
\begin{prooftree}
\AxiomC{$x \not \in dom(h)$}
\noLine
\UnaryInfC{o = \stdobj}
\AxiomC{$\constr{C}{x}{\overline{b}} \in P$}
\noLine
\UnaryInfC{\entails{HC(h), OC(x, o)}{\overline{b}}}
\RightLabel{R-New}
\BinaryInfC{$\eval{h}{\newInst{C}{\overline{f}}{\overline{x}}}{h, x \mapsto o}{x}$}
\end{prooftree}
% R-Field
\begin{prooftree}
\AxiomC{$(\pathEq{x.f}{y}) \in HC(h)$}
\RightLabel{R-Field}
\UnaryInfC{\eval{h}{x.f}{h}{y}}
\end{prooftree}
% R-Call
\begin{prooftree}
\AxiomC{$S = \{ \pair{\overline{a}}{e}\ |\ \pair{\overline{a}}{e} \in MImpl(m, x) \land \entails{HC(h)}{\overline{a}} \}$}
\AxiomC{$\pair{\overline{a}}{e} \in S$}
\noLine
\BinaryInfC{$\forall \pair{\overline{a'}}{e'} \in S.\ (e' \neq e) \longrightarrow (\entails{\overline{a'}}{\overline{a}}) \land \neg(\entails{\overline{a}}{\overline{a'}})$}
\RightLabel{R-Call}
\UnaryInfC{\eval{h}{m(x)}{h}{e}}
\end{prooftree}
% RC-Field
\begin{prooftree}
\AxiomC{\eval{h}{e}{h'}{e'}}
\RightLabel{RC-Field}
\UnaryInfC{\eval{h}{e.f}{h'}{e'.f}}
\end{prooftree}
% RC-Call
\begin{prooftree}
\AxiomC{\eval{h}{e}{h'}{e'}}
\RightLabel{RC-Call}
\UnaryInfC{\eval{h}{m(e)}{j'}{m(e')}}
\end{prooftree}
% RC-New
\begin{prooftree}
\AxiomC{\eval{h}{e}{h'}{e'}}
\RightLabel{RC-New}
\UnaryInfC{ % TODO: too long for box
  \eval
    {h}
    {\newInst{C}
      {\overline{f} \equiv \overline{x}, f}
      {e, \overline{f'} \equiv \overline{e'}}}
    {h'}
    {\newInst{C}
      {\overline{f} \equiv \overline{x}, f}
      {e', \overline{f'} \equiv \overline{e'}}}}
\end{prooftree}

\caption{Operational semantics}
\end{figure}
% end Operational Semantics figure

\subsubsection{Type Checking}
% begin Type assignment figure
\begin{figure}
TODO
\caption{Type assignment}
\end{figure}
% end Type assignment figure
% begin Type Checking figure
\begin{figure}
TODO
\caption{Type checking}
\end{figure}
% end Type Checking figure

%%% Local Variables: 
%%% mode: latex
%%% TeX-master: "../thesis"
%%% End: 


%%% Local Variables: 
%%% mode: latex
%%% TeX-master: "../thesis"
%%% End: 
