\chapter{Introduction}
\section{Motivation}
\section{Contributions}
The main contributions of the thesis are:
\begin{itemize}
  \item An implementation of the $DC_C$ calculus
        using SMT solving to resolve the constraint system.
  \item A first-order model of the $DC_C$ constraint system,
        which translates the rules of a sequent calculus
        to universal quantified formulae.
  \item Optimizations to the rules of the first-order model,
        such that a SMT solver can make better use of them.
  \item An implementation of the $DC_C$ relation for
        type assignments to expressions.
  \item A way to use type information of an expression
        in the interpretation of that expression
        to reduce the time to solve the constraint system.
\end{itemize}

\section{Structure}
\Cref{chp:pre} presents preliminaries.
We introduce Satisfiability Modulo Theories,
Dependent Classes and the $DC_C$ calculus.\\
\\
\Cref{chp:impl} gives an implementation of the $DC_C$ calculus.
We implement
the constraint system in \Cref{sec:constraintsystem},
the operational semantics in \Cref{sec:interp} and
the type relation in \Cref{sec:types}.\\
\\
\Cref{chp:learning} presents an approach
to use information gained from compile time (type checker)
during runtime (interpreter).\\
\\
Related work is discussed in \Cref{chp:related}
and we conclude the thesis in \Cref{chp:discuss}.

%%% Local Variables: 
%%% mode: latex
%%% TeX-master: "../thesis"
%%% End: 
