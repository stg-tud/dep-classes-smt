\section{Interpreter}
In this section we will implement
the operational semantics of the $DC_C$ calculus
presented in \Cref{fig:dcc-opsemantics}.
For this we will define an interpreter,
which will evaluate expressions relative to a heap.
The operational semantics is a small-step semantics over
the structure of expressions.
%
\begin{lstlisting}[caption={Interpreter},label=lst:interp,captionpos=b,frame={lines}]
def interp(heap: Heap, expr: Expression)
          : (Heap, Expression) = expr match {
  // R-Field
  case FieldAccess(X@Id(_), F@Id(_)) =>
    HC(heap).filter{
        case PathEquivalence(FieldPath(X, F), Id(_)) => true
        case PathEquivalence(Id(_), FieldPath(X, F)) => true
        case _ => false
        } match {
      case PathEquivalence(FieldPath(X, F), y@Id(_)) :: _ =>
        (heap, y)
      case PathEquivalence(y@Id(_), FieldPath(X, F)) :: _ =>
        (heap, y)
      case _ => (heap, expr) // x does not has field f
    }
  // R-Call
  case MethodCall(m, x@Id(_)) =>
    // Applicable methods
    val S: List[(List[Constraint], Expression)] =
        mImplSubst(m, x).filter{
          case (as, _) => entails(HC(heap), as)}

    var (a, e) = S.head // Most specific method
    S.foreach{
      case (a1, e1) if e != e1 =>
        if (entails(a1, a) && !entails(a, a1)) {
          a = a1
          e = e1
        }
    }
    (heap, e)
  // R-New
  case ObjectConstruction(cls, args)
    if args.foldRight(true){ // if args are values (Id)
      case ((_, Id(_)), rst) => rst
      case _ => false
    } =>
    val x: Id  = freshvar()
    val args1: List[(Id, Id)] = args.map{
      case (f, Id(z)) => (f, Id(z))} // case (f, _) => (f, Id('notReduced)) guard makes sure everything is an Id TODO: remove this comment after describing the case because its too friggin long for the thesis
    val o: Obj = (cls, args1)
    // cls in Program: alpha renaming of y to x in b
    val (y: Id, b: List[Constraint]) =
      classInProgram(cls, P).getOrElse(return (heap, expr))
    val b1 = alphaConversion(y, x, b)
    // heap constraints entail cls constraints
    if (entails(HC(heap) ++ OC(x, o), b1))
      (heap + (x -> o), x)
    else
      (heap, expr) // stuck
      
  // RC-Field
  case FieldAccess(e, f) =>
    val (h1, e1) = interp(heap, e)

    if(h1 == heap && e1 == e) {
      (heap, expr) // stuck
    } else {
      interp(h1, FieldAccess(e1, f)) // recursive call for big-step TODO: remove this comment like the one above
    }
    
  // RC-Call
  case MethodCall(m, e) =>
    val (h1, e1) = interp(heap, e)

    if(h1 == heap && e1 == e) {
      (heap, expr) // stuck
    } else {
      interp(h1, MethodCall(m, e1))
    }
    
  // RC-New
  case ObjectConstruction(cls, args) =>
    val (h1, args1) = objArgsInterp(heap, args)

    if(h1 == heap && args1 == args) {
      (heap, expr) // stuck
    } else {
      interp(h1, ObjectConstruction(cls, args1))
    }
}
\end{lstlisting}
%
\newpage
We define function \scala{interp} in \Cref{lst:interp}.
The function takes a heap and an expression
as arguments and returns a heap and an expression.
The implementation follows the structure of the operational
semantics.
This is done via pattern matching on the argument expression
and allows to recreate each rule from \Cref{fig:dcc-opsemantics}
as a separate case in the implementation.

% TODO: describe cases


%%% Local Variables: 
%%% mode: latex
%%% TeX-master: "../thesis"
%%% End: 
