\chapter{Related Work}
\label{chp:related}

\begin{minipage}{0.9\textwidth}
\subsubsection{Logically Qualified Data Types (Liquid Types)}
Liquid Types~\cite{liquid-types} are a system
combining Hindley-Milner type inference with Predicate Abstraction
to infer dependent types.
The inferred dependent types are precise enough
to prove various safety properties.
Liquid Types adopt the benefits of
static verification of critical properties and
the elimination of runtime checks
from dependent types,
without the need for manual annotations.
%An implementation of the Liquid Type inference
%is available. TODO: remove this or verwurschteln?
\\
\subsubsection{Liquid Haskell: Haskell as a Theorem Prover}
Liquid Haskell~\cite{liquid-haskell} is an usable
program verifier, integrating
the specification of correctness properties
as logical refinements of Haskell's types.
It uses the abstract interpretation framework of Liquid Types,
to check correctness of specifications via SMT solving,
requiring no explicit proofs or annotations.
The specification language for Liquid Haskell
is arbitrary expressive to allow
the writing of general correctness properties,
thus turning Haskell into a theorem prover.
\\
\subsubsection{Dependent Object Types (DOT)}
DOT~\cite{dot1,dot2} is a calculus
modeling Scala's path-dependent types and abstract type members.
It uses refinement types to model the mixture of
nominal and structural typing in Scala.
DOT normalizes the type system of Scala
through the unification of the constructs for type members.
It provides intersection and union types
to simplify the computations for greatest lower-bounds
and least upper-bounds.
DOT is at the core of \textit{dotty},
a compiler for Scala under development since 2013
considered for inclusion in Scala 3.
\end{minipage}

\begin{minipage}{0.9\textwidth}
\subsubsection{Veritas}
VeriTaS~\cite{veritas1,veritas2} is an ongoing project
for the specification and verification
of domain-specific languages (DSL).
VeriTaS is aimed at automatically proving type soundness 
using first-order theorem proving.
It allows for specifying
the syntax and semantics of DSL
and proof goals and axioms.
It includes a compiler to translate
these language specifications and the proof goals on them
and proofs can be structured via proof graphs.
\\
\subsubsection{Multiple Dispatch as Dispatch on Tuples}
Single dispatch selects a method using
the dynamic class of the message's receiver.
Multiple dispatch is a generalization of single dispatch.
It selects a method based on the dynamic class
of any subset of the message's arguments.
Multiple Dispatch as Dispatch on Tuples~\cite{multidispatch-tuple}
proposes a way to add multiple dispatch
to existing languages with single dispatch,
without affecting existing code.
This is achieved through the addition
of tuples as primitive expressions
and the ability of messages to be sent to tuples.
Methods are selected based on the dynamic classes
of the elements of the tuple.
\end{minipage}



%%% Local Variables: 
%%% mode: latex
%%% TeX-master: "../thesis"
%%% End: 
