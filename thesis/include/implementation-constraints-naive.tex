\subsection{Na\"ive approach}
%- naive approach
%- goal is to be as close to the calculus rules as possible
%- "preserve" structure of calculus rules
The goal of this model of the constraint system is to
be close to the given sequent calculus and to preserve
the structure of the calculus rules in the first-order formulae.\\
% Sorts
\begin{figure}[t]
\centering
\begin{subfigure}[c]{0.45\textwidth}
% BNF
\begin{align*}
\mIt{Path} &::=
     \mIt{String}\\
  &\quad|\ \mIt{Path.String}
\end{align*}
\end{subfigure}
\begin{subfigure}[c]{0.45\textwidth}
% BNF
\begin{align*}
\mIt{Constraint} &::=
     \pathEq{Path}{Path}\\
  &\quad|\ \instOf{Path}{String}\\
  &\quad|\ \instBy{Path}{String}
\end{align*}
\end{subfigure}
\caption{Sorts}
\label{subfig:axioms-naive-general-sorts}
\end{figure}\\
First we define a set of sorts, predicates and functions.
These definitions form the general structure of the model
and provide basic functionality needed in the modeling of the calculus rules.

Sorts for paths and constraints are defined in \Cref{subfig:axioms-naive-general-sorts}.
The definitions are similar to the syntax specification in \Cref{fig:dcc-syntax}.
A path is either a variable name or a field path consiting
of another path followed by a field name, where variable and field names
are modeled as strings.
The three types of $DC_C$ constraints are translate as follows in our model:
\begin{itemize}
  \item For two paths $p, q$ the constraint \pathEq{p}{q} requires $p$ and $q$
        to be equivalent.
  \item For a path $p$ and a class name $C$ modeled as a string,
        \instOf{p}{C} requires path $p$ to be an instance of class $C$.
  \item For a path $p$ and a class name $C$ modeled as a string,
        \instBy{p}{C} requires path $p$ to be directly instantiated by class $C$.
\end{itemize}\quad
% Functions for substitution
\begin{figure}[t]
\centering
\begin{align*}
% path substitution
&\substpath{p_1: Path}{x: String}{p_2: Path}:\mIt{Path} = p_1 \match\\
&\quad y \Rightarrow \ite{x=y}{p_2}{p_1}\\
&\quad q.f \Rightarrow \substpath{q}{x}{p_2}.f\\
%&\quad \If \is{var}(p_1) \\
%&\qquad \Then x = \ite{id(p_1)}{p_2}{p_1}\\ % replace id(p_1)
%&\qquad \Else \substpath{obj(p_1)}{x}{p_2}.f\\ %: replace obj(p_1) and f
\\
% constraint substitution
&\substconstr{c: Constraint}{x: String}{p: Path}\mathit{: Constraint} = c \match \\
&\quad \case{\pathEq{q_1}{q_2}}
  {\pathEq
    {\substpath{q_1}{x}{p}}
    {\substpath{q_2}{x}{p}}}\\
&\quad \case{\instOf{q}{C}}
  {\instOf{\substpath{q}{x}{p}}{C}}\\
&\quad \case{\instBy{q}{C}}
  {\instBy{\substpath{q}{x}{p}}{C}}\\\\
% constraints substitution
&\substconstrs{cs: List[Constraint]}{x: String}{p: Path}\mathit{: List[Constraint]} =\\
&\quad cs \match\\
&\qquad \case{\epsilon}{\epsilon}\\
&\qquad \case{hd :: tl}
  {\substconstr{hd}{x}{p} ::
  \substconstrs{tl}{x}{p}}
\end{align*}
\caption{Path Substitution Functions}
\label{subfig:axioms-naive-general-funs}
\end{figure}\\
Functions defining path substitution are given in \Cref{subfig:axioms-naive-general-funs}.
Path substitution is the process of substituting the variable name of a path
with another path.
Therefore substitution can only occur on the innermost part of a path and
substitution only happens if the variable name of the path equals the variable name to be substituted.
Substitution for constraints propagates substitution to each path contained within a constraint.\\
E.g. for path equivalence constraints \pathEq{p_1}{p_2} a substitution from $x$ to $q$
substitutes both $p_1$ and $p_2$ with the substitution from $x$ to $q$.
Substitution for lists of constraints applies the substitution
to each constraint contained in the list.
Substitution for strings $x$ and paths $p$ will be written
for constraints $a$ as \subst{a}{x}{p} and
for constraint lists \ovl a as \subst{\ovl a}{x}{p},
function application is distinguishable based on the sort of the first argument.
% Predicates
\begin{figure}[h]
\centering
\begin{align*}
&\mathit{class}(\mathit{String}) \\
&\mathit{variable}(\mathit{String})\\
&\inprog(\mathit{String},\mathit{List[Constraint]}, \mathit{Constraint}) \\
&\mathit{entails}(\mathit{List[Constraint]}, \mathit{Constraint})\\\\
&\mathit{Entails(cs_1: List[Constraint], cs_2: List[Constraint])} = cs_2 \match\\
&\quad \case{\epsilon}{true}\\
&\quad \case{hd :: tl}{\mIt{entails}(cs_1, hd)} \land \mIt{Entails}(cs_1, tl)\\\\
&\mathit{subst(c_1: Constraint, x: String, p: Path, c_2: Constraint)} = \\
&\quad\substconstr{c_1}{x}{p} = c_2
\end{align*}
\caption{Predicates}
\label{subfig:axioms-naive-general-predicates}
\end{figure}\\
Predicates are declared in \Cref{subfig:axioms-naive-general-predicates}.
The predicate \mIt{class(C)} ensures that $C$ is a valid class name
and \mIt{variable(x)} ensures that $x$ is a valid variable name
of the program.
Class- and variable names are modeled as strings.
The predicate \inprog\ models the existence check of program entailments
as seen in rule C-Prog in \Cref{fig:dcc-constraint-entailment}.
For strings $x$, constraint lists \ovl{a} and constraints $a$,
$\inprog(x, \ovl{a}, a)$ ensures that a declaration \progEnt{x}{\ovl{a}}{a}
exists in the program.
The predicate \mIt{entails} models the entailment judgement of the sequent calculus.
The predicate \mIt{Entails} models the entailment judgement for multiple constraints,
which is defined as $\entails{\ovl a}{\ovl b} = \bigwedge_{b \in \ovl b}{\entails{\ovl a}{b}}$.
For constraint lists \ovl{a} and constraints $a$
$\mIt{entails}(\ovl{a}, a)$ ensures that \entails{\ovl{a}}{a} holds.
For constraint lists \ovl{a} and constraint lists \ovl{b},
$\mIt{Entails}(\ovl{a}, \ovl{b})$ holds if
$\mIt{entails}(\ovl{a}, b)$ holds for each $b \in \ovl{b}$.
For better readability in the rules, we write
%$\inprog(x, \ovl{a}, a)$ as $\progEnt{x}{\ovl{a}}{a} \in P$ and
$\mIt{entails}(\ovl{a}, a)$ as \entails{\ovl{a}}{a} and
$\mIt{Entails}(\ovl{a}, \ovl{b})$ as \entails{\ovl{a}}{\ovl{b}}.
The writing styles of \mIt{entails} and \mIt{Entails} can be distinguished
based on the sort of the right-hand side argument.
The predicate \mIt{subst} models constraint equality after substitution.
For constraints $a, b$, strings $x$, paths $p$
$\mIt{subst}(a, x, p, b)$ holds if
$x$ substituted with $p$ in $a$ equals $b$.
%- explain rules
%- list problems of these rules
%  - too complex (e.g. quantified variables, no direct way of deduction in subst rule)
%  - not "structured" (permutation)
%- pretty printing
%  - symbols for functions, etc
%  - substitution
%  - generalization
%  - list notion
%  - list concatenation + inserting
% figure naive axioms
\begin{figure}[h]
\begin{align*}
&\forall c: \Constr.\ \entails{[c]}{c} && \text{(C-Ident)} \\
&\forall p: \Path.\ \entails{\nil}{\pathEq{p}{p}} && \text{(C-Refl)} \\
&\forall \ovl{a}: \Constrs, p: \Path, C: \String.&& \text{(C-Class)} \\
&\quad \mIt{class}(C) \land \entails{\ovl{a}}{\instBy{p}{C}}
       \rightarrow \entails{\ovl{a}}{\instOf{p}{C}} \\
&\forall \ovl{a}, \ovl{a'}: \Constrs, b, c: \Constr. && \text{(C-Cut)} \\
&\quad \entails{\ovl a}{c} \land \entails{c :: \ovl{a'}}{b}
       \rightarrow \entails{\ovl a \conc \ovl{a'}}{b}\\
&\forall \ovl a: \Constrs, x: \String, p_1, p_2: \Path, a, a_1, a_2: \Constr. && \text{(C-Subst)} \\
&\quad \entails{\ovl a}{\pathEq{p_2}{p_1}} \land \mIt{variable}(x)
         \land \mIt{subst}(a, x, p_1, a_1) \land \mIt{subst}(a, x, p_2, a_2)\\
&\quad   \land \entails{\ovl a}{a_1}
       \rightarrow \entails{\ovl a}{a_2} \\
&\forall \ovl a, \ovl b: \Constrs, a: \Constr, x: \String, p: \Path. && \text{(C-Prog)} \\
&\quad \inprog(x, \ovl a, a) \land \entails{\ovl b}{\subst{\ovl a}{x}{p}}
       \rightarrow \entails{\ovl b}{\subst{a}{x}{p}} \\
&\forall \ovl a: \Constrs, a, b: \Constr. && \text{(C-Weak)} \\
&\quad \entails{\ovl a}{b}
       \rightarrow \entails{a :: \ovl a}{b}\\
&\forall \ovl a, \ovl b: \Constrs, a: \Constr. && \text{(C-Perm)} \\
&\quad \forall b: \Constr.\ 
           b \in \ovl a \rightarrow b \in \ovl b
           \land
           \neg b \in \ovl a \rightarrow \neg b \in \ovl b\\
&\quad \land \entails{\ovl a}{a}
       \rightarrow \entails{\ovl b}{a}
\end{align*}
\caption{First-order Model Of Constraint Entailment}
\label{fig:axioms-naive}
\end{figure}\\
With the previous sort, predicate and function definitions
we can model the calculus rules.
A first-order representation of the calculus rules is given in \Cref{fig:axioms-naive}

We start with the structural rules for weakening, contraction and permutation
of the sequent calculus that are implicitly assumed in the $DC_C$ constraint system.
Since the entailment judgement of the $DC_C$ constraint system
has only a single constraint on its right-hand side
we only need to consider structural rules on the left-hand side.

Weakening allows the addition of arbitrary elements to a sequence.
The weakening rule in the $DC_C$ sequent calculus would look as follows:
\begin{prooftree}
\RightLabel{C-Weak}
\AxiomC{\entails{\ovl a}{b}}
\UnaryInfC{\entails{a, \ovl a}{b}}
\end{prooftree}
The rule states that the context of an entailment can always be restricted.
The first-order representation of this rule models this as follows:
For constraint lists \ovl a and constraints $a, b$,
if \entails{\ovl a}{b} holds then \entails{a :: \ovl a}{b} holds as well.
Here we restrict the context \ovl a by prepending an additional constraint $a$.

Contraction and permutation assure that the ordering of a sequent
and the existence of multiple occurences of the same element in a sequent
do not matter. Rule C-Perm covers both cases.
For constraint lists \ovl a, \ovl b and constraints $a$:
if \entails{\ovl a}{a} holds then \entails{\ovl b}{a} holds as well
if both lists \ovl a and \ovl b consist of the same elements.
This is checked with an additional quantifier ranging over constraints $b$.
If $b$ is an element of \ovl a then it must also be an element of \ovl b
and if $b$ is not in \ovl a then it must not be in \ovl b.
This ensures that the ordering of the context does not matter
and since we do not require the length of \ovl a and \ovl b to be equal
it does also allow for the removal of multiple occurences of the same element
as long as one remains.
\\\\
The modeling of the remaining rules follow the structure of the rules
specified in the $DC_C$ constraint system in \Cref{fig:dcc-constraint-entailment}.

Rules C-Ident and C-Refl are closure rules.
Rule C-Ident states that each constraint entails itself.
For all constraints $c$, $c$ is entailed by a context consisting
of the identical constraint $[c]$.
Rule C-Refl states that path equivalence is reflexive,
each path is equivalent to itself without any prerequisites.
For all constraints $p$, \pathEq{p}{p} is entailed by the empty context \nil.

Rule C-Class establishes the fact that an object created with
a constructor of a class is an instance of that class.
For all constraints \ovl a, paths $p$ and strings $C$:
\instOf{p}{c} is entailed by context \ovl a if
\instBy{p}{C} is entailed by context \ovl a.
We make sure that $C$ is a classname by requiring that \mIt{class(C)} holds.

Rule C-Cut is a standard rule of the sequent calculus.
It states that if a constraint $c$ is entailed by a context \ovl a
and a constraint $b$ is entailed by a context \ovl b containing $c$
then $b$ is also entailed by \ovl a and \ovl b without $c$,
meaning that we can replace $c$ in \ovl b with constraints \ovl a
entailing $c$.
In our model we use the list structure of our context to
make this replacement.
In the model the context of $b$ is a list $\ovl b = c :: \ovl{a'}$
and the context of $c$ is \ovl a.
We then concatenate \ovl a with \ovl{a'}, which is the remainder
of $b$ without $c$ to eliminate $c$ from the context of $b$.

Rule C-Subst establishes that equivalent paths can be substituted
within constraints.
A constraint $a_2$ is entailed by \ovl a if
a constraint $a_1$ is entailed by \ovl a
and \ovl a entails the equivalence of paths $p_2$ and $p_1$.
For this $a_2$ needs to be equal to the substitution of $a$ from $x$ to $p_2$
and $a_1$ needs to be equal to the substitution of $a$ from $x$ to $p_1$.
\mIt{variable(x)} ensures that $x$ is a valid variable name.

Rule C-Prog applies program entailments expressing inheritance to the calculus.
For constraint lists \ovl a, \ovl b, constraints $a$, strings $x$ and paths $p$:
if a program entailment \progEnt{x}{\ovl a}{a} exists in the program
and \ovl a is entailed by \ovl b then $a$ is entailed by \ovl b as well.
$x$ is eliminated in \ovl a and $a$ through substitution,
since $x$ is the bound variable name of the program entailment.
The existence check of the program entailment is modeled with the predicate
$\inprog(x, \ovl a, a)$.

\begin{example}[Constraint entailment]
\label{ex:application-entailment}
In this example we will show the application of the sequent calculus for constraint entailment
and compare it to the application of our model of the calculus.
We will show the resolution of the entailment \entails{\pathEq{x}{y}}{\pathEq{y}{x}}
stating the symmetry of path equivalence.\\
\\
We begin with the application of the sequent calculus:
\begin{prooftree}
\AxiomC{}
\RightLabel{C-Refl}
\UnaryInfC{\entails{\epsilon}{\pathEq{y}{y}}}
\RightLabel{C-Weak}
\UnaryInfC{\entails{\pathEq{x}{y}}{\pathEq{y}{y}}}
\UnaryInfC{\entails{\pathEq{x}{y}}{\subst{\pathEq{y}{x}}{x}{y}}}
\AxiomC{}
\RightLabel{C-Ident}
\UnaryInfC{\entails{\pathEq{x}{y}}{\pathEq{x}{y}}}
\RightLabel{C-Subst}
\BinaryInfC{\entails{\pathEq{x}{y}}{\subst{\pathEq{y}{x}}{x}{x}}}
\end{prooftree}
For showing that \entails{\pathEq{x}{y}}{\pathEq{y}{x}} holds we
apply rule C-Subst.
We choose \pathEq{x}{y} to be our equivalent paths used for substitution.
To show that \pathEq{x}{y} is entailed by \pathEq{x}{y} we use rule C-Ident
to close this branch.
What is left to show is that \pathEq{x}{y} entails \subst{\pathEq{y}{x}}{x}{y}.
We apply the substitution to gain the constraint \pathEq{y}{y}.
Finally we apply rule C-Weak to remove the information \pathEq{x}{y} from the context
and apply rule C-Refl to show the entailment \entails{\epsilon}{\pathEq{y}{y}}
and close the branch.
With this derivation we showed \entails{\pathEq{x}{y}}{\subst{\pathEq{y}{x}}{x}{x}}.
We apply the substitution to obtain our original goal \entails{\pathEq{x}{y}}{\pathEq{y}{x}}.\\
\\
We now try to find a solution for the same entailment in our model.
The entailment has two valid variable names $x$ and $y$ which we need to add in our model.
We do so by requiring \mIt{variable(\str x)} and \mIt{variable(\str y)}.
Since there are no valid classnames we do not have to add any requirements for the \mIt{class} predicate.

To show \entails{\pathEq{x}{y}}{\pathEq{y}{x}} we instantiate rule C-Subst.
Since we want to show the entailment we need it to appear on the right-hand side of the implication,
for this we instantiate \ovl a as $[\pathEq{x}{y}]$ and $a_2$ as \pathEq{y}{x}.
We then choose $x := \str x$, $p_1 := y$ and $p_2 := x$ for usage in the substitutions.
The instantiation of $x$ with $\str x$ ensures that \mIt{variable(x)} holds.
With the instantiations of $x$, $p_1$, $p_2$ and $a_2$ we get
\mIt{subst(a, \str x, y, a_1)} and \mIt{subst(a, \str x, x, \pathEq{y}{x})}.
We now need to choose $a$ and $a_2$ such that both \mIt{subst} requirements are fulfilled.
We instantiate $a := \pathEq{y}{x}$ and can resolve \mIt{subst(\pathEq{y}{x}, \str x, x, \pathEq{y}{x})}
requiring $\subst{\pathEq{y}{x}}{\str x}{x} = \pathEq{y}{x}$.
Applying the substitution shows that \pathEq{y}{x} is equal to \pathEq{y}{x}.
From \mIt{subst(\pathEq{y}{x}, \str x, y, a_1)} we know that
$a_1$ needs to be equal to $\subst{\pathEq{y}{x}}{\str x}{y} = \pathEq{y}{y}$
and we choose $a_1 := \pathEq{y}{y}$.
This leaves us with two new subgoals:
\begin{enumerate}
    \item \entails{[\pathEq{x}{y}]}{\pathEq{x}{y}} and
    \item \entails{[\pathEq{x}{y}]}{\pathEq{y}{y}}.
\end{enumerate}
To show (1) we instantiate rule C-Ident with $c := \pathEq{x}{y}$.\\
To show (2) we instantiate rule C-Weak with
$a := \pathEq{x}{y}$ and $\ovl a := \nil$ to match the context
and $b := \pathEq{y}{y}$ to match the constraint to be entailed by this context.
This leads to the new subgoal \entails{\nil}{\pathEq{y}{y}} which can be
resolved using rule C-Refl and instantiating $p := y$.\\
\\
By comparing the application of the sequent calculus with the application of our model,
we can see that both derivations are similar.
Both derivations use the same rule applications in the same order to obtain the result
that the entailment holds.
We can also observe that we chose the variable instantiations of rule C-Subst
in the application of our model to match the path equivalence and substitution used
in the application of the sequent calculus.
\end{example}

%%% Local Variables: 
%%% mode: latex
%%% TeX-master: "../thesis"
%%% End: 
