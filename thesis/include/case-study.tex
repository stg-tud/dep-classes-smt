\chapter{Runtime Evaluation} % TODO: other title pls?
In this chapter we evaluate the implementation of the $DC_C$ calculus.
To do this, we extend the natural numbers program with the support of
abstract syntax of expressions.

\section{Program Definition}
\subsection{Natural Numbers}
First, we recap the implementation of the natural numbers:
\begin{align*}
&\constr{\texttt{Zero}}{x}{\epsilon}\\
&\progEnt{x}{\instOf{x}{\texttt{Zero}}}{\instOf{x}{\texttt{Nat}}}\\
&\constr{\texttt{Succ}}{x}{\instOf{x.p}{\texttt{Nat}}}\\
&\progEnt{x}{\instOf{x}{\texttt{Succ}}, \instOf{x.p}{\texttt{Nat}}}{\instOf{x}{\texttt{Nat}}}\\
&\mDecl{\texttt{prev}}{x}{\instOf{x}{\texttt{Nat}}}{\type{y}{\instOf{y}{\texttt{Nat}}}}\\
&\mImpl{\texttt{prev}}{x}{\instOf{x}{\texttt{Zero}}}{\type{y}{\instOf{y}{\texttt{Nat}}}}{x}\\
&\mImpl{\texttt{prev}}{x}{\instOf{x}{\texttt{Succ}}, \instOf{x.p}{\texttt{Nat}}}{\type{y}{\instOf{y}{\texttt{Nat}}}}{x.p}
\end{align*}
%
A natural number (\texttt{Nat}) is either the number zero (\texttt{Zero})
or a successor of a natural number (\texttt{Succ}).
\texttt{Succ} has a field $p$ of class \texttt{Nat}.
The function \texttt{prev} computes the previous element of a natural number.
The previous number of zero is defined as zero.
We changed the implementation for this case.
The program defined in \Cref{ex:dcc-naturalnumbers} created a new
instance of class \texttt{Zero}.
The new implementation returns the parameter instead,
as it is constrained to be an instance of \texttt{Zero}.

\subsection{Numeric Expressions}
We define a program for the abstract syntax of numeric expressions:
%
\begin{align*}
&\constr{\texttt{Lit}}{x}{\instOf{x.value}{\texttt{Nat}}}\\
&\progEnt{x}{\instOf{x}{\texttt{Lit}}, \instOf{x.value}{\texttt{Nat}}}{\instOf{x}{\texttt{Exp}}}\\
&\constr{\texttt{Plus}}{x}{\instOf{x.l}{\texttt{Exp}}, \instOf{x.r}{\texttt{Exp}}}\\
&\progEnt{x}{\instOf{x}{\texttt{Plus}}, \instOf{x.l}{\texttt{Exp}}, \instOf{x.r}{\texttt{Exp}}}{\instOf{x}{\texttt{Exp}}}
\end{align*}
%
The constructor \texttt{Lit} defines a field \mIt{value},
representing numeric literals with the value \mIt{value}.
The constructor \texttt{Plus} defines fields $l$ and $r$,
representing the addition of $r$ and $l$.
The two constraint entailment declarations express
the inheritance relation of
\texttt{Lit} and \texttt{Plus} to \texttt{Exp}.

\subsection{Numeric Expression Evaluation}
We extend the abstract syntax of numeric expressions
with a method to evaluate expressions:
%
\begin{align*}
&\mDecl{\texttt{eval}}{x}{\instOf{x}{\texttt{Exp}}}{\type{y}{\instOf{y}{\texttt{Exp}}}}\\
&\mImpl{\texttt{eval}}
       {x}
       {\instOf{x}{\texttt{Lit}}, \instOf{x.value}{\texttt{Nat}}}
       {\type{y}{\instOf{y}{\texttt{Exp}}}}
       {x}\\
%&\mImpl{\texttt{eval}}
%       {x}
%       {\instOf{x}{\texttt{Plus}},
%        \instOf{x.l}{\texttt{Lit}},
%        \instOf{x.r}{\texttt{Lit}},
%        \instOf{x.l.value}{\texttt{Nat}},
%        \instOf{x.r.value}{\texttt{Zero}}}
%       {\type{y}{\instOf{y}{\texttt{Exp}}}}
%       {x.l}\\
&\texttt{eval}(x. \instOf{x}{\texttt{Plus}},
                  \instOf{x.l}{\texttt{Lit}},
                  \instOf{x.r}{\texttt{Lit}},
                  \instOf{x.l.value}{\texttt{Nat}},\\
&\qquad\qquad\qquad\qquad\qquad
                  \instOf{x.r.value}{\texttt{Zero}} )
    : \type{y}{\instOf{y}{\texttt{Exp}}} := \\
&\quad x.l\\
%&\mImpl{\texttt{eval}}
%       {x}
%       {\instOf{x}{\texttt{Plus}},
%        \instOf{x.l}{\texttt{Lit}},
%        \instOf{x.r}{\texttt{Lit}},
%        \instOf{x.l.value}{\texttt{Nat}},
%        \instOf{x.r.value}{\texttt{Succ}},
%        \instOf{x.r.value.p}{\texttt{Nat}}}
%       {\type{y}{\instOf{y}{\texttt{Exp}}}}{}\\
%         &\quad\texttt{eval}(
%           \new \texttt{Plus}(\\
%             &\qquad\pathEq{l}{
%               \objConstr{\texttt{Lit}}
%                         {\pathEq{value}{ \objConstr{\texttt{Succ}}{\pathEq{p}{x.l.value}}}}},\\
%             &\qquad\pathEq{r}{
%               \objConstr{\texttt{Lit}}
%                         {\pathEq{value}{\texttt{prev}(x.r.\mIt{value})}}}
%           )
%         )\\
&\texttt{eval}(x. \instOf{x}{\texttt{Plus}},
                  \instOf{x.l}{\texttt{Lit}},
                  \instOf{x.r}{\texttt{Lit}},
                  \instOf{x.l.value}{\texttt{Nat}},\\
&\qquad\qquad
                  \instOf{x.r.value}{\texttt{Succ}},
                  \instOf{x.r.value.p}{\texttt{Nat}}
              ): \type{y}{\instOf{y}{\texttt{Exp}}} :=\\
&\quad\texttt{eval}(
           \new \texttt{Plus}(\\
             &\qquad\pathEq{l}{
               \objConstr{\texttt{Lit}}
                         {\pathEq{value}{ \objConstr{\texttt{Succ}}{\pathEq{p}{x.l.value}}}}},\\
             &\qquad\pathEq{r}{
               \objConstr{\texttt{Lit}}
                         {\pathEq{value}{\texttt{prev}(x.r.\mIt{value})}}}
           )
         )\\
&\mImpl{\texttt{eval}}
       {x}
       {\instOf{x}{\texttt{Plus}},
        \instOf{x.l}{\texttt{Exp}},
        \instOf{x.r}{\texttt{Plus}},
        \instOf{x.r.l}{\texttt{Exp}},
        \instOf{x.r.r}{\texttt{Exp}}}
       {\type{y}{\instOf{y}{\texttt{Exp}}}}{}\\
       &\quad\texttt{eval}(
         \objConstr{\texttt{Plus}}
                   {
                     \pathEq{l}{x.l},
                     \pathEq{r}{\texttt{eval(x.r)}}
                   }
       )\\
&\mImpl{\texttt{eval}}
       {x}
       {\instOf{x}{\texttt{Plus}},
        \instOf{x.l}{\texttt{Plus}},
        \instOf{x.r}{\texttt{Exp}},
        \instOf{x.l.l}{\texttt{Exp}},
        \instOf{x.l.r}{\texttt{Exp}}}
       {\type{y}{\instOf{y}{\texttt{Exp}}}}{}\\
       &\quad\texttt{eval}(
         \objConstr{\texttt{Plus}}
                   {
                     \pathEq{l}{\texttt{eval(x.l)}},
                     \pathEq{r}{x.r}
                   }
       )
\end{align*}
We abstractly declare method \texttt{eval}
to evaluate an expression to an expression,
where literals are the values of the expression language.
We use the constraints on the parameter to
model a behavior similar to pattern matching.
We overload the method for each case
we want to match and set the requirements
on the parameter accordingly.

\subsubsection{Literals}
Since literals are considered to be values,
we evaluate an input $x$ conforming to the requirements
\instOf{x}{\texttt{Lit}} and \instOf{x.value}{\texttt{Nat}} to itself.
The requirements make sure that the input is a proper literal.

\subsubsection{Addition}
We implement the evaluation of \texttt{Plus} expressions
through evaluation on the right-hand side.



%%% Local Variables: 
%%% mode: latex
%%% TeX-master: "../thesis"
%%% End: 
