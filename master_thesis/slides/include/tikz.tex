
\usetikzlibrary{positioning,chains,fit,calc,arrows,decorations.pathmorphing}
\usetikzlibrary{shapes.arrows,shapes.geometric,shapes.symbols}


%%%%%%%
%% tikz stuff
%%%%%%%


\tikzstyle{invisible} = []

\tikzstyle{model}
  = [ shape=rectangle
    , draw
    ]
\tikzstyle{transformation}
  = [ model
    , shape=single arrow
    , draw
    ]

\tikzstyle{meta} = [ double ]
\tikzstyle{generated} = [ dashed ]

\tikzstyle{mopdependency}
  = [ -stealth'
    , semithick
    ]
\tikzstyle{instance}
  = [ mopdependency
    , -open triangle 60
    ]

% Draws the outline of a bend arrow around a rectangle.
% 1. Parameter: name of the rectangle that contains the text
% 2. Parameter: head extend (use 0.25cm to fit with default tikz style)
% 3. Parameter: arc radius (again, 0.25cm seems to work well)
%
% This produces a path, so use it as follows:
%
% \draw[options here] \bendArrow{name of rectangle}{.25cm}{.25cm};
\newcommand{\bendArrow}[3]{
  let \p{content size} = ($(#1.north east) - (#1.south west)$) in
  let \n{tip length 1} = {veclen(0.5 * \y{content size}, 0.5 * \y{content size})} in
  let \n{tip length 2} = {veclen(#2,#2)} in

  (#1.north east)
  -- ++ (315 : \n{tip length 1})
  -- ++ (225 : \n{tip length 1} + \n{tip length 2})
  -- ++ (0, #2)
  -- (#1.south west)
  arc (270 : 180 : \y{content size} + #3)
  -- ++ (\y{content size}, 0)
  arc (180 : 270 : #3)
  (#1.north east)
  -- ++ (135 : \n{tip length 2})
  -- ++ (0, -#2)
  -- (#1.north west)}


%%%%%%%%%%%%%%%%%
%% diagram styles
%%%%%%%%%%%%%%%%%

\tikzstyle{document}
  = [ shape=rectangle
    , draw
    , minimum height=1.5em
    , text width=5em
    , text centered
    ]

\tikzstyle{component}
  = [ font=\scriptsize\it
    ]

\tikzstyle{code}
  = [ shape=rectangle
    , draw
    ]

\tikzstyle{process}
  = [ shape=single arrow
    , single arrow head extend=.75em
    , single arrow head indent=.25em
    , minimum width=3em
    , draw
    ]

\tikzstyle{point}
  = [ coordinate
    , minimum width=1em
    ]

\tikzstyle{flow diagram}
  = [ start chain
    , node distance=1em
    , every node/.style={on chain}
    ]

\tikzstyle{ast}
  = [ node distance=1em and 0.38em
    , every node/.style=ast node
    ]

\tikzstyle{ast node}
  = [ shape=circle
    , minimum size=0.5em
    , inner sep=0
    , fill
    ]

\tikzstyle{dependency}
  = [ dashed
    , -stealth'
    ]

\tikzstyle{red}
  = [ shape=rectangle
    , color=red
    ]

\tikzstyle{blue}
  = [ shape=circle
    , color=blue
    ]

\tikzstyle{green}
  = [ shape=diamond
    , color=green!80!black
    , minimum size=0.7em
    ]

\tikzstyle{note}
  = [ font=\scriptsize\it
    ]
    
\tikzstyle{zoomed arrow}
  = [ solid
    , decorate
    , decoration=snake
    , -stealth'
    ]

\tikzstyle{zoom}
  = [ dashed
    ]

\tikzstyle{uml class}
  = [ shape=rectangle
    , draw
    , font=\sf
    ]

\tikzstyle{uml package}
  = [ uml class
    , inner sep=1em
    ]

\tikzstyle{uml dependency}
  = [ dependency, 
    , dashed
%    , thick
    ]


\newcommand{\bluenode}{\tikz \node [ast node, blue, color=blue, text width=] {};}
\newcommand{\rednode}{\tikz \node [ast node, red, color=red, text width=] {};}
\newcommand{\greennode}{\tikz \node [ast node, green, color=green!80!black, text width=] {};}


\tikzstyle{double arrow}
  = [ shape=double arrow
    , double arrow head extend=.75em
    , double arrow head indent=.25em
    , minimum width=3em
    , draw
    , font=\sf
    ]

%% name graphs

\newdimen\nodedistance
\nodedistance=4em

\tikzstyle{name graph}
  = [ node distance=\nodedistance
    , every node/.style={namenode}
    , every path/.style={ref}
    ]


\tikzstyle{namenode}
  = [ circle
    , thick
    , anchor=center
    , draw
    , minimum size=2em
    , inner sep=2pt
    ]

\tikzstyle{synthesized}
  = [ namenode
    , fill=gray!45
    ]

\tikzstyle{ref}
  = [ -stealth'
    , semithick
    , draw
    ]

\tikzstyle{badref}
  = [ ref
    , dashed
    ]

%% expression trees

\newdimen\nodedistance
\nodedistance=4em

\tikzstyle{exp tree}
  = [ node distance=\nodedistance
    , every node/.style={exp node}
    , every path/.style={link}
    , execute at begin node={\strut}
    ]

\tikzstyle{exp node}
  = [ anchor=center
    ]
\tikzstyle{link}
  = [ semithick
    , draw
    ]

\tikzstyle{ctx node}
  = [ node distance = -.3em,
    ]
\tikzstyle{ctx edge}
  = [ color = red
    , -stealth'
    ]

\tikzstyle{type node}
  = [ node distance = -.5em,
    ]
\tikzstyle{type edge}
  = [ color = blue
    , -stealth'
    ]

%%%%%%%%%%%%%%%%%%%%%%%%%%%
% Pretty printing with tikz
%
% configuration

%%%%%%
%% TikZ and lstlistings
%%%%%%


% remembers a position on the page as a tikz coordinate
\newcommand{\coord}[1]{\tikz[remember picture] \coordinate (#1);}

% translation from baseline to upper border of box
\newcommand{\distanceTop}{7.15pt}

% translation from baseline to lower border of box
\newcommand{\distanceBottom}{-2.55pt}

% translation from baseline to left border of box
\newcommand{\distanceLeft}{-0.5pt}

% translation from baseline to right border of box
\newcommand{\distanceRight}{0.5pt}


% draws a rectangle around 2 points on the page
%
% optional parameter: TikZ style for the line
% 1. mandatory parameter: upper left corner, at text baseline height
% 2. mandatory parameter: lower right corner, at text baseline height
% 3. mandatory parameter: extra margin in all directions
\newcommand{\drawrect}[4][]{\begin{tikzpicture}[remember picture, overlay]
\draw[layout box, #1]
  ($(#2) +(\distanceLeft, \distanceTop) + (-#4, #4)$) rectangle
  ($(#3) + (\distanceRight, \distanceBottom) + (#4, -#4)$);
\end{tikzpicture}}

% draws a rectangle around 2 points on the page
%
% optional parameter: TikZ style for the line
% 1. mandatory parameter: left-hand-side of the first line, at text baseline height
% 2. mandatory parameter: a point on the left side of the rectangle
% 3. mandatory parameter: lower right corner, at text baseline height
% 4. mandatory parameter: extra margin in all directions
\newcommand{\drawbox}[5][]{\begin{tikzpicture}[remember picture, overlay]
\draw[layout box, #1]
  ($(#2) + (\distanceLeft, \distanceTop) + (-#5, #5)$) --
  ($(#4 |- #2) + (\distanceRight, \distanceTop) + (#5, #5)$) --
  ($(#4) + (\distanceRight, \distanceBottom) + (#5, -#5)$) --
  ($(#3 |- #4) + (\distanceLeft, \distanceBottom) + (-#5, -#5)$) --
  ($(#3 |- #2) + (\distanceLeft, \distanceBottom) + (-#5, -#5)$) --
  ($(#2) + (\distanceLeft, \distanceBottom) + (-#5, -#5)$) --
  cycle;
\end{tikzpicture}}

% draws a small \ppBox with three rows around the current point
% #1 = first.col
% #2 = left.col
% #3 = right.col
% #4 = last.col
\newcommand{\drawMiniBox}[4]{
  +(#2 * .2em, 0) --
  +(#2 * .2em, 1ex) --
  +(#1 * .2em, 1ex) --
  +(#1 * .2em, 1.5ex) --
  +(#3 * .2em, 1.5ex) --
  +(#3 * .2em, .5ex) --
  +(#4 * .2em, .5ex) --
  +(#4 * .2em, 0) --
  cycle
}

% A small box (for use inside text)
% optional argument = path options
% #1 = first.col
% #2 = left.col
% #3 = right.col
% #4 = last.col
\newcommand{\minibox}[5][]{%
  \begin{tikzpicture}[baseline=0pt]
  \path [mini layout box, #1]
    (0, 0) \drawMiniBox{#2}{#3}{#4}{#5};
  \end{tikzpicture}}

\newcommand{\miniboxA}{\minibox{-1}{0}{3}{3}}
\newcommand{\miniboxB}{\minibox{0}{0}{3}{3}}
\newcommand{\miniboxC}{\minibox{1}{0}{3}{3}}

% A small rectangle (for use inside text)
\newcommand{\minirect}{\hbox to 9pt{\drawrect[thin,scale=0.3,black]{0pt, 15pt}{26pt, -5pt}{0pt}}}


\tikzstyle{layout box}
  = [ blue
    % , semitransparent
    , semithick
    , draw
    ]

\tikzstyle{mini layout box}
  = [ black
    , very thin
    , draw
    ]

\tikzstyle{annotation}
  = [ font=\small\it
    ]

\tikzstyle{code annotation}
  = [ font=\small\it
    ]


% the font used for pretty printing
\newcommand{\selectCodeFont}{\sf\small}

% the font used for highlighting keywords
\newcommand{\selectKeywordFont}{\bfseries}

% the font used for highlighting string literals
\newcommand{\selectStringLitFont}{\tt}

% the font used for highlighting identifiers
\newcommand{\selectIdentifierFont}{\relax}

% the font used for highlighting operators
\newcommand{\selectOperatorFont}{\tt}

% the text height of token nodes
\newcommand{\tokenHeight}{2ex}

% the text depth of token nodes
\newcommand{\tokenDepth}{.5ex}

%%%%%%%%%%%%%%%%%%%%%%%%%%%
% Pretty printing with tikz
%
% tikz layer

% style for paths or scopes that do pretty printing
\tikzstyle{pretty print}
  = [ start chain=going base right
    , text height=\tokenHeight
    , text depth=\tokenDepth
    , inner sep=0
    , node distance=0em
    ]

% helper style for nodes that start a new line
% (automatically applied by style 'new line' below)
\tikzstyle{new line/helper}
  = [ on chain
    , anchor=north west,
      at={($(#1.west |- \tikzchainprevious.base) + (0, -1.1\baselineskip)$)}
    ]

\tikzstyle{new line tight/helper}
  = [ on chain
    , anchor=north west,
      at={($(#1.west |- \tikzchainprevious.base) + (0, -0.9\baselineskip)$)}
    ]


% style for nodes that start a new line
\tikzstyle{new line}[\tikzchainprevious]
  = [ on chain=placed {new line/helper=#1}
    ]

% style for nodes that start a new line
\tikzstyle{new line tight}[\tikzchainprevious]
  = [ on chain=placed {new line tight/helper=#1}
    ]

% helper style for nodes that tab forward to a labeled position
% (automatically applied by style 'tab forward' below)
\tikzstyle{tab forward/helper}
  = [ on chain
    , anchor=north west,
      at={(#1 |- \tikzchainprevious.base)}
    ]

% style for nodes that tab forward to a labeled position
\tikzstyle{tab forward}
  = [ on chain=placed {tab forward/helper=#1}
    ]

% style for nodes that indent
\tikzstyle{indent}[1em]
  = [ on chain=placed {base right=#1 of \tikzchainprevious}
    ]

% style for nodes that contain an identifier token
\tikzstyle{identifier}
  = [ on chain
    , font=\selectIdentifierFont
    ]

% style for nodes that contain an operator token
\tikzstyle{operator}
  = [ on chain
    , font=\selectOperatorFont
    ]

% style for nodes that contain a keyword token
\tikzstyle{keyword}
  = [ on chain
    , font=\selectKeywordFont
    , text=keyword
    ]

% style for nodes that contain a string literal
\tikzstyle{stringlit}
  = [ on chain
    , font=\selectStringLitFont
    , text=blue
    ]

%%%%%%%%%%%%%%%%%%%%%%%%%%%
% Pretty printing with tikz
%
% \ppBox layer

\newcommand{\Empty}{}
\newcommand{\ignore}[1]{\relax}

\makeatletter
\newcommand{\ppBox}[2][]{{
  % insert keyword token
  \newcommand{\KW}[1]{
    \node[keyword] {##1};
    \HandleToken{\tikzchaincurrent}}

  % insert identifier token
  \newcommand{\ID}[1]{
    \node[identifier] {##1};
    \HandleToken{\tikzchaincurrent}}

  % insert operator token
  \newcommand{\OP}[1]{
    \node[operator] {##1};
    \HandleToken{\tikzchaincurrent}}

  % insert whitespace token
  \newcommand{\SP}{
    \node[on chain] { };
    \HandleInsensitiveToken{\tikzchaincurrent}}

  % label a position (with a tikz node name)
  \newcommand{\LB}[1]{
    \coordinate[on chain] (##1);}

  % tab forward to a labeled position
  \newcommand{\TB}[1]{
    \coordinate[tab forward=##1];
  }

  % insert a string literal
  \newcommand{\STR}[1]{
    \node[stringlit] {##1};
    \HandleToken{\tikzchaincurrent}}

  % insert a line break
  \renewcommand{\\}{
    \coordinate[new line=\NewlineNode];
    \let\HandleToken=\HandleTokenSubsequentLine}

  \newcommand{\newlineTight}{
    \coordinate[new line tight=\NewlineNode];
    \let\HandleToken=\HandleTokenSubsequentLine}

  % the name of the first token
  \let\FirstToken=\Empty
  \newcommand{\AdjustFirst}[1]{
    \ifx\FirstToken\Empty
    \edef\FirstToken{##1}
    \fi
  }

  % the name of the last token
  \let\LastToken=\Empty
  \newcommand{\AdjustLast}[1]{
    \edef\LastToken{##1}
  }

  % the name of the left-most token
  % (except for tokens in the first line)
  \let\LeftToken=\Empty
  \newcommand{\AdjustLeft}[1]{
    \ifx\LeftToken\Empty
      \edef\LeftToken{##1}
    \else
      \pgf@process{\pgfpointanchor{##1}{west}}
      \setlength{\pgf@xa}{\pgf@x}
      \pgf@process{\pgfpointanchor{\LeftToken}{west}}
      \setlength{\pgf@xb}{\pgf@x}
      \ifdim\pgf@xa<\pgf@xb
        \edef\LeftToken{##1}
      \fi
    \fi
  }

  % the name of the rightmost token
  \let\RightToken=\Empty
  \newcommand{\AdjustRight}[1]{
    \ifx\RightToken\Empty
      \edef\RightToken{##1}
    \else
      \pgf@process{\pgfpointanchor{##1}{east}}
      \setlength{\pgf@xa}{\pgf@x}
      \pgf@process{\pgfpointanchor{\RightToken}{east}}
      \setlength{\pgf@xb}{\pgf@x}
      \ifdim\pgf@xa>\pgf@xb
        \edef\RightToken{##1}
      \fi
    \fi
  }

  % this is called for all layout-sensitive tokens in the first
  % line of a pretty printing group
  \newcommand{\HandleTokenFirstLine}[1]{
    \AdjustFirst{##1}
    \AdjustRight{##1}
    \AdjustLast{##1}
  }

  % this is called for all layout-sensitive tokens in subsequent
  % lines of a pretty printing group
  \newcommand{\HandleTokenSubsequentLine}[1]{
    \AdjustLeft{##1}
    \AdjustRight{##1}
    \AdjustLast{##1}}

  % this is called for layout-insensitive tokens in a pretty
  % printing group (such as whitespace, comments, and code that
  % uses explicit layout)
  \newcommand{\HandleInsensitiveToken}[1]{
  }

  % current token handler. will be set to one of:
  %  \HandleTokenFirstLine
  %  \HandleTokenSubsequentLine
  %  \HandleInsensitiveToken
  \newcommand{\HandleToken}[1]{}

  \newcommand{\ppSubBox}[2][]{
    \coordinate[on chain];
    {
      \let\NewlineNode\tikzchaincurrent

      % reset the token registers
      \let\FirstToken=\Empty
      \let\LastToken=\Empty
      \let\LeftToken=\Empty
      \let\RightToken=\Empty

      % start in first line
      \let\HandleToken=\HandleTokenFirstLine

      % typeset content
      ##2

      % draw boundary
      \ifx\LeftToken\Empty
        \path [##1]
          (\FirstToken.north west) rectangle
          (\LastToken.south east);
      \else
        \path [##1]
          (\RightToken.east |- \FirstToken.north) --
          (\FirstToken.north west) --
          (\FirstToken.south west) --
          (\LeftToken.west |- \FirstToken.south) --
          (\LeftToken.west |- \LastToken.south) --
          (\LastToken.south east) --
          (\LastToken.north east) --
          (\RightToken.east |- \LastToken.north) --
          cycle;
      \fi

      % remember the token registers outside the group
      \global\let\InnerFirstToken=\FirstToken
      \global\let\InnerLastToken=\LastToken
      \global\let\InnerLeftToken=\LeftToken
      \global\let\InnerRightToken=\RightToken

      % remember the state outside the group
      \global\let\InnerHandleToken=\HandleToken
    }
    % handle the significant inner tokens
    % and significant inner linebreaks
    \HandleToken{\InnerFirstToken}
    \ifx\HandleToken\HandleTokenFirstLine
    \let\HandleToken=\InnerHandleToken
    \fi
    \ifx\InnerLeftToken\Empty
    \else
    \HandleToken{\InnerLeftToken}
    \fi
    \HandleToken{\InnerRightToken}
    \HandleToken{\InnerLastToken}
  }

  % indent a ppBox
  \newcommand{\IN}[1]{\coordinate[indent]; ##1}
  \newcommand{\DE}[1]{\coordinate[indent=-1em]; ##1}

  \let\ppBox=\ppSubBox
  \ppBox[#1]{#2}}}
\makeatother




%%% Local Variables: 
%%% mode: latex
%%% TeX-master: "../thesis"
%%% End: 
