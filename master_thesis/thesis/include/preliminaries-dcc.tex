\subsection{$DC_C$ Calculus}
\label{sec:dcc}
The $vc^n$ calculus does not support abstract classes and methods.
The $DC_C$ calculus~\cite{vaidas:thesis} extends $vc^n$ with support for abstract classes and methods
and symmetric method dispatch.
$DC_C$ encodes dependent classes with a constraint system,
which is used in both the static and dynamic semantics.
The main relation of the $DC_C$ calculus is constraint entailment.
%replacing $vc^n$'s relations for type equivalence, subtyping, static and dynamic dispatch.

The runtime structure is heap-based,
with explicit object identities
and relationships between objects based on these identities.
Expressions evaluate to an identifier pointing to an object in the heap.
Heaps preserve object identity and enable shared references to objects.
The heap provides a direct interpretation for equivalent paths,
two paths are equivalent if they point to the same object at runtime.
Heaps can be easily translated to a set of constraints describing its objects and their relations,
which enables the usage of the constraint system for dynamic dispatch and expression typing.
%- DCc extends $vc^n$ with abstract methods and classes and symmetric method dispatch
%- gave up unification of methods and classes
%- runtime structure: heap
%- heap structure preserves object identity, enables shared references
%- heap provides direct interpretation for equivalent paths
%    - paths are equiv if they point to the same object at runtime
%- heaps can be easily translated to a set of constraints describing its objects and their relations
%    - enables using constraint system for dynamic dispatch and expression typing
\\ \\
Here we show the calculus by Vaidas Gasiūnas.
\subsubsection{Syntax}
% begin syntax figure
\begin{figure}[t]
\begin{align*}
\mathit{Program} &::= \ovl{Decl} \\
\mathit{Decl} &::= \constr{C}{x}{\overline{Constr}} % C(x. $\overline{Constr}$)
                 \ |\ \progEnt{x}{\overline{Constr}}{Constr} \\
              &\ \ |\ \mDecl{m}{x}{\overline{Constr}}{Type} % m(x. $\overline{Constr}$): Type
                 \ |\ \mImpl{m}{x}{\overline{Constr}}{Type}{Expr} \\ % m(x. $\overline{Constr}$): Type := Expr
\mathit{Type} &::= \type{x}{\ovl{Constr}} \\
\mathit{Constr} &::= \pathEq{Path}{Path} % Path $\equiv$ Path
                \ |\ \instOf{Path}{C} % Path :: C
                \ |\ \instBy{Path}{C} \\ % Path.\textbf{cls} $\equiv$ C
\mathit{Path} &::= x \ |\ \mathit{Path.f}\\
\mathit{Expr} &::= x
              \ |\ \mathit{Expr.f}
              \ |\ \newInst{C}{\overline{f}}{\overline{Expr}} % \textbf{new} C($\overline{f}$ $\equiv$ $\overline{Expr}$)
              \ |\ m(\mathit{Expr})
\end{align*}
%\hrule
\begin{align*}
MType(m, x, y) &= \{ \langle\overline{a}, \overline{b}\rangle | (m(x. \overline{a}): \type{y}{\ovl{b}}...) \in P \} \\
MImpl(m, x) &= \{\langle\overline{a}, e\rangle | (m(x. \overline{a}): \type{y}{\ovl{b}} := e) \in P \}
\end{align*}

$x, y \in$ variable names\\
$f \in$ field names\\
$C \in$ class names\\
$m \in$ method names
\caption{Syntax}
\label{fig:dcc-syntax}
\end{figure}
% end syntax figure
The syntax of $DC_C$ is given in \Cref{fig:dcc-syntax}.
Types are lists of constraints to be satisfied by their instances.
Types have the form \type{x}{\ovl{a}}, where $x$ is a bound variable
and \ovl{a} is a list of constraints on $x$.
An object belongs to a type if it fulfills its constraints.

Constraints of the form $p \equiv q$ express that two paths $p$ and $q$ are equivalent
and paths are considered to be equivalent if they refer to the same object at runtime.
$p :: C$ specifies that path $p$ refers to an instance of class $C$.
The stronger form $p.\textbf{cls} \equiv C$ denotes that
path $p$ refers to an object instantiated by a constructor of class $C$,
excluding indirect instances of $C$ inferred through inheritance rules.

A Program $P$ consists of a list of declarations $D$.
Possible declarations are constructor declarations,
abstract method declarations, method implementations and constraint entailment rules.

A Path expression can be a variable $x$ or navigation over fields starting from a variable e.g. $x.f$.

Expressions can be variables, field access, object construction and method invocation.
Field assignments are not supported since the calculus is functional.
%
\begin{example}[Natural Numbers]
\label{ex:dcc-naturalnumbers}
\begin{align*}
&\constr{\texttt{Zero}}{x}{\epsilon}\\
&\progEnt{x}{\instOf{x}{\texttt{Zero}}}{\instOf{x}{\texttt{Nat}}}\\
&\constr{\texttt{Succ}}{x}{\instOf{x.p}{\texttt{Nat}}}\\
&\progEnt{x}{\instOf{x}{\texttt{Succ}}, \instOf{x.p}{\texttt{Nat}}}{\instOf{x}{\texttt{Nat}}}\\
&\mDecl{\texttt{prev}}{x}{\instOf{x}{\texttt{Nat}}}{\type{y}{\instOf{y}{\texttt{Nat}}}}\\
&\mImpl{\texttt{prev}}{x}{\instOf{x}{\texttt{Zero}}}{\type{y}{\instOf{y}{\texttt{Nat}}}}{\newInstNoArgs{\texttt{Zero}}}\\
&\mImpl{\texttt{prev}}{x}{\instOf{x}{\texttt{Succ}}, \instOf{x.p}{\texttt{Nat}}}{\type{y}{\instOf{y}{\texttt{Nat}}}}{x.p}
\end{align*}
\end{example}
%
\Cref{ex:dcc-naturalnumbers} shows a program defining natural numbers.
The program has constructors \texttt{Zero} representing zero
and \texttt{Succ} representing the successor of its field $p$.
The inheritance of \texttt{Zero} and \texttt{Succ} to \texttt{Nat}
is expressed through the two constraint entailment rules.
The method \texttt{prev} is abstractly declared to
be a function from \texttt{Nat} to \texttt{Nat}.
It has two implementations,
one for \texttt{Zero} and one for \texttt{Succ}.

\subsubsection{Constraint System}
% begin constraint system figure
\begin{figure}[t]
% C-Ident
\begin{prooftree}
\AxiomC{}
\RightLabel{(C-Ident)}
\UnaryInfC{\entails{a}{a}}
\end{prooftree}
% C-Refl
\begin{prooftree}
\AxiomC{}
\RightLabel{(C-Refl)}
\UnaryInfC{\entails{\epsilon}{\pathEq{p}{p}}}
\end{prooftree}
% C-Class
\begin{prooftree}
\AxiomC{\entails{\overline{a}}{\instantiatedBy{p}{C}}}
\RightLabel{(C-Class)}
\UnaryInfC{\entails{\overline{a}}{\instanceOf{p}{C}}}
\end{prooftree}
% C-Cut
\begin{prooftree}
\AxiomC{\entails{\overline{a}}{c}}
\AxiomC{\entails{\overline{a'}, c}{b}}
\RightLabel{(C-Cut)}
\BinaryInfC{\entails{\overline{a},\overline{a'}}{b}}
\end{prooftree}
% C-Subst
\begin{prooftree}
\AxiomC{\entails{\overline{a}}{a_{\sub{x}{p}}}}
\AxiomC{\entails{\overline{a}}{\pathEq{p'}{p}}}
\RightLabel{(C-Subst)}
\BinaryInfC{\entails{\overline{a}}{a_{\sub{x}{p'}}}}
\end{prooftree}
% C-Prog
\begin{prooftree}
\AxiomC{$(\progEnt{x}{\overline{a}}{a}) \in P$}
\AxiomC{\entails{\overline{b}}{\overline{a}_{\sub{x}{p}}}}
\RightLabel{(C-Prog)}
\BinaryInfC{\entails{\overline{b}}{a_{\sub{x}{p}}}}
\end{prooftree}
\caption{Constraint Entailment}
\label{fig:dcc-constraint-entailment}
\end{figure}
% end constraint system figure
The constraint system is given in the style of the sequent calculus
and the rules for the constraint system are specified in \Cref{fig:dcc-constraint-entailment}.
The sequent \entails{\ovl{a}}{a} is interpreted as constraint entailment:
constraints \ovl{a} entail constraint $a$.
The constraints on the left-hand side are referred to as the context
and the constraint on the right-hand side as the constraint entailed by the context.
The notion \entails{\ovl{a}}{\ovl{b}} is used differently than in the sequent calculus.
For $DC_C$ it is used as a shortcut for a list of judgments
\entails{\ovl{a}}{b_i} for each $b_i \in \ovl{b}$,
meaning that all $b_i$ are entailed by \ovl{a}.

Rules C-Ident and C-Cut are standard rules of the sequent calculus,
the remainder of the rules are specific to the programming language.
The standard structural rules of the sequent calculus allowing
permutation, weakening and contraction of the context
are implicitly assumed to be specified.

The properties of path equivalence are specified with rules C-Refl and C-Subst.
Rule C-Refl establishes reflexivity of path equivalence.
Rule C-Subst specifies that paths can be substituted with equivalent paths
at any position of any other constraint.
Other typical rules of equivalence as symmetry and transitivity can be
derived from these rules.

Rule C-Class specifies that a direct instance of a class
is an instance of that class, describing that
\instOf{p}{C} is a weaker relationship than \instBy{p}{C}.

With Rule C-Prog it is possible to specify new axioms for the constraint system in programs.
This is used to express inheritance declarations between dependent classes:
the constraint at the right-hand side of the implication must be \instOf{x}{p},
where $x$ is the bound variable of the rule.
The rule is restricted to avoid an undecidable constraint system
and the restrictions are specified by the well-formedness rule WF-RD in \Cref{fig:dcc-wf}.

\subsubsection{Operational Semantics}
The operational semantics is given in \Cref{fig:dcc-opsemantics}.
It is defined as a small-step reduction relation of a heap and an expression.
A heap is a list of mappings from variables to objects.
Each object is specified by a class and a list of the values of its fields,
which are again references in the heap.
For heaps $h$ and variables $x$, $h(x)$ denotes the object referenced by $x$ in $h$.

The function $HC$ takes a heap and gives the constraints satisfied by all variables of the heap.
The function $OC$ converts an object %definition TODO: keep this?
to a list of constraints on a given variable.
A Constraint $a$ is satisfied by a heap $h$, if $\entails{HC(h)}{a}$ holds.

Evaluation in $DC_C$ is a process of moving information from the expression to the heap
and the values of $DC_C$ are references in the heap.
Evaluation must yield a heap and a variable.
During reduction only new objects can be added to the heap,
while existing objects remain unchanged.

The congruence rules RC-Field, RC-New and RC-Call propagate reduction to subexpressions.
The computation rules R-Field, R-New and R-Call can only be applied
when all subexpressions of an expression are reduced to normal forms (variables).

Field access $x.f$ is reducible to the value of $x.f$ in the heap
if the heap constraints include \pathEq{x.f}{y},
where $y$ represents the value retrieved from the heap.
$x$ does not have field $f$ in the heap if the constraint is not included.

Object construction \newInst{C}{\ovl{f}}{\ovl{x}} is reduced
to a fresh variable $x$ representing the new object,
as well as extending the heap with an object $o = \obj{C}{\ovl{f}}{\ovl{x}}$ of class $C$ with
\ovl{x} as the values of the fields of that object.
It is checked that a constructor of class $C$ exists in the program
and that the constraints \ovl{b} specified by the constructor
are satisfied by the new object \entails{HC(h),OC(x,o)}{\ovl{b}}.

A method call $m(x)$ is reduced to the body of the most specific applicable method implementation.
Applicability of the implementations and selection of the most specific one
are determined by constraints.
A method declaration is applicable if its arguments are entailed by the context.
Set $S$ contains the applicable methods.
A method declaration with argument constraints \ovl{a}
is more specific than a method declaration with argument constraints \ovl{a'}
if \entails{\ovl{a'}}{\ovl{a}},
but not the other way around $\neg\entails{\ovl{a}}{\ovl{a'}}$.

\subsubsection{Type Checking}
\label{sec:dcc-types}
% begin Type assignment figure
\begin{figure}
% T-Field
\begin{prooftree}
\AxiomC{\typeass{\ovl{c}}{e}{\type{x}{\ovl{a}}}}
\AxiomC{\entails{\ovl{c},\ovl{a}}{\instOf{x.f}{C}}}
\AxiomC{\entails{\ovl{c}, \ovl{a}, \pathEq{x.f}{y}}{\ovl{b}}}
\AxiomC{$x \not \in \FV{\ovl{b}}$}
\RightLabel{T-Field}
\QuaternaryInfC{\typeass{\overline{c}}{e.f}{\type{y}{\ovl{b}}}}
\end{prooftree}
% T-Var
\begin{prooftree}
\AxiomC{\entails{\ovl{c}}{\instOf{x}{C}}}
\RightLabel{T-Var}
\UnaryInfC{\typeass{\ovl{c}}{x}{\type{y}{\pathEq{y}{x}}}}
\end{prooftree}
% T-Call
\begin{prooftree}
\AxiomC{\entails{\ovl{c}, \ovl{a}}{\ovl{a'}}} % 3
\AxiomC{\typeass{\ovl{c}}{e}{\type{x}{\ovl{a}}}} % 1
\AxiomC{$\pair{\ovl{a'}}{\ovl{b}} \in MType(m, x, y)$} % 2
\noLine
\BinaryInfC{\entails{\ovl{c},\ovl{a},\ovl{b}}{\ovl{b'}}} % 4
\AxiomC{$x \not \in \FV{\ovl{b'}}$} % 5
\RightLabel{T-Call}
\TrinaryInfC{\typeass{\ovl{c}}{m(e)}{\type{y}{\ovl{b'}}}}
\end{prooftree}
% T-New
\begin{prooftree}
\AxiomC{$\forall i.\ \typeass{\ovl{c}}{e_i}{\type{x_i}{\ovl{a_i}}}$} % 1
\noLine
\UnaryInfC{$\ovl{b} = (\instBy{x}{C}), \bigcup_i \ovl{a_i}_{\sub{x_i}{x.f_i}}$} % 3
\AxiomC{$\constr{C}{x}{\ovl{b'}} \in P$} % 2
\noLine
\UnaryInfC{\entails{\ovl{c},\ovl{b}}{\ovl{b'}}} % 4
\RightLabel{T-New}
\BinaryInfC{\typeass{\ovl{c}}{\newInst{C}{\ovl{f}}{\ovl{e}}}{\type{x}{\ovl{b}}}}
\end{prooftree}
% T-Sub
\begin{prooftree}
\AxiomC{\typeass{\ovl{c}}{e}{\type{x}{\ovl{a'}}}}
\AxiomC{\entails{\ovl{c},\ovl{a'}}{\ovl{a}}}
\RightLabel{T-Sub}
\BinaryInfC{\typeass{\ovl{c}}{e}{\type{x}{\ovl{a}}}}
\end{prooftree}
\caption{Type Assignment}
\label{fig:dcc-typeass}
\end{figure}
% end Type assignment figure
Type assignment for expressions is defined in \Cref{fig:dcc-typeass}.
The context of type assignment is a list of constraints providing information
about variables that occur in the expression.
The typing rules ensure that all free variables of the assigned types
appear in the context.
An expression type \type{x}{\ovl{a}} is a collection of constraints \ovl{a}
with a bound variable $x$
that will hold for all possible values of the expression at runtime,
if the runtime environment satisfies the constraints of the context.
The relation does not guarantee unique type assignments,
since it is possible to have different constraints satisfied by an expression.
The rule T-Sub explicitly allows weakening the type of an expression.

The type of a variable $x$ is specified by rule T-Var.
The type of $x$ asserts equivalence of the bound variable $y$ of the type to $x$.
Further it is checked that \instOf{x}{C} can be satisfied by the context for some class $C$.

Rule T-Field specifies type assignment of a field access $e.f$.
The constraints of such a field access are the constraints on $x.f$
entailed by the type of $e$
, where $x$ is the bound variable of the type of $e$.
The elimination of $x$ is done via the usage of
constraints free of $x$ entailed by the type of $e$ and \pathEq{y}{x.f},
where $y$ is used as the bound variable of the type of $e.f$
and \entails{\ovl{c},\ovl{a}}{\instOf{x.f}{C}} is used to check if
field $f$ of $e$ is available at runtime.

Rule T-Call specifies type assignment of method calls $m(e)$.
The type of $e$ is \type{x}{\ovl{a}}.
The rule checks the existence of a declaration of $m$,
whose parameter constrains are \ovl{a'} and return type constraints are \ovl{b}.
It is checked with \entails{\ovl{c},\ovl{a}}{\ovl{a'}} if
the parameter constraints are entailed by the type of $e$.
The type constraints of $m(e)$ are derived from the declared return type of $m$
and the type of the argument $e$.
The argument variable $x$ is eliminated, because it does not appear in the context
and the entailed constraints of the return types and the argument types
free of $x$ are taken.

Rule T-New specifies type assignment of object constructions.
The type \type{x}{\ovl{b}} of an object construction \newInst{C}{\ovl{f}}{\ovl{e}}
consists of a constraint stating that $C$ is the class of the object and the constraints of its fields.
The field constraints are taken from the types of the expressions $e_i$
assigned to the fields $f_i$.
Additionally \entails{\ovl{c},\ovl{b}}{\ovl{b'}} checks
that the new object satisfies the constraints of at least
one constructor \constr{C}{x}{\ovl{b'}} of class $C$.
\\
\\
% begin Type Checking figure
\begin{figure}[t]
% WF-CD
\begin{prooftree}
\AxiomC{\FVeq{\ovl{a}}{x}}
\RightLabel{WF-CD}
\UnaryInfC{\wf{\constr{C}{x}{\ovl{a}}}}
\end{prooftree}
% WF-MS
\begin{prooftree}
\AxiomC{\FVeq{\ovl{a}}{x}}
\AxiomC{\FVeq{\ovl{b}}{x,y}}
\RightLabel{WF-MS}
\BinaryInfC{\wf{(\mDecl{m}{x}{\ovl{a}}{\type{y}{\ovl{b}}})}}
\end{prooftree}
% WF-RD
\begin{prooftree}
\AxiomC{\FVeq{\ovl{a}}{x}}
\AxiomC{$\instOf{x}{C'} \in \ovl{a}$}
\RightLabel{WF-RD}
\BinaryInfC{\wf{(\progEnt{x}{\ovl{a}}{\instOf{x}{C}})}}
\end{prooftree}
% WF-MI
\begin{prooftree}
\AxiomC{\FVeq{\ovl{a}}{x}}
\AxiomC{\FVeq{\ovl{b}}{x,y}}
\AxiomC{\typeass{\ovl{a}}{e}{\type{y}{\ovl{b}}}}
\RightLabel{WF-MI}
\TrinaryInfC{\wf{(\mImpl{m}{x}{\ovl{a}}{\type{y}{\ovl{b}}}{e})}}
\end{prooftree}
% WF-Prog
\begin{prooftree}
\AxiomC{$\forall D \in P.\ \wf{D}$}
\noLine
\UnaryInfC{$\forall m.\ \forall \pair{\ovl{a}}{\ovl{b}}, \pair{\ovl{a'}}{\ovl{b'}} \in MType(m, x, y).\ \ovl{b} = \ovl{b'}$}
\noLine
\UnaryInfC{$\forall m.\ unique(m)$}
\noLine
\UnaryInfC{$\forall m.\ \forall \pair{\ovl{a}}{\ovl{b}} \in MType(m, x, y).\ complete(m, \type{x}{\ovl{a}})$}
\RightLabel{WF-Prog}
\UnaryInfC{\wf{P}}
\end{prooftree}
\caption{Type Checking}
\label{fig:dcc-wf}
\end{figure}
% end Type Checking figure
Typechecking a program is defined in \Cref{fig:dcc-wf}.
A program is well-formed if all its declarations are well-formed,
method implementations need to be unique and complete
and all declarations of a method need to have the same return type.
Completeness and uniqueness of method implementations are properties
of well-formed heaps.
A declaration is well-formed if its used variables are bound.
Rule WF-MI additionally checks if the body of a method declaration
respects the declared return type.
Rule WF-RD restricts constraint derivation rules:
constraints of the form \instOf{x}{C} can only be derived
if some other class of $x$ is known.
Such derivation rules are used to encode inheritance between
dependent classes.

% begin Operational Semantics figure}
\begin{figure}[h]
\begin{align*}
o &::= \stdobj && \text{(objects)} \\ % \langle C; \overline{f} \equiv \overline{x} \rangle \\
h &::= \stdheap\ \ (x_i \text{ distinct}) && \text{(heaps)}
\end{align*}
\begin{align*}
OC(x, o) &= (\instBy{x}{C}, \pathEq{x.\overline{f}}{\overline{x}}) &&\text{where } o = \stdobj \\ % \langle C; \overline{f}\equiv\overline{x} \rangle \\
HC(h) &= \bigcup_i OC(x_i, o_i) &&\text{where } h = \stdheap % \overline{x} \mapsto \overline{o}
\end{align*}
% vertical inference rule example
%\begin{prooftree}
%\AxiomC{$A\lor B$}
%\AxiomC{$[A]$}
%\noLine
%\UnaryInfC{$C$}
%\AxiomC{$[B]$}
%\noLine
%\UnaryInfC{$C$}
%\TrinaryInfC{$C$}
%\end{prooftree}
% R-New
\begin{prooftree}
\AxiomC{$x \not \in dom(h)$} % 1
\noLine
\UnaryInfC{$\constr{C}{x}{\overline{b}} \in P$} % 3
\AxiomC{o = \stdobj} % 2
\noLine
\UnaryInfC{\entails{HC(h), OC(x, o)}{\overline{b}}} % 4
\RightLabel{R-New}
\BinaryInfC{$\eval{h}{\newInst{C}{\overline{f}}{\overline{x}}}{h, x \mapsto o}{x}$}
\end{prooftree}
% R-Field
\begin{prooftree}
\AxiomC{$(\pathEq{x.f}{y}) \in HC(h)$}
\RightLabel{R-Field}
\UnaryInfC{\eval{h}{x.f}{h}{y}}
\end{prooftree}
% R-Call
\begin{prooftree}
\AxiomC{$S = \{ \pair{\overline{a}}{e}\ |\ \pair{\overline{a}}{e} \in MImpl(m, x) \land \entails{HC(h)}{\overline{a}} \}$}
\AxiomC{$\pair{\overline{a}}{e} \in S$}
\noLine
\BinaryInfC{$\forall \pair{\overline{a'}}{e'} \in S.\ (e' \neq e) \longrightarrow (\entails{\overline{a'}}{\overline{a}}) \land \neg(\entails{\overline{a}}{\overline{a'}})$}
\RightLabel{R-Call}
\UnaryInfC{\eval{h}{m(x)}{h}{e}}
\end{prooftree}
% RC-Field
\begin{prooftree}
\AxiomC{\eval{h}{e}{h'}{e'}}
\RightLabel{RC-Field}
\UnaryInfC{\eval{h}{e.f}{h'}{e'.f}}
\end{prooftree}
% RC-Call
\begin{prooftree}
\AxiomC{\eval{h}{e}{h'}{e'}}
\RightLabel{RC-Call}
\UnaryInfC{\eval{h}{m(e)}{j'}{m(e')}}
\end{prooftree}
% RC-New
\begin{prooftree}
\AxiomC{\eval{h}{e}{h'}{e'}}
\RightLabel{RC-New}
%\UnaryInfC{
%  \eval
%    {h}
%    {\newInst{C}
%      {\overline{f} \equiv \overline{x}, f}
%      {e, \overline{f'} \equiv \overline{e'}}}
%    {h'}
%    {\newInst{C}
%      {\overline{f} \equiv \overline{x}, f}
%      {e', \overline{f'} \equiv \overline{e'}}}}
\UnaryInfC{
    \pair
        {h}
        {\newInst{C}
          {\overline{f} \equiv \overline{x}, f}
          {e, \overline{f'} \equiv \overline{e'}}}
    \ \ \ \ \ \ \ }
\noLine
\UnaryInfC{\ \ \ \ \ \ \ $\rightarrow$
    \pair
        {h'}
        {\newInst{C}
          {\overline{f} \equiv \overline{x}, f}
          {e', \overline{f'} \equiv \overline{e'}}}
}
\end{prooftree}
\caption{Operational Semantics}
\label{fig:dcc-opsemantics}
\end{figure}
% end Operational Semantics figure

%%% Local Variables: 
%%% mode: latex
%%% TeX-master: "../thesis"
%%% End: 
