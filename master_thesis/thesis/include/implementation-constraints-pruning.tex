\subsection{Pruning}
\label{sec:pruning}
We found that the solver sometimes times out
given a valid entailment, if we supply rules
that are unused in the proof of the entailment.
We observed this in combination
with the enumeration of variables and paths of rule C-Subst.\\
\\
While all the generated rules C-Subst are valid,
the solver does sometimes focus on one substitution
and loop instantiations of the same rule
building an implication chain of the same rule.
This happens if the substitution produces implications
of the form $\entails{\ovl a}{a} \rightarrow \entails{\ovl a}{a}$
for the input.
In our observations, this happened with reflexive path pairs \pathEq{p}{p}.
To counter this behavior we introduce \textit{pruning} into function \scala{generateSubstRules},
through skipping the generation of rules C-Subst for reflexive paths.
%
\begin{figure}[h]
\begin{lstlisting}
def generateSubstRules(
  vars: List[Id], paths: List[Path], pruning: Boolean = false)
    : Seq[SMTLibCommand] = {
  var rules: Seq[SMTLibCommand] = Seq()
  val pathPairs = makePathPairs(paths)

  vars.foreach(x => pathPairs.foreach{
    case (p, q) if x == p && p == q => () // skip
    case (p, q) if pruning && p == q => () // skip
    case (p, q) => rules = rules
                         :+ instantiateSubstRule(x, p, q)
  })
  rules
}
\end{lstlisting}
\caption{Pruning}
\label{fig:scala-pruning}
\end{figure}\\
The incorporation of pruning into function \scala{generateSubstRules}
is given in \Cref{fig:scala-pruning}.
We extended the arguments of the function with a
boolean parameter \mIt{pruning} with the default value \mIt{false}.
In the implementation, we skip rule generations
if \mIt{pruning} is enabled and $p = q$.\\
\\
Pruning removes potentially needed rules.
Therefore, we incorporate pruning into function
\scala{entails} as a two-pass process.
\Cref{fig:scala-entails-twopass} shows the updated
implementation of function \scala{entails} defined in \Cref{lst:entails}.
In the updated implementation, we enable the \mIt{pruning} parameter
for the first call to \scala{generateSubstRules}.
We check for the result of the solver.
In case its \smtlib{unknown},
meaning that either the input is undecidable or
the solver timed out,
we start a second run of the solver without pruning.
For this, we flush the added commands to the solver.
Afterwards we add the flushed rules again,
but this time we call \scala{generateSubstRules} with disabled pruning.
%
\begin{figure}[t]
\begin{lstlisting}
def entails(
    context: List[Constraint], c: Constraint): Boolean = {
  [...]
  solver.addCommands( // pruning enabled
    SMTLibConverter.generateSubstRules(vars, paths, true))
  [...]

  solver.checksat() match {
    case Sat => false
    case Unsat => true
    case Unknown =>
      solver.flush() // Second pass without pruning
      solver.addCommands(
        SMTLibConverter.generateSubstRules(vars, pths, false))
      [...]

      solver.checksat() match {
        case Unsat => true
        case _ => false
      }
  }
}
\end{lstlisting}
\caption{Function Entails With Pruning}
\label{fig:scala-entails-twopass}
\end{figure}


%%% Local Variables: 
%%% mode: latex
%%% TeX-master: "../thesis"
%%% End: 
