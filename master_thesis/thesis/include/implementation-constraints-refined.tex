\subsection{Rule Refinement}
\label{sec:rule-refinement}
In \Cref{ex:application-entailment} we have seen the application of our model
of the constraint entailment. The structure of the derivation followed the application
of the sequent calculus.
In the example we used rule C-Subst and chose the quantified variables
to fulfill the various requirements of the rule.
These requirements included \mIt{variable} and \mIt{subst} as well
as the relationship between variables $a$, $a_1$ and $a_2$.
We found it difficult for the solver to make these variable instantiations
and we refine the modeled rules to improve usage with the solver.

\subsubsection{Rule C-Subst}
We start with refining rule C-Subst.
The identified problems of this rule are
\begin{enumerate}
    \item the amount of quantified variables and
    \item the non-algorithmic nature of the rule.
\end{enumerate}
\quad\\
As observed in (1) the rule quantifies over seven variables.
This leads to the explosion of the search space for this rule
and the quantifier instantiation routines of the solver have a
problem to instantiate the quantified variables in a fitting way.
To solve this problem we try to ground at least some
of the quantified variables of the rule.
We notice that all variable names and paths are known at runtime
and none of the rules introduces new ones.
We can ground variables $x$, $p_1$ and $p_2$ of rule C-Subst
by enumerating each possible combination of those variables
and by creating a specialized subst rule for each of these combinations.

Following this we can transform the rule C-Subst given in \Cref{fig:axioms-naive}
by removing the quantified variables $x$, $p_1$ and $p_2$
and since we enumerate over each variable name
we do also remove the \mIt{variable(x)} requirement.
With this we reduced the number of quantified variables
of the rule to four.
We note that these variables are now free in the rule
and would need to be supplied from the outside.
Since we do not want a single rule with free variables,
but rather a template for generating subst rules,
we treat the now free variables as placeholders for
a routine that replaces these for each possible combination.
For better visibility we write the placeholders in bold: % TODO: upper case bold?
\begin{align*}
&\forall \ovl a: \Constrs, a, a_1, a_2: \Constr. && \text{(C-Subst)} \\
&\quad \entails{\ovl a}{\pathEq{\mathbf{p_2}}{\mathbf{p_1}}}
         \land \mIt{subst}(a, \mathbf{x}, \mathbf{p_1}, a_1) \land \mIt{subst}(a, \mathbf{x}, \mathbf{p_2}, a_2)
       \land \entails{\ovl a}{a_1}\\
&\quad \rightarrow \entails{\ovl a}{a_2}
\end{align*}
%
For (2) we take a look the instantiations of $a$, $a_1$ and $a_2$
made in \Cref{ex:application-entailment}.
In the example we started with instantiating $a_2$ to match
the entailment to be shown.
Afterwards we chose $a$ such that the substitution of
$x$ with $p_2$ in $a$ would equal $a_2$.
We see that to instantiate $a$ we solved a substitution
where $a$ is used as the input and $a_2$ as the output of the substitution.
Finally we instantiated $a_1$ with $x$ substituted with $p_1$ in $a$.
We observe that this resembles a linear instantiation scheme starting with $a_2$,
but we note that we had to make an educated guess for choosing $a$.
So with the current rule, the solver needs to make random instantiations
for $a$ to afterwards check if the requirements on $a$ are met
instead of following the observed instantiation pattern we did manually.
To resolve this problem we need to change the rule that $a$ can be
instantiated by direct derivation from $a_2$.
As previously stated we used $a$ on an input position
and $a_2$ on the output position of a function call.
To reverse these positions, we introduce \mIt{generalization} as
the inverse of substitution.
% Functions for generalization
\begin{figure}[t]
\centering
\begin{align*}
% path generalization
&\genp(p_1: Path, p_2: Path, x: String):\mIt{Path} = \\
&\quad\If p_1 = p_2\\
&\qquad \Then x\\
&\qquad \Else p_1 \match\\
&\qquad\quad y \Rightarrow y\\
&\qquad\quad q.f \Rightarrow \genp(q, p_2, x).f\\
%\ite{(p_1 = p_2)}{x}{
%p_1 \match\\
%&\qquad y \Rightarrow y\\
%&\qquad q.f \Rightarrow \genp(q, p_2, x).f
%}\\
\\
% constraint generalization
&\genc(c: \Constr, p: Path, x: String): \Constr = c \match\\
&\quad \case{\pathEq{q_1}{q_2}}
  {\pathEq
    {\genp(q_1, p, x)}
    {\genp(q_2, p, x)}}\\
&\quad \case{\instOf{q}{C}}
  {\instOf{\genp(q, p, x)}{C}}\\
&\quad \case{\instBy{q}{C}}
  {\instBy{\genp(q, p, x)}{C}}\\\\
\end{align*}
\caption{Path Generalization Functions}
\label{fig:axioms-general-gen}
\end{figure}\\
Functions for generalizing paths and constraints are defined in \Cref{fig:axioms-general-gen}.
Generalization for paths functions reverse of substitution for paths.
Instead of substituting a variable with a path we generalize a subpath with a variable.
This generalization of a subpath cannot happen at random places inside a path,
but only starting on the innermost/leftmost variable of the path (the root of the abstract syntax tree (AST) of that path).
For example we can generalize $x.f$ with $y$ in $x.f.g$ to obtain $y.g$,
but we cannot generalize $f.g$ with $h$ in $x.f.g$
since $f$ is not the innermost occurrence in $x.f.g$.
So we can only start generalization from the variable of a path and not its fields.
For paths $p, q$ and strings $x$ path generalization is defined
such that $gen(subst(p, x, q), q, x) = p$.\\
Generalization for constraints does,
as substitution for constraints did,
propagate path generalization to the paths contained in the constraint
to be generalized.

For the notation of generalization we use \gen{c}{p}{x} for
generalizing paths $p$ with variable names $x$ in constraints $c$.
This notion is similar to the one used for substitution.
We can distinguish both notations based on the swapped positions
of $p$ and $x$.
%
\begin{figure}[h]
\begin{align*}
&\forall \ovl a: \Constrs, a_2: \Constr. && \text{(C-Subst)}\\
&\quad \Let a := \gen{a_2}{\mathbf{p_2}}{\mathbf{x}}\ \In\\%\genc(a_2, \mathbf{p_2}, \mathbf{x})\ \In\\
&\qquad \Let a_1 := \subst{a}{\mathbf{x}}{\mathbf{p_1}}\\ %\substc(a, \mathbf{x}, \mathbf{p_1})\\
&\qquad \In
        \entails{\ovl a}{\pathEq{\mathbf{p_2}}{\mathbf{p_1}}}
        \land
        \entails{\ovl a}{a_1}\\
&\rightarrow \entails{\ovl a}{a_2}
\end{align*}
\caption{C-Subst Template}
\label{fig:axioms-csubst}
\end{figure}\\
We can now solve the problem of choosing an instantiation for $a$.
With the concept of generalization as previously presented
we can choose $a$ to be a generalization of $a_2$,
where $a_2$ is used in an input position
and the output is used as the instantiation of $a$.
This brings the instantiations of $a$, $a_1$ and $a_2$
into a derivation chain starting from
$a_2$ over $a$ to $a_1$.
Since we can derive $a$ and $a_1$ from $a_2$ through a defined function,
we can remove $a$ and $a_1$ from the list of quantified variables.\\
\\
We incorporate these changes into the template presented in step (1)
to obtain the new template for rule C-Subst as shown in \Cref{fig:axioms-csubst}.
In this template we reduced the amount of quantified variables to two.
These two quantified variables can be used to match the right-hand side
of the implication.
We used let bindings to derive $a$ from $a_2$ and $a_1$ from $a$.\\
\\
We can combine the observations made in steps (1) and (2)
to improve the rule generation out of the template.
A procedure to generate subst rules out of the template
needs to enumerate over all possible combinations of
variable names and path pairs.

The first observation to make is that in order to be complete
we do not need all possible combinations.
We can safely remove combinations where $x = p_1 = p_2$ for strings $x$ and paths $p_1, p_2$,
since there is no new information to be gained out of such a combination.
For $a$ we derive $\gen{a_2}{p_2}{x}$
and since $p_2 = x$ we obtain $a := a_2$.
The following derivation would then be $\subst{a}{x}{p_1}$
and since $x = p_1$ and $a = a_2$ we obtain $a_1 := a_2$.
This results in the implication:
\[ \entails{\ovl a}{\pathEq{p_2}{p_1}} \land \entails{\ovl a}{a_1} \rightarrow \entails{\ovl a}{a_2} \]
Since $p_1 = p_2$ and path equivalence is reflexive
\entails{\ovl a}{\pathEq{p_2}{p_1}} holds for any \ovl a
and since $a_2 = a = a_1$ we remain with
\[ \entails{\ovl a}{a_2} \rightarrow \entails{\ovl a}{a_2} \]
from where we can not gain any information.
More so this could lead for the solver to repetitively try to use this rule
and loop.\\
\\
The second observation follows from the first.
We can omit
\begin{enumerate}
    \item the generalization \gen{a_2}{p_2}{x} if $x = p_2$,
          since $\gen{a_2}{x}{x} = a_2$.
    \item the substitution \subst{a}{x}{p_1} if $x = p_1$,
          since $\subst{a}{x}{x} = a$.
    \item the entailment check \entails{\ovl a}{\pathEq{p_2}{p_1}} if $p_1 = p_2$,
          since \entails{\ovl a}{\pathEq{p}{p}} is derivable through using rule C-Weak
          until we obtain \entails{\nil}{\pathEq{p}{p}} followed
          by using rule C-Refl to close the proof.
\end{enumerate}
By omitting the generalization or the substitution
we can remove the respective let binding
and replace the now free variable in its subexpression.
\Cref{ex:subst-optimzed} shows this for the case $x = p_1$.
\begin{example}[Optimized subst rule with $x = p_1$]
\label{ex:subst-optimzed}
\begin{align*}
&\forall \ovl a: \Constrs, a_2: \Constr.\\
&\quad \Let a := \gen{a_2}{\mathbf{p_2}}{\mathbf{x}}\\
&\quad \In
       \entails{\ovl a}{\pathEq{\mathbf{p_2}}{\mathbf{p_1}}}
       \land
       \entails{\ovl a}{a}\\
&\rightarrow \entails{\ovl a}{a_2}
\end{align*}
\end{example}
%
\subsubsection{Rule C-Prog}
Rule C-Prog has similar problems as the ones observed for rule C-Subst.
The rule is again defined in a non-algorithmic way.\\
The check if a program entailment exists in the program is done via
the predicate \inprog(x, \ovl a, a) for strings $x$, constraint lists \ovl a and constraints $a$,
where $x$ denotes a variable binding that can occur in \ovl a and $a$.
The predicate models the existence of a declaration
\progEnt{x}{\ovl a}{a} in the program.
This is aggravated by the fact that the rule requires us
to find a substitution from $x$ to some path $p$ to eliminate
the bound variable of the program entailment $x$,
since it would be otherwise free in the constraint entailment.\\
\\
To solve this problem we apply the same technique as
we previously did with rule C-Subst.\\
We can enumerate over all possible combinations of variable names and paths
to eliminate the need to find a proper substitution.
As well as finding a way to express the existence check
of program entailments in a derivable way,
where the quantified constraint $a$ can be used as an input
and would result in an output to be usable as an instantiation for $\ovl a$.

With this requirement specification we see that we want to have
some sort of lookup function ranging from constraint $a$ to constraints $\ovl a$.
Such a lookup function would need to be specifically generated for a program
and we cannot give an universal function definition.

We develop the generation process of a lookup function which models
the existence check from rule C-Prog through a series of examples.
Since such a lookup function needs to be specific to a given program,
we start by defining a set of program entailments in \Cref{ex:progent-inheritance}.
%
\begin{example}[Program entailment class hierarchy]\quad\\
\label{ex:progent-inheritance}
The program for which we want to generate a lookup function
contains the following entailment declarations.
\begin{align*}
&\progEnt{x}{\instOf{x}{B}}{\instOf{x}{A}}\\
&\progEnt{x}{\instOf{x}{C}, \instOf{x.f}{D}}{\instOf{x}{A}}\\
&\progEnt{x}{\instOf{x}{E}}{\instOf{x}{D}}
\end{align*}
Since program entailments are used to express inheritance relations,
the given entailments form the following class hierarchy.
\begin{center}
\begin{tikzpicture}
  \node (A) at (-1,0) {A};
  \node (B) at (-3,-1.5) {B};
  \node (C) at (1,-1.5) {C};
  \path (A) edge (B);
  \path (A) edge (C);
  
  \node (D) at (3,0) {D};
  \node (E) at (3,-1.5) {E};
  \path (D) edge (E);
% \node (comment) at (-0.5, -0.5) {asdf};
\end{tikzpicture}
\end{center}
\end{example}\quad\\
We recall from \Cref{fig:dcc-wf} that a program entailment \progEnt{y}{\ovl b}{b}
is only well-formed if constraint $b$ is
of the form \instOf{q}{Cls} and $\instOf{q}{Cls'} \in \ovl b$.\\
With this in mind, rule C-Prog states that to show that
path $q$ is an instance of class $Cls$ it suffices to show
that $q$ is an instance of subclass $Cls'$
and that the field requirements of $Cls'$ are fulfilled.

To translate this into a lookup function,
we use $b$ as an argument to the function and \ovl b as our return value.
Since a function needs to be total we need to combine all
existing program entailments into one function,
as well as the need for a result in the case where no match exists.

Since the well-formedness requires \ovl b to contain at least
one element, \ovl b cannot be empty and we use the
empty list $\nil$ as the return value for the non matching case.\\
To differentiate between program entailments we use if statements
to check if the input matches one of the program entailments.\\
Such a lookup function is shown in \Cref{ex:lookup-fun-ite}.

\begin{example}[Lookup function using if statements]
\label{ex:lookup-fun-ite}
\begin{align*}
&\mIt{lookup}(a: \Constr): \Constrs = \\
&\If a = \instOf{x}{A}\\
&\Then [\instOf{x}{B}]\\
&\Else \If a = \instOf{x}{A}\\
&\quad \Then [\instOf{x}{C}, \instOf{x.f}{D}]\\
&\quad \Else\If a = \instOf{x}{D}\\
&\qquad \Then [\instOf{x}{E}]\\
&\qquad \Else \nil\\
\end{align*}
\end{example}
%
We can observe in \Cref{ex:lookup-fun-ite} that we
created one if statement per program entailment.
This practice can lead to the shadowing of program entailments
as seen in the example, where the first and the second if statement share the same guard.
This is the case because there can be more than one entailment
with the same right-hand side constraint,
as seen in the given program entailments in \Cref{ex:progent-inheritance}.
This leads to shadowing, since we use the
right-hand side as the guard for the if statements.\\
\\
To avoid this shadowing of entailments,
we change the return type of the lookup function
from returning a list of constraints (\Constrs)
to a list of constraint lists (\Constrss).\\
This enables us to merge program entailments that
share the same right-hand side together.
With this we can give the function definition shown in
\Cref{ex:lookup-fun-shadowing} that avoids shadowing.
%
\begin{example}[Avoid shadowing in lookup]
\label{ex:lookup-fun-shadowing}
\begin{align*}
&\mIt{lookup}(a: \Constr): \Constrss = \\
&\If a = \instOf{x}{A}\\
&\Then [[\instOf{x}{B}], [\instOf{x}{C}, \instOf{x.f}{D}]]\\
&\Else \If a = \instOf{x}{D}\\
&\quad\Then [[\instOf{x}{E}]]\\
&\quad\Else \nil\\
\end{align*}
\end{example}
%
We can see in \Cref{ex:lookup-fun-shadowing}
that we merged the two entailments whose right-hand side
is $\instOf{x}{A}$ and return both possible constraint lists
as the result for a match to these entailments.

To translate program entailments \progEnt{y}{\ovl b}{b}
we used $b$ to check for equality to the argument to the function
and \ovl b as the return value.
As of now we neglected the bound variable $y$ of the program entailment.\\
So we only solved the non-algorithmic nature of the rule
and we still need to find a suitable substitution
of the quantified variable $a$ from rule C-Prog
to input into the lookup function, such that the substitution of $a$ matches $b$.
We can incorporate the enumeration of combinations of variable names and paths
into the lookup function generation to resolve this.\\
\\
To do this we look at rule C-Prog and
see that the bound variable of the program entailment is used
for the substitution.
This is because, according to the well-formedness,
for program entailments \progEnt{y}{\ovl b}{b}
only the bound variable $y$ can occur in \ovl b
and $b$ needs to be of the form \instOf{y}{Cls}
for class names \mIt{Cls}.
So for finding a valid substitution,
we can fix the variable name to be substituted to $y$.\\
Since we fixed the variable name to be substituted,
what remains is the need to instantiate each program entailment
with each possible path.
Previously we only needed to generate a lookup function per program.
Now we need to generate a new lookup function
for each constraint entailment we want to show,
since we enumerate over all possible paths.

With this in mind we generate a lookup function for the
program entailments shown in \Cref{ex:progent-inheritance}
with valid paths $y$ and $y.g$.
The resulting lookup function is shown in \Cref{ex:lookup-fun}.
\newpage
\begin{example}[Program entailment lookup function]\quad\\
\label{ex:lookup-fun}
\begin{align*}
&\mIt{lookup}(a: \Constr): \Constrss = \\
&\If a = \instOf{y}{A}\\
&\Then [[\instOf{y}{B}], [\instOf{y}{C}, \instOf{y.f}{D}]]\\
&\Else \If a = \instOf{y}{D}\\
&\quad \Then [[\instOf{y}{E}]]\\
&\quad \Else \If a = \instOf{y.g}{A}\\
&\qquad \Then [[\instOf{y.g}{B}], [\instOf{y.g}{C}, \instOf{y.g.f}{D}]]\\
&\qquad \Else \If a = \instOf{y.g}{D} \\
&\qquad\quad \Then [[\instOf{y.g}{E}]]\\
&\qquad\quad \Else \nil
\end{align*}
\end{example}
If we compare \Cref{ex:lookup-fun-shadowing} and \Cref{ex:lookup-fun},
we can observe that the function in \Cref{ex:lookup-fun} uses
twice as much if statements as the function in \Cref{ex:lookup-fun-shadowing}.
This is because we instantiated each of the entailments with two possible paths,
doubling the amount of cases to consider.\\
\\
With the final lookup function definition from \Cref{ex:lookup-fun}
we can refine rule \mbox{C-Prog}.
We do this by replacing the declarative requirement of the $\inprog$ predicate
with the algorithmic call to the lookup function.
%
\begin{figure}[h]
\begin{align*}
&\forall \ovl a: \Constrs, a: \Constr. && \text{(C-Prog)}\\
&\Let \ovl{\ovl a} := lookup(a)\\   %\bar{\bar a}
&\In
     \neg (\ovl{\ovl a} = \nil)
   \land
     \bigvee_{\ovl b \in \ovl{\ovl a}}{\entails{\ovl a}{\ovl b}}
   \rightarrow \entails{\ovl a}{a}
\end{align*}
\caption{Rule C-Prog Using Lookup}
\label{fig:axioms-cprog}
\end{figure}\\\\
%
The refined rule C-Prog is defined in \Cref{fig:axioms-cprog}.\\
The rule calls function \mIt{lookup} to check
if an entailment declaration matching $a$ exists in the program.
The result is a list containing possible constraint lists
to evaluate for entailment to $a$.\\
Since we modeled the lookup function to return the empty list $\nil$
if no matching entailment declaration exists,
we require the lookup result \ovl{\ovl a} to be non empty.\\
Since we merged entailment declarations with matching right-hand sides
to avoid shadowing, it suffices to show that at least one
of the constraint lists contained in \ovl{\ovl a}
is entailed by \ovl a to show that $a$ is entailed by \ovl{a}.\\
\\
The refined rule C-Prog does not contain any free variables
and is therefore not a template like the refined C-Subst,
but a general rule that does not need any preprocessing.
This is the case because we moved the enumeration process
of possible paths into the generation of the lookup function.

\subsubsection{Rule C-Perm}
The permutation rule C-Perm is defined in a
non-structural way.
The SMT solver can use the rule without a problem to check for
two known entailments \entails{\ovl a}{a} and \entails{\ovl b}{a}
if \ovl a is a permutation of \ovl b.
The solver struggles with using this rule
for generating new entailments \entails{\ovl b}{a}
from a single known entailment \entails{\ovl a}{a},
such that \ovl b is a permutation of \ovl a.\\
\\
The permutation generating aspect is, in our model,
used for closures with rule C-Ident in combination with rule C-Weak
since our model is list based.
Rule C-Ident requires a list containing only one
element for its context and rule C-Weak can only
eliminate the first element from the context.
Therefore to close a proof for \entails{\ovl c}{c} with C-Ident
where $c \in \ovl c$ and $\neg (last(\ovl c) = c)$,
we need to find a permutation \ovl{c'} of \ovl c
where $last(\ovl{c'}) = c$.
%
\begin{figure}[h]
\begin{align*}
&\forall \ovl c: \Constrs, c: \Constr. && \text{C-DirectIdent}\\
&\quad c \in \ovl c \rightarrow \entails{\ovl c}{c}
\end{align*}
\caption{Rule C-DirectIdent}
\label{fig:axioms-cdirectident}
\end{figure}\\
%
To avoid the generation of permutations by the solver,
we define a rule for direct closure for such entailments
using rule C-Ident.\\
We introduce rule C-DirectIdent in \Cref{fig:axioms-cdirectident}.
Rule C-DirectIdent can be used to directly close entailments
of the form \entails{\ovl c}{c} if $c \in \ovl c$.\\
Rule C-DirectIdent introduces a new quantified variable,
potentially leading to an increased search space for the solver.
This is not the case for the rule.
The newly introduced quantified variable \ovl c
is contained in the entailment we want to show
and therefore used in an input position
which can be easily matched by the solver.
This is also true for the the quantified variable $c$
in combination with the element of check.
Both quantified variables are used in input positions.
Therefore the solver does not need to generate
instantiations that match the property $c \in \ovl c$,
but can rather check if the property is fulfilled
for the known values.\\
\\
Rule C-DirectIdent does not contradict any of the other rules.
It can solely be used as a "shortcut" for rule C-Ident.
We show that rule C-DirectIdent and rule C-Ident are interchangeable.\\
\\
Each entailment \entails{\ovl c}{c} closable with rule C-DirectIdent
can also be closed with the usage of rule C-Ident.
We produce a derivation for \entails{\ovl c}{c}
without the usage of C-DirectIdent.
Since the entailment is closable by C-DirectIdent,
we know that $c \in \ovl c$.
We generate a permutation \ovl{c'} of \ovl c
with the property $last(\ovl{c'}) = c$.
We do this by appending $c$ at the end of \ovl c
and obtain $\ovl{c'} := \ovl c \conc [c]$.
\ovl{c'} is a valid permutation in our model,
since rule C-Perm does not require
both lists to be of equal length and
we know that $c \in \ovl c$.
From there we repeatedly apply rule C-Weak
until we are left with the subgoal \entails{[c]}{c}.
We show this entailment by applying rule C-Ident.\\
\\
Each entailment \entails{\ovl c}{c} closable with rule C-Ident
can also be closed with the usage of rule C-DirectIdent.
The entailment is closable with rule C-Ident,
we therefore know that \ovl c is of the form $[c]$.
Since $c \in [c]$, we show the entailment \entails{[c]}{c}
by directly applying rule C-Ident.\\
\\
We showed that each entailment closable by C-DirectIdent
can also be closed by C-Ident and vice versa.
We replace rule C-Ident with rule C-DirectIdent.

%%% Local Variables: 
%%% mode: latex
%%% TeX-master: "../thesis"
%%% End: 
