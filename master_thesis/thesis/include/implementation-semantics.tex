\section{Interpreter}
\label{sec:interp}
In this section we implement
the operational semantics of the $DC_C$ calculus
presented in \Cref{fig:dcc-opsemantics}.
For this we define an interpreter,
which will evaluate expressions relative to a heap.
The operational semantics is a small-step semantics over
the structure of expressions.
%
\begin{lstlisting}[caption={Interpreter},label=lst:interp,captionpos=b,frame={lines}]
def interp(heap: Heap, expr: Expression)
          : (Heap, Expression) = expr match {
  case FieldAccess(X@Id(_), F@Id(_)) =>   // R-Field
    HC(heap).filter{
        case PathEquivalence(FieldPath(X, F), Id(_)) => true
        case PathEquivalence(Id(_), FieldPath(X, F)) => true
        case _ => false
        } match {
      case PathEquivalence(FieldPath(X, F), y@Id(_)) :: _ =>
        (heap, y)
      case PathEquivalence(y@Id(_), FieldPath(X, F)) :: _ =>
        (heap, y)
      case _ => (heap, expr) // f no field of x
    }
  case MethodCall(m, x@Id(_)) =>   // R-Call
    // Applicable methods
    val S: List[(List[Constraint], Expression)] =
        mImplSubst(m, x).filter{
          case (as, _) => entails(HC(heap), as)}
    if (S.isEmpty) // m not in program
      return (heap, expr)

    var (a, e) = S.head // Most specific method
    S.foreach{
      case (a1, e1) if e != e1 =>
        if (entails(a1, a) && !entails(a, a1)) {
          a = a1; e = e1
        }
    }
    interp(heap, e)
  // R-New
  case ObjectConstruction(cls, args)
    if args.forall{ // if args are values (Id)
      case (_, Id(_)) => true
      case _ => false
    } =>
    val x: Id  = freshvar()
    val o: Obj = (cls, args.asInstanceOf[List[(Id, Id)]])
    // cls in Program: alpha renaming of y to x in b
    val (y: Id, b: List[Constraint]) =
      classInProgram(cls, P).getOrElse(return (heap, expr))
    val b1 = substitute(y, x, b)
    // heap constraints entail cls constraints
    if (entails(HC(heap) ++ OC(x, o), b1))
      (heap + (x -> o), x)
    else
      (heap, expr) // stuck
      
  // RC-Field
  case FieldAccess(e, f) =>
    val (h1, e1) = interp(heap, e)

    if(h1 == heap && e1 == e) {
      (heap, expr) // stuck
    } else {
      interp(h1, FieldAccess(e1, f))
    }
    
  // RC-Call
  case MethodCall(m, e) =>
    val (h1, e1) = interp(heap, e)

    if(h1 == heap && e1 == e) {
      (heap, expr) // stuck
    } else {
      interp(h1, MethodCall(m, e1))
    }
    
  // RC-New
  case ObjectConstruction(cls, args) =>
    val (h1, args1) = objArgsInterp(heap, args)

    if(h1 == heap && args1 == args) {
      (heap, expr) // stuck
    } else {
      interp(h1, ObjectConstruction(cls, args1))
    }
    
    case Id(_) => (heap, expr) // variables are values
}
\end{lstlisting}
%
\Cref{fig:scala-heapconstr} implements functions \mIt{HC} and \mIt{OC}
defined in \Cref{fig:dcc-opsemantics}.
\\
The constraints of a heap are the constraints
on the objects contained in the heap.
The implementation of $\mIt{HC}(heap)$ maps \mIt{OC} over
all elements contained in $heap$.

The constraints of an object state the class of the object
and the values of its fields.
In our implementation, an object consists of a class name
and a list of pairs of field names and variable names.
In the implementation of $\mIt{OC}(x, o)$ with $o := (\mIt{cls}, \mIt{fields})$,
we create a constraint \instBy{x}{cls} stating the class of $o$.
Further, we create for each field-value pair $(f, v) \in \mIt{fields}$
a constraint \pathEq{x.f}{v} stating that field $f$ of $x$ has value $v$.
%
\begin{figure}[h]
\begin{lstlisting}
// Heap Constraints
private def HC(heap: Heap): List[Constraint] =
  heap.flatMap{case (x, o) => OC(x, o)}.toList

// Object Constraints
private def OC(x: Id, o: Obj): List[Constraint] = o match {
  case (cls, fields) =>
    val init = InstantiatedBy(x, cls)
    val fieldCs = fields.map{case (f, v) =>
      PathEquivalence(FieldPath(x, f), v)}

    init :: fieldCs
}
\end{lstlisting}
\caption{Heap And Object Constraints}
\label{fig:scala-heapconstr}
\end{figure}
\\
We define function \scala{interp} in \Cref{lst:interp}.
The function takes a heap $heap$ and an expression $expr$
as arguments and returns a heap and an expression.
We do not return intermediate steps in the function,
but fully evaluate the argument expression to a value.
The implementation follows the structure of the operational
semantics.
This is done via pattern matching on the argument expression
and allows to recreate each rule from \Cref{fig:dcc-opsemantics}
as a separate case in the implementation.\\
\\
The first case implements rule R-Field,
it is marked in the Scala code with the comment R-Field.
We match for field access $e.f$, where the subexpression $e$
is evaluated to a variable.
Variables are the values of the $DC_C$ calculus.
We then build the heap constraints $HC(heap)$
and filter them to check if they contain \pathEq{x.f}{y}
or \pathEq{y}{x.f} for some variable $y$.
If such a path equivalence exists, we
reduce $x.f$ to $y$ and return
the pair $(heap, y)$.
Otherwise we can not reduce $x.f$,
because field $f$ is not a member of the object
corresponding to $x$ in the heap.
We return the arguments $(heap, expr)$
as there are no subsequent transitions.\\
\\
For rule R-Call, we match for method calls $m(x)$,
where $x$ is a value.
First, we generate the list of applicable methods $S$.
For this we call $\mIt{mImplSubst}(m, x)$
to search for all implementations of method $m$
and filter its result to check if the
argument constraints are entailed by the heap.
Function \mIt{mImplSubst} implements \mIt{mImpl}
defined in \Cref{fig:dcc-syntax}.
In \mIt{mImplSubst} we explicitly incorporated the unification
of the formal argument of method implementations with $x$,
since the unification is implicit in the operational semantics.
If $S$ is empty, there is no implementation for $m$ in the program
and we can not reduce $m(x)$.
Afterwards, we determine the most specific method implementation.
With $(a, e) := S.head$, we set the first applicable method
as the currently most specific method.
We then try for each applicable method $(a_1, e_1)$ where $e \not= e_1$,
if \entails{a_1}{a} and not \entails{a}{a_1}.
If so we update the most specific method $(a, e) := (a_1, e_1)$.
After all applicable methods have been checked, we reduce
$m(x)$ to the body $e$ of the most specific implementation of $m$.
Since we do not want to produce intermediate results and $e$
is not guaranteed to be a value we recursively call \scala{interp(heap, e)}.
%
\begin{figure}[h]
\begin{lstlisting}
private def mImplSubst(
    m: Id, x: Id): List[(List[Constraint], Expression)] =
  P.foldRight(Nil: List[(List[Constraint], Expression)]){
    case (MethodImplementation(`m`, y, a, _, e), rst) =>
      (substitute(y, x, a),
       alphaRename(y, x, e)) :: rst
    case (_, rst) => rst
  }
\end{lstlisting}
\caption{Function \scala{mImplSubst}}
\label{fig:scala-mimpl}
\end{figure}\\
%
The implementation of $\mIt{mImplSubst}(m, x)$ is given in \Cref{fig:scala-mimpl}.
In the implementation we fold over declarations of program $P$.
In the case of a method implementation \mImpl{m}{y}{a}{\_}{e},
we substitute the formal argument $y$ with $x$ in constraints $a$
and apply $\alpha$-renaming of $y$ to $x$ in the method body $e$.\\
%
For rule R-New we match for object creations \objConstr{cls}{args},
where $\mIt{args}$ has the type \scala{List[(Id, Expression)]}
and the argument expressions are reduced to values.\\
We generate a fresh variable $x$ for the object to be constructed
and set object $o := (\mIt{cls}, \mIt{args})$, % $o := (\mIt{cls}, \mIt{args.asInstanceOf}[\mIt{List}[(\mIt{Id}, \mIt{Id})]])$,
where \mIt{args} are cast to \scala{List[(Id, Id)]}.
This is a safe typecast, since we ensured
that the argument expressions are values.\\
We search for a constructor \constr{cls}{y}{b} in the program.
If there is no constructor for class \mIt{cls},
we can not reduce \objConstr{cls}{args}.
If we found a constructor,
we substitute the formal argument of the constructor $y$
with the variable of the object to be created $x$
in the constructor constraints $b$.
We obtain $b_1 := \subst{b}{y}{x}$.
We check if the heap constraints combined
with the constraints of $o$ entail $b_1$.
If so, we reduce \objConstr{cls}{args} to $x$
and extend the heap with the mapping from $x$ to $o$.
Otherwise, we can not reduce \objConstr{cls}{args}.\\
\\
We implemented the rules RC-Field, RC-Call and RC-New
by matching for their respective expression structure
and the rules have no requirements on the subexpressions.\\
\\
For field access expressions $e.f$,
we interpret subexpression $e$ in the heap $heap$
to obtain an updated heap $h_1$ and evaluated subexpression $e_1$.
We check if $e = e_1$ and $heap = h_1$.
If not, we proceed with interpreting $e_1.f$ in the heap $h_1$.
Otherwise, the subexpression can not be reduced
and we return $(\mIt{heap}, \mIt{expr})$ to avoid endless recursion.\\
\\
The implementation for method calls $m(e)$ is the same
as the previously shown implementation for field access expressions $e.f$.
We evaluate subexpression $e$ to $e_1$ and proceed
interpreting $m(e_1)$ if we are not stuck.
%
\begin{figure}[h]
\begin{lstlisting}
private def objArgsInterp(
    heap: Heap, args: List[(Id, Expression)])
      : (Heap, List[(Id, Expression)]) = args match {
  case Nil => (heap, Nil)
  case (f, x@Id(_)) :: rst =>
    val (h1, args1) = objArgsInterp(heap, rst)
    (h1, (f, x) :: args1)
  case (f, e) :: rst =>
    val (h1, e1) = interp(heap, e)
    val (h2, args1) = objArgsInterp(h1, rst)
    (h2, (f, e1) :: args1)
}
\end{lstlisting}
\caption{Object Argument Interpretation}
\label{fig:scala-objArgsInterp}
\end{figure}\\
%
The implementation for object constructions \objConstr{cls}{args}
is similar to the implementations for field access expressions $e.f$
and method calls $m(e)$,
but differs from the rule RC-New presented in \Cref{fig:dcc-opsemantics}.
Rule RC-New is defined to only reduce one argument at a time.
Our implementation interprets all the argument expressions in the same step.
This argument interpretation is done via function \scala{objArgsInterp}.
We obtain a new heap $h_1$ and evaluated arguments $args_1$
from the call to \scala{objArgsInterp}.
We check if $\mIt{args} = \mIt{args_1}$ and $\mIt{heap} = h_1$.
If not, we proceed with interpreting \objConstr{cls}{args_1} in heap $h_1$.
Otherwise, we can not reduce \objConstr{cls}{args}.\\
\\
The function \scala{objArgsInterp} is defined in \Cref{fig:scala-objArgsInterp}.
It iterates over each argument, evaluates it
and passes the updated heap to the evaluation of the subsequent arguments
to ensure that the side-effects occurring during evaluation
of the argument expressions are respected.
The function takes a heap $\mIt{heap}$ and a list of tuples of
identifiers and expressions $\mIt{args}$ as arguments.
It returns a heap and a list of tuples of identifiers and expressions.
In the implementation, we match for the structure of the $args$ list.

If the list is empty, we are done and return the input heap $heap$
and the empty list \nil.

If the first element is of the form $(f, x)$ where $x$ is a value,
we continue with interpreting the rest of the list
to obtain a new heap $h_1$ and evaluated arguments $\mIt{args_1}$.
We return the updated heap $h_1$ and evaluated arguments $(f, x) :: \mIt{args_1}$.

If the first element is of the form $(f, e)$,
we interpret $e$ and obtain heap $h_1$ and the evaluated expression $e_1$.
We proceed to interpret the rest of the list in heap $h_1$
to obtain an updated heap $h_2$ and evaluated arguments $\mIt{args_1}$.
We return the latest heap $h_2$ and $(f, e_1) :: \mIt{args_1}$.\\
\\
The \scala{case Id(_) => (heap, expr)} does not implement
a rule from the operational semantics presented in \Cref{fig:dcc-opsemantics}.
We added the case in the implementation,
to avoid that the function has undefined inputs.
Since variables are the values of the language,
we do not need to further evaluate them.
%
% TODO: example
% - prev(Zero) or prev(Succ(Zero))
% - maybe have the parameter be a variable already on the heap
\begin{example}[Interpreting a method call]\quad\\
\label{ex:eval-call}
In this example we demonstrate the process of interpreting expressions.
For this, we take the program describing natural numbers from \Cref{ex:dcc-naturalnumbers}.
We set heap $h := [x \mapsto o]$ with $o := (\texttt{Zero}, [])$, containing
a mapping from variable $x$ to an instance of class \texttt{Zero}.
Our expression to interpret is $\texttt{prev}(x)$.
We call \scala{interp} with arguments $h$ and $\texttt{prev}(x)$.\\
\\
The expression $\texttt{prev}(x)$ is a method call and
$x$ is a value.
We match the implementation of R-Call.
First we build the list of applicable methods.
We call \scala{mImplSubst} to retrieve
the argument constraints and the bodies of
implementations of method $\texttt{prev}$
and substitute the formal parameter
of the implementations with $x$.
Method $\texttt{prev}$ has two implementations
and we obtain
\[ [([\instOf{x}{\texttt{Zero}}], \newInstNoArgs{\texttt{Zero}}),
    ([\instOf{x}{\texttt{Succ}}, \instOf{x.p}{\texttt{Nat}}], x.p)] \]
from \scala{mImplSubst}.
We check for each method implementation $(\ovl a, e)$
if \entails{HC(h)}{\ovl a}.
We call $HC(h)$ to generate the constraints of the heap.
Since $h$ contains only one object mapping from $x$ to $(\texttt{Zero}, [])$,
we call $OC(x, o)$.
From the call to $OC$ we obtain \instBy{x}{\texttt{Zero}}
constraining $x$ to be a direct instance of \texttt{Zero}.
We do not generate further constraints, since object $o$ has no fields.

We call function \scala{entails} to solve the entailment
\entails{\instBy{x}{\texttt{Zero}}}{\instOf{x}{\texttt{Zero}}}
with the result \mIt{true}
as well as the entailment
\entails{\instBy{x}{\texttt{Zero}}}{\instOf{x}{\texttt{Succ}}}
with the result \mIt{false}.
With this, we set the list of applicable methods
\[ S := [([\instOf{x}{\texttt{Zero}}], \newInstNoArgs{\texttt{Zero}})]. \]
We set $(a, e) := S.\mIt{head} = ([\instOf{x}{\texttt{Zero}}], \newInstNoArgs{\texttt{Zero}})$
to be the most specific method
and since $S$ has only one element we do not need to compare
the method against the elements of $S$
and $e$ is the most specific method body.\\
\\
We proceed with interpreting \newInstNoArgs{\texttt{Zero}} in $h$.
We match the implementation of rule R-New.
We generate a fresh variable $y$
and set $o := (\texttt{Zero}, [])$.
We search for a constructor of class $\texttt{Zero}$ in the program
and retrieve \constr{\texttt{Zero}}{x}{\epsilon}.
Since the constraints required by the constructor are empty,
we do not need to check if the heap entails the constructor constraints.
We extend heap $h$ with the new object binding to obtain
\[ h_1 := [x \mapsto (\texttt{Zero}, []), y \mapsto (\texttt{Zero}, [])] \]
and return $(h_1, y)$ as the result of the evaluation of $\texttt{prev}(x)$ in $h$.
We observe that we created a new object of class \texttt{Zero} in $h_1$,
because the body of method \texttt{prev} for \texttt{Zero}
is an object creation instead of a variable access.
\end{example}

%%% Local Variables: 
%%% mode: latex
%%% TeX-master: "../thesis"
%%% End: 
